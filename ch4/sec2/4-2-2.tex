\documentclass[../../solutions]{subfiles}
\title{Template}
\author{}
\begin{document}
\maketitle

% \settheorem{1}{2}{3}
% \begin{lemma}
%
% \end{lemma}
% \popthm

\setexercise{4}{2}{2}
\begin{exercise}
  Explain each step needed to convert the statement of Lemma 4.2.2
  into the statement of the dual Lemma 4.2.3.
\end{exercise}

\begin{proof}
  We state the essence of Lemma 4.2.2 as: Given an adjunction $F\ladj
  G$, there is a natural transformation $\Func{\eta}{1_\sC}{GF}$ whose
  component $\func{\eta_c}{c}{GFc}$ at~$c$ is defined to be the
  transpose of the identity morphism~$1_{Fc}$.

  We can write the adjoint explicitly as
  $$
  \begin{tikzcd}
    \sC
    \ar[r, shift left=1ex, "F"]
    \ar[r, leftarrow, shift right=1ex, "G"']
    \ar[r, phantom, "\scriptstyle\bot"]
    & \sD
  \end{tikzcd}
  \qquad\qquad
  \sD(Fc,d)\cong \sC(c,Gd)
  $$
  Dualising this, and writing $\func{F^{\op}}{\sC^\op}{\sD^\op}$ and
  $\func{G^\op}{\sD^\op}{\sC^\op}$ for clarity, we have
  $$
  \begin{tikzcd}
    \sC^\op
    \ar[r, shift left=1ex, "F^\op"]
    \ar[r, leftarrow, shift right=1ex, "G^\op"']
    \ar[r, phantom, "\scriptstyle\top"]
    & \sD^\op
  \end{tikzcd}
  \qquad\qquad
  \sD^\op(d,F^\op c)\cong \sC^\op(G^\op d,c)
  $$
  Note that in this dualised form, we have $G^\op \ladj F^\op$.  By
  Lemma 4.2.2 applied to this adjunction, there is a natural
  transformation $\Func{\bar{\epsilon}}{1_{\sD^\op}}{F^\op G^\op}$
  whose component $\func{\bar{\epsilon}}{d}{F^\op G^\op d}$ at~$d$ is
  defined to be the transpose of the identity morphism~$1_{G^\op d}$.

  Dualising this result back to our original setting, we obtain a
  natural transformation $\Func{\epsilon}{FG}{1_\sD}$ whose component
  $\func{\epsilon}{FGd}{d}$ at~$d$ is defined to be the transpose of
  the identity morphism~$1_{Gd}$, which is Lemma 4.2.3.
\end{proof}

\end{document}

\documentclass[../../solutions]{subfiles}
\title{Template}
\author{}
\begin{document}
\maketitle

% \settheorem{1}{2}{3}
% \begin{lemma}
%
% \end{lemma}
% \popthm

\setexercise{4}{2}{5}
\begin{exercise}
  A \textbf{morphism of adjunctions} from $F\ladj G$ to $F'\ladj G'$
  is comprised of a pair of functors
  \begin{center}
    \begin{tikzcd}
      \sC \ar[r, "H"]
      \ar[d, shift right=1ex, "F"']
      \ar[d, shift left=1ex, leftarrow, "G"]
      \ar[d, phantom, "\ladj"]
      & \sC'
      \ar[d, shift right=1ex, "F'"']
      \ar[d, shift left=1ex, leftarrow, "G'"]
      \ar[d, phantom, "\ladj"] \\
      \sD \ar[r, "K"']
      & \sD'
    \end{tikzcd}
  \end{center}
  so that the square with the left adjoints and the square with the
  right adjoints both commute (i.e., $KF=F'H$ and $HG=G'K$) and
  satisfying one additional condition, which takes a number of
  equivalent forms.  Prove that the following are equivalent:
  \begin{itemize}
  \item $H\eta=\eta'H$, where $\eta$ and $\eta'$ denote the respective
    units of the adjunctions.
  \item $K\epsilon=\epsilon'K$, where $\epsilon$ and $\epsilon'$
    denote the respective counits of the adjunctions.
  \item Transposition across the adjunctions commutes with application
    of the functors $H$ and $K$, i.e., for every $c\in\sC$ and
    $d\in\sD$, the diagram
    \begin{center}
      \begin{tikzcd}
        \sD(Fc,d) \ar[r, "\cong"] \ar[d, "K"']
        & \sC(c,Gd) \ar[d, "H"] \\
        \sD(KFc,Kd) \ar[d, equal]
        & \sC(Hc,HGd) \ar[d, equal] \\
        \sD'(F'Hc,Kd) \ar[r, "\cong"']
        & \sC'(Hc,G'Kd)
      \end{tikzcd}
    \end{center}
    commutes.
  \end{itemize}
\end{exercise}

\begin{proof}
  (i) $\Leftrightarrow$ (iii).  Writing out the isomorphisms
  explicitly, the diagram in~(iii) becomes
  \begin{center}
    \begin{tikzcd}
      \sD(Fc,d) \ar[r, "G"] \ar[d, "K"']
      & \sC(GFc,Gd) \ar[r, "(\eta_c)^*"]
      & \sC(c,Gd) \ar[d, "H"] \\
      \sD(KFc,Kd) \ar[d, equal]
      && \sC(Hc,HGd) \ar[d, equal] \\
      \sD'(F'Hc,Kd) \ar[r, "G'"']
      & \sC'(G'F'Hc,G'Kd) \ar[r, "(\eta'_{Hc})^*"']
      & \sC'(Hc,G'Kd)
    \end{tikzcd}
  \end{center}
  The diagram commutes if and only if for every
  $f^\sharp\in\sD(Fc,d)$, $H(Gf^\sharp\cdot \eta_c)=G'Kf^\sharp\cdot
  \eta'_{Hc}$, that is, $HGf^\sharp\cdot H\eta_c=G'Kf^\sharp\cdot
  \eta'_{Hc}$.  As $HG=G'K$, this is equivalent to $HGf^\sharp\cdot
  H\eta_c = HGf^\sharp\cdot\eta'_{Hc}$ for all~$f^\sharp$.

  If (i) holds, then by definition of whiskering, this means that
  $H\eta_c=\eta'_{Hc}$ for all $c\in\sC$, so the diagram commutes for
  all $c\in\sC$ and $d\in\sD$, so (iii)~holds.  Conversely, if
  (iii)~holds, then for each $c\in\sC$, we may take $d=Fc$ and
  $f^\sharp=1_{Fc}$, showing that $H\eta_c=\eta'_{Hc}$, so (i)~holds.

  \bigskip
  (ii) $\Rightarrow$ (iii).  The argument is very similar.  We take
  the isomorphisms in the diagram in~(iii) in the opposite direction
  (recalling Lemma 1.5.10), so the diagram in~(iii) commutes if and
  only if the following diagram commutes:
  \begin{center}
    \begin{tikzcd}
      \sD(Fc,d) \ar[r, leftarrow, "(\epsilon_d)_*"] \ar[d, "K"']
      & \sD(Fc,FGd) \ar[r, leftarrow, "F"]
      & \sC(c,Gd) \ar[d, "H"] \\
      \sD(KFc,Kd) \ar[d, equal]
      && \sC(Hc,HGd) \ar[d, equal] \\
      \sD'(F'Hc,Kd) \ar[r, leftarrow, "(\epsilon'_{Kd})_*"']
      & \sD'(F'Hc,F'G'Kd) \ar[r, leftarrow, "F'"']
      & \sC'(Hc,G'Kd)
    \end{tikzcd}
  \end{center}
  This diagram commutes if and only if for every
  $f^\flat\in\sC(c,Gd)$,
  $K\epsilon_d\cdot KFf^\flat = \epsilon'_{Kd}\cdot F'Hf^\flat$, and
  as $F'H=KF$, this is equivalent to
  $K\epsilon_d\cdot KFf^\flat = \epsilon'_{Kd}\cdot KFf^\flat$ for
  all~$f^\flat$.

  If (ii) holds, then $K\epsilon_d=\epsilon'_{Kd}$ for all $d\in\sD$,
  so the diagram commutes for all $c\in\sC$ and $d\in\sD$, so
  (iii)~holds.  Conversely, if (iii)~holds, then for each $d\in\sD$,
  we may take $c=Gd$ and $f^\flat=1_{Gd}$, showing that
  $K\epsilon_d=\epsilon'_{Kd}$, so (ii)~holds.

  Thus all three conditions are equivalent.
\end{proof}

\end{document}

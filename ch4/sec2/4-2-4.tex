\documentclass[../../solutions]{subfiles}
\title{Template}
\author{}
\begin{document}
\maketitle

% \settheorem{1}{2}{3}
% \begin{lemma}
%   
% \end{lemma}
% \popthm

\setexercise{4}{2}{4}
\begin{exercise}
  Each component of the counit of an adjunction is a terminal object
  in some category.  What category?
\end{exercise}

\begin{proof}
  From Proposition 2.4.8 and its proof, we have the following two dual
  results:
  \begin{itemize}
  \item If the covariant set-valued functor $F$ on~$\sC$ is
    representable, with $\isom{\alpha}{\sC(c,-)}{F}$ being a natural
    isomorphism, then the category of elements $\el F$ has an initial
    element $(c,\alpha_c(1_c))$.
  \item If the contravariant set-valued functor $F$ on~$\sC$ is
    representable, with $\isom{\alpha}{\sC(-,c)}{F}$ being a natural
    isomorphism, then the category of elements $\el F$ has a terminal
    element $(c,\alpha_c(1_c))$.
  \end{itemize}

  In our situation, with $\afunc{F}{G}{\sC}{\sD}$ being adjoint
  functors with $F\ladj G$, we have $\sD(F-,d)\cong \sC(-,Gd)$ for
  each $d\in\sD$, so the contravariant functor
  $\func{\sD(F-,d)}{\sC^\op}{\Set}$ is representable.  Thus the
  category of elements $\el \sD(F-,d)$ has a terminal element which is
  $(Gd,\epsilon_d)$, as $\epsilon_d$~is defined to be the transpose of
  the identity on the representing element~$Gd$.  An alternative way
  of describing the category $\el \sD(F-,d)$ is as the comma category
  $F\downarrow d$.

  Dually, as $\sD(Fc,-)\cong \sC(c,G-)$ for each $c\in\sC$, the
  covariant functor $\func{\sC(c,G-)}{\sD}{\Set}$ is representable.
  Thus the category of elements $\el \sC(c,G-)$ has an initial element
  which is $(Fc,\eta_c)$, as $\eta_c$~is the transpose of the identity
  on the representing element~$Fc$.  An alternative way of describing
  the category $\el \sC(c,G-)$ is as the comma category
  $c\downarrow G$.
\end{proof}

\end{document}


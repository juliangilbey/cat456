\documentclass[../../solutions]{subfiles}
\title{Template}
\author{}
\begin{document}
\maketitle

\settheorem{4}{2}{7}
\begin{remark}
  \addcontentsline{toc}{subsection}{Remark 4.2.7}%
  A fully-specified adjunction can be presented in four equivalent
  forms.  It consists of a pair of functors $\afunc{F}{G}{\sC}{\sD}$
  together with one of the following equivalent requirements:
  \begin{enumerate}[label=(\roman*)]
  \item a natural family of isomorphisms $\sD(Fc,d)\cong \sC(c,Gd)$
    for all $c\in\sC$ and $d\in\sD$;
  \item natural transformations $\Func{\eta}{1_\sC}{GF}$ and
    $\Func{\epsilon}{FG}{1_\sD}$ so that $G\epsilon\cdot\eta G=1_G$
    and $\epsilon F\cdot F\eta = 1_F$;
  \item a natural transformation $\Func{\eta}{1_\sC}{GF}$ so that the
    function
    \begin{center}
      \begin{tikzcd}
        \sD(Fc,d) \ar[r, "G"] &
        \sC(GFc,Gd) \ar[r, "(\eta_c)^*"] &
        \sC(c,Gd)
      \end{tikzcd}
    \end{center}
    defines an isomorphism for all $c\in\sC$ and $d\in\sD$;
  \item a natural transformation $\Func{\epsilon}{FG}{1_\sD}$ so that
    the function
    \begin{center}
      \begin{tikzcd}
        \sC(c,Gd) \ar[r, "F"] &
        \sD(Fc,FGd) \ar[r, "(\epsilon_d)_*"] &
        \sD(Fc,d)
      \end{tikzcd}
    \end{center}
    defines an isomorphism for all $c\in\sC$ and $d\in\sD$.
  \end{enumerate}
\end{remark}
\popthm

\begin{proof}[Proof]
  The equivalence of (i) and (ii) was shown in Proposition 4.2.6.

  Suppose (i) and (ii) are true.  Then the proof of Proposition 4.2.6
  shows that the functions given in (iii) and (iv) are inverses of
  each other, and hence define isomorphisms.  (The function in (iii)
  sends $f^\sharp\in\sD(Fc,d)$  to $f^\flat\colon
  c\xrightarrow{\eta_c} GFc\xrightarrow{Gf^\sharp} Gd$, and similarly
  for the function in~(iv), which are the functions in Proposition
  4.2.6.)

  Conversely, suppose that the condition in (iii) holds; we prove that
  the family of isomorphisms is natural in $c$ and~$d$, so that
  (i)~holds.  Let $\func{h}{c'}{c}$ be a morphism in~$\sC$.  We need
  to show that the outer rectangle in the following diagram commutes:
  \begin{center}
    \begin{tikzcd}
      \sD(Fc,d) \ar[d, "Fh^*"'] \ar[r, "G"] &
      \sC(GFc,Gd) \ar[r, "(\eta_c)^*"] \ar[d, "GFh^*"] &
      \sC(c,Gd) \ar[d, "h^*"] \\
      \sD(Fc',d) \ar[r, "G"'] &
      \sC(GFc',Gd) \ar[r, "(\eta_{c'})^*"'] &
      \sC(c',Gd)
    \end{tikzcd}
  \end{center}
  Suppose $f^\sharp\in\sD(Fc,d)$.  Then the left hand square commutes
  because $Gf^\sharp\cdot GFh=G(f^\sharp\cdot Fh)$ and the right hand
  square commutes because $Gf^\sharp\cdot \eta_c\cdot h=Gf^\sharp
  \cdot GFh\cdot \eta_{c'}$ by the naturality of~$\eta$.

  For naturality in $d$, suppose $\func{h}{d}{d'}$.  We need to show
  that the outer rectangle in this diagram commutes:
  \begin{center}
    \begin{tikzcd}
      \sD(Fc,d) \ar[d, "h_*"'] \ar[r, "G"] &
      \sC(GFc,Gd) \ar[r, "(\eta_c)^*"] \ar[d, "Gh_*"] &
      \sC(c,Gd) \ar[d, "Gh_*"] \\
      \sD(Fc,d') \ar[r, "G"'] &
      \sC(GFc,Gd') \ar[r, "(\eta_c)^*"'] &
      \sC(c,Gd')
    \end{tikzcd}
  \end{center}
  Suppose $f^\sharp\in\sD(Fc,d)$.  Then the left hand square commutes
  because $Gh\cdot Gf^\sharp=G(h\cdot f^\sharp)$ and the right hand
  square commutes because pre-composition with $\eta_c$ and
  post-composition with $Gh$ commute: they both give $Gh\cdot
  Gf^\sharp\cdot \eta_c$.

  Therefore condition (iii) implies condition (i), and similarly
  condition (iv) implies condition (i), showing that all four
  conditions are equivalent.
\end{proof}

\end{document}


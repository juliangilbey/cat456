\documentclass[../../solutions]{subfiles}
\title{Template}
\author{}
\begin{document}
\maketitle

% \settheorem{1}{2}{3}
% \begin{lemma}
%
% \end{lemma}
% \popthm

\setexercise{4}{2}{3}
\begin{exercise}
  Pick your favorite forgetful functor from Example 4.1.10 and prove
  that it is a right adjoint by defining its left adjoint, the unit,
  and the counit, and demonstrating that the triangle identities
  hold.
\end{exercise}

\begin{proof}
  We show this for examples (i)--(vi) from Example 4.1.10; we state
  the example at the start of each one, and name the left adjoint
  as~$F$ in each case.

  \begin{enumerate}[label=(\roman*)]
  % (i)
  \item $\func{U}{\Setp}{\Set}$.  The left adjoint carries a set $X$
    to the pointed set $FX=X_+:= X\sqcup\{X\}$, with a freely-adjoined
    basepoint.

    Taking $X\in\Set$ and $Y_*\in\Setp$, we have
    $\Setp(FX, Y_*)\cong \Set(X,UY_*)$, with the isomorphism taking a
    based function $\func{f^\sharp}{FX}{Y_*}$ to the function
    $\func{f^\flat}{X}{UY_*}$ which just ignores the basepoint.

    We then have
    \begin{center}
      \begin{tikzcd}[row sep=0.2em]
        X \ar[r, "\eta_X"] & UFX\\
        x \ar[r, mapsto] & x
      \end{tikzcd}
    \end{center}
    where $UFX=X\sqcup\{X\}$ as a set, and writing $Y:=UY_*$,
    \begin{center}
      \begin{tikzcd}[row sep=0.2em]
        FUY_* \ar[r, "\epsilon_{Y_*}"] & Y_* \\
        y \ar[r, mapsto] & y \\
        \{Y\} \ar[r, mapsto] & *
      \end{tikzcd}
    \end{center}

    We then obtain the triangle identities:
    \begin{center}
      \begin{tikzcd}[row sep=0.2em]
        FX \ar[r, "F\eta_X"] & FUFX \ar[r, "\epsilon_{FX}"] & FX \\
        x \ar[r, mapsto] & x \ar[r, mapsto] & x \\
        \{X\} \ar[r, mapsto] & \{X\sqcup\{X\}\} \ar[r, mapsto] & \{X\}
      \end{tikzcd}
    \end{center}
    as the basepoint of $FUFX$ is mapped to the basepoint of~$FX$,
    hence $\epsilon_{FX}\cdot F\eta_X=1_{FX}$, which is the first
    triangle identity.

    Similarly,
    \begin{center}
      \begin{tikzcd}[row sep=0.2em]
        UY_* \ar[r, "\eta_{UY_*}"] & UFUY_* \ar[r, "U\epsilon_{Y_*}"]
        & UY_* \\
        y \ar[r, mapsto] & y \ar[r, mapsto] & y
      \end{tikzcd}
    \end{center}
    hence $U\epsilon_{Y_*}\cdot \eta_{UY_*}=1_{UY_*}$, which is the
    second triangle identity.

  % (ii)
  \item $\func{U}{\Monoid}{\Set}$.  The free monoid on a set $X$
    is the set $FX=\coprod_{n\ge0} X^{\times n}$ of finite lists of
    elements of~$X$, including the empty list.

    Composition in $FX$ is concatenation of lists.

    Letting $X\in\Set$ and $M\in\Monoid$, we have
    $\Monoid(FX, M)\cong \Set(X,UM)$, with the isomorphism taking a
    morphism $\func{f^\sharp}{FX}{M}$ to the function
    $\func{f^\flat}{X}{UM}$ which maps $x\in X$ to $f^\sharp((x))$
    regarded as an element of the underlying set of~$M$.

    We then have
    \begin{center}
      \begin{tikzcd}[row sep=0.2em]
        X \ar[r, "\eta_X"] & UFX\\
        x \ar[r, mapsto] & (x)
      \end{tikzcd}
    \end{center}
    where $(x)\in X^{\times1}\subset \coprod_{n\ge0} X^{\times n}$,
    regarded as a set, and
    \begin{center}
      \begin{tikzcd}[row sep=0.2em]
        FUM \ar[r, "\epsilon_M"] & M \\
        (m_1, m_2, \dots, m_k) \ar[r, mapsto] & m_1m_2\dots m_k
      \end{tikzcd}
    \end{center}
    where the multiplication is in $M$.

    We then obtain the triangle identities:
    \begin{center}
      \begin{tikzcd}[row sep=0.2em]
        FX \ar[r, "F\eta_X"] & FUFX \ar[r, "\epsilon_{FX}"] & FX \\
        (x_1,x_2,\dots,n_k) \ar[r, mapsto] &
        ((x_1), (x_2), \dots, (x_k)) \ar[r, mapsto] &
        (x_1,x_2,\dots,n_k)
      \end{tikzcd}
    \end{center}
    since $(x_1)(x_2)\dots(x_k)=(x_1,x_2,\dots,x_k)$ in $FX$, hence
    $\epsilon_{FX}\cdot F\eta_X=1_{FX}$, demonstrating the first
    triangle identity.

    Similarly,
    \begin{center}
      \begin{tikzcd}[row sep=0.2em]
        UM \ar[r, "\eta_{UM}"] & UFUM \ar[r, "U\epsilon_M"] & UM \\
        m \ar[r, mapsto] & (m) \ar[r, mapsto] & m
      \end{tikzcd}
    \end{center}
    hence $U\epsilon_M\cdot \eta_{UM}=1_{UM}$, demonstrating the
    second triangle identity.

  % (iii)
  \item $\func{U}{\Ring}{\Ab}$, forgetting the multiplicative
    structure.  The free ring on an abelian group~$A$
    is $FA=\bigoplus_{n\ge0} A^{\otimes n}$.  (Note that $A^{\otimes
      0}=\ZZ$.)

    Addition is componentwise, and multiplication is given by
    $$(a_1\otimes \dots \otimes a_r)(b_1\otimes \dots \otimes b_s) =
    a_1\otimes \dots \otimes a_r \otimes b_1 \otimes \dots \otimes
    b_s$$
    and then extended by linearity.  (More precisely, there is a
    natural isomorphism $A^{\otimes r}\otimes A^{\otimes s}\cong
    A^{\otimes(r+s)}$, and that is what is used to define the
    product.)

    Letting $A\in\Ab$ and $R\in\Ring$, we have
    $\Ring(FA, R)\cong \Ab(A,UR)$, with the isomorphism taking a
    morphism $\func{f^\sharp}{FA}{R}$ to the morphism
    $\func{f^\flat}{A}{UR}$ which maps $a\in A$ to $f^\sharp(a)$, with
    $a\in A^{\otimes 1}$, regarded as an element of the underlying
    abelian group of~$R$.

    We then have
    \begin{center}
      \begin{tikzcd}[row sep=0.2em]
        A \ar[r, "\eta_A"] & UFA\\
        a \ar[r, mapsto] & a
      \end{tikzcd}
    \end{center}
    where the image is $a\in A^{\otimes1}$ in the direct sum, and
    \begin{center}
      \begin{tikzcd}[row sep=0.2em]
        FUR \ar[r, "\epsilon_R"] & R \\
        r_1 \otimes r_2 \otimes \dots \otimes r_k \ar[r, mapsto]
        & r_1r_2\dots r_k
      \end{tikzcd}
    \end{center}
    and extends by linearity.

    We then obtain the triangle identities:
    \begin{center}
      \begin{tikzcd}[row sep=0.2em]
        FA \ar[r, "F\eta_A"] & FUFA \ar[r, "\epsilon_{FA}"] & FA \\
        a_1 \otimes a_2 \otimes \dots \otimes a_k \ar[r, mapsto] &
        a_1 \otimes a_2 \otimes \dots \otimes a_k \ar[r, mapsto] &
        a_1 \otimes a_2 \otimes \dots \otimes a_k
      \end{tikzcd}
    \end{center}
    where the middle term lies in $(UFA)^{\otimes1}$, so
    $\epsilon_{FA}\cdot F\eta_A=1_{FA}$, demonstrating the first
    triangle identity.

    Similarly,
    \begin{center}
      \begin{tikzcd}[row sep=0.2em]
        UR \ar[r, "\eta_{UR}"] & UFUR \ar[r, "U\epsilon_R"] & UR \\
        r \ar[r, mapsto] & r \ar[r, mapsto] & r
      \end{tikzcd}
    \end{center}
    where the middle term lies in $(UR)^{\otimes1}$, so
    $U\epsilon_R\cdot \eta_{UR}=1_{UR}$, demonstrating the second
    triangle identity.

  % (iv)
  \item $\func{U}{\Ab}{\Set}$.  The free abelian group on a set~$X$ is
    the set $FX=\ZZ[X]:=\bigoplus_X \ZZ$ of finite formal sums of
    elements of~$X$ with integer coefficients.

    This is identical to the more general (v), just writing $\ZZ$ in
    place of~$R$.

  % (v)
  \item  $\func{U}{\Mod{R}}{\Set}$.  The free $R$-module on a set~$X$ is
    $FX=R[X]:=\bigoplus_X R$ of finite formal sums of
    elements of~$X$ with coefficients in~$R$.  A special case defines
    the free $\mathbb{k}$-vector space on a set, defining the left
    adjoint to $\func{U}{\Vect{\mathbb{k}}}{\Set}$ considered in the
    introduction to this chapter.

    Here, addition is componentwise, and we write
    $\sum_{x\in X} \beta_x x$ for such a finite formal sum.

    Letting $X\in\Set$ and $M\in\Mod{R}$, we have
    $\Mod{R}(FX, M)\cong \Set(X,UM)$, with the isomorphism taking an
    $R$-module homomorphism $\func{f^\sharp}{FX}{M}$ to the function
    $\func{f^\flat}{X}{UM}$ which maps $x\in X$ to $f^\sharp(1x)$,
    regarded as an element of the underlying set of~$M$.

    We then have
    \begin{center}
      \begin{tikzcd}[row sep=0.2em]
        X \ar[r, "\eta_X"] & UFX\\
        x \ar[r, mapsto] & 1x
      \end{tikzcd}
    \end{center}
    where the image is the sum $\sum_{x'\in X} \beta_{x'} x'$ with
    $\beta_x=1$, $\beta_{x'}=0$ otherwise, and this sum is taken in
    the underlying set of~$FX$.  We also have
    \begin{center}
      \begin{tikzcd}[row sep=0.2em]
        FUM \ar[r, "\epsilon_M"] & M \\
        \sum_{m\in M} \beta_m m \ar[r, mapsto]
        & \sum_{m\in M} \beta_m m
      \end{tikzcd}
    \end{center}
    where the sum in the image is interpreted in $M$.  (In the case of
    abelian groups of example~(iv), this sum would be interpreted in
    the abelian group regarded as a $\ZZ$-module for the purposes of
    the scalar multiplication.)

    We then obtain the triangle identities:
    \begin{center}
      \begin{tikzcd}[row sep=0.2em]
        FX \ar[r, "F\eta_X"] & FUFX \ar[r, "\epsilon_{FX}"] & FX \\
        \sum_{x\in X} \beta_x x \ar[r, mapsto] &
        1\cdot \sum_{x\in X} \beta_x x \ar[r, mapsto] &
        \sum_{x\in X} \beta_x x
      \end{tikzcd}
    \end{center}
    where the middle term is a formal sum in $\bigoplus_{UFX}R$, so
    $\epsilon_{FX}\cdot F\eta_X=1_{FX}$, demonstrating the first
    triangle identity.

    Similarly,
    \begin{center}
      \begin{tikzcd}[row sep=0.2em]
        UM \ar[r, "\eta_{UM}"] & UFUM \ar[r, "U\epsilon_M"] & UM \\
        m \ar[r, mapsto] & 1m \ar[r, mapsto] & m
      \end{tikzcd}
    \end{center}
    where the middle term lies in the underlying set of
    $\bigoplus_M R$, so $U\epsilon_M\cdot \eta_{UM}=1_{UM}$,
    demonstrating the second triangle identity.

  % (vi)
  \item $\func{U}{\Ring}{\Set}$.  By composing the left adjoints to
    the forgetful functors $\Ring\to \Ab\to \Set$, the free ring on a
    set~$X$ is the free monoid on the free abelian group on the set:
    $FX=\bigoplus_{n\ge0}\bigl(\bigoplus_X \ZZ\bigr)^{\otimes n}$.

    More details about the composition of adjoint functions are given
    in Section 4.4, in particular in Proposition 4.4.4.  But here we
    will not make use of the results there, and work directly with the
    functors given here.

    Letting $X\in\Set$ and $R\in\Ring$, we have
    $\Ring(FX, R)\cong \Set(X,UR)$, with the isomorphism taking a
    morphism $\func{f^\sharp}{FX}{R}$ to the function
    $\func{f^\flat}{X}{UR}$ which maps $x\in X$ to $f^\sharp(1x)$,
    regarded as an element of the underlying set of~$R$, where $1x\in
    \bigl(\bigoplus_X \ZZ\bigr)^{\otimes 1}$.

    We then have
    \begin{center}
      \begin{tikzcd}[row sep=0.2em]
        X \ar[r, "\eta_X"] & UFX\\
        x \ar[r, mapsto] & 1x
      \end{tikzcd}
    \end{center}
    where $1x\in \bigl(\bigoplus_X \ZZ\bigr)^{\otimes 1}$, regarded as
    an element of the underlying set of~$FX$, and
    \begin{center}
      \begin{tikzcd}[column sep=0em]
        FUR \ar[d, "\epsilon_R"'] & \in &
        \sum\limits_{R} \beta_r^{(1)}r \otimes \sum\limits_{R}
        \beta_r^{(2)}r \otimes \dots \otimes \sum\limits_{R}
        \beta_r^{(k)}r \ar[d, mapsto] \\
        R & \in &
        \left(\sum\limits_{R} \beta_r^{(1)}r\right)\left(\sum\limits_{R}
          \beta_r^{(2)}r\right) \dots \left(\sum\limits_{R}
          \beta_r^{(k)}r\right)
      \end{tikzcd}
    \end{center}
    where the product and sums in the image are evaluated in~$R$, and
    each term $\beta_r^{(i)}r$ is evaluated by regarding $R$ as a
    $\ZZ$-module.  The definition of $\epsilon_R$ then extends by
    linearity to all of $FUR$.

    We now obtain the triangle identities:
    \begin{center}
      \begin{tikzcd}[column sep=0em]
        FX \ar[d, "F\eta_X"'] & \in &
        \sum\limits_X \beta_x^{(1)}x \otimes \dots \otimes
        \sum\limits_X \beta_x^{(k)}x \ar[d, mapsto] \\
        FUFX \ar[d, "\epsilon_{FX}"'] & \in &
        1\cdot\left(\sum\limits_X \beta_x^{(1)}x \otimes \dots
          \otimes \sum\limits_X \beta_x^{(k)}x\right) \ar[d, mapsto] \\
        FX & \in &
        \sum_{X} \beta_x^{(1)}x \otimes \dots \otimes
        \sum_{X} \beta_x^{(k)}x
      \end{tikzcd}
    \end{center}
    where the middle term lies in
    $\bigl(\bigoplus_{FX}\ZZ\bigr)^{\otimes1}$, so
    $\epsilon_{FX}\cdot F\eta_X=1_{FX}$, demonstrating the first
    triangle identity.

    Similarly,
    \begin{center}
      \begin{tikzcd}[row sep=0.2em]
        UR \ar[r, "\eta_{UR}"] & UFUR \ar[r, "U\epsilon_R"] & UR \\
        r \ar[r, mapsto] & 1r \ar[r, mapsto] & r
      \end{tikzcd}
    \end{center}
    where the middle term lies in
    $\bigl(\bigoplus_{UR}\ZZ\bigr)^{\otimes1}$, so
    $U\epsilon_R\cdot \eta_{UR}=1_{UR}$, demonstrating the second
    triangle identity.
  \end{enumerate}
\end{proof}

\end{document}

\documentclass[../../solutions]{subfiles}
\title{Template}
\author{}
\begin{document}
\maketitle

% \settheorem{1}{2}{3}
% \begin{lemma}
%
% \end{lemma}
% \popthm

\setexercise{4}{3}{1}
\begin{exercise}
  Dualize Definition 4.2.5 to define mutual left adjoints and mutual
  right adjoints as a pair of contravariant functors equipped with
  appropriate natural transformations.
\end{exercise}

\begin{proof}
  Definition 4.2.5 reads:
  \begingroup
  \renewcommand{\theHtheorem}{\theHsection.\arabic{theorem}.book}
  \settheorem{4}{2}{5}
  \begin{definition}
    An adjunction consists of an opposing pair of functors
    $\afunc{F}{G}{\sC}{\sD}$, together with natural transformations
    $\Func{\eta}{1_\sC}{GF}$ and $\Func{\epsilon}{FG}{1_\sD}$ that
    satisfy the triangle identities:
    \begin{center}
      \begin{tikzcd}
        F \ar[r, Rightarrow, "F\eta"] \ar[dr, Rightarrow, "1_F"']
        & FGF \ar[d, Rightarrow, "\epsilon F"]
        & G \ar[r, Rightarrow, "\eta G"] \ar[dr, Rightarrow, "1_G"']
        & GFG \ar[d, Rightarrow, "G\epsilon"] \\
        & F && G
      \end{tikzcd}
    \end{center}
  \end{definition}
  \popthm
  \endgroup

  We now wish to dualise this to mutually adjoint left or right
  functors.  Let us start with the case of mutually adjoint left
  functors.  We have $\func{F}{\sC^\op}{\sD}$ and
  $\func{G}{\sD^\op}{\sC}$ with isomorphisms
  $$\sD(Fc,d)\cong \sC(Gd,c)$$
  for all $c\in\sC$ and $d\in\sD$, natural in both variables.

  Before we embark on a conversion of Definition 4.2.5, let us just
  clarify what naturality of this collection of isomorphisms means.
  Naturality in~$c$ means that for any $\func{h}{c}{c'}$ in~$\sC^\op$,
  so $\func{Fh}{Fc'}{Fc}$, the following diagram commutes:
  \begin{center}
    \begin{tikzcd}
      \sD(Fc,d) \ar[r, "\cong"] \ar[d, "(Fh)^*"']
      & \sC(Gd,c) \ar[d, "h_*"] \\
      \sD(Fc',d) \ar[r, "\cong"']
      & \sC(Gd,c')
    \end{tikzcd}
  \end{center}
  Similarly, naturality in~$d$ means that for any $\func{k}{d}{d'}$
  in~$\sD$, so $\func{Gk}{Gd'}{Gd}$, the following diagram commutes:
  \begin{center}
    \begin{tikzcd}
      \sD(Fc,d) \ar[r, "\cong"] \ar[d, "k_*"']
      & \sC(Gd,c) \ar[d, "(Gk)^*"] \\
      \sD(Fc,d') \ar[r, "\cong"']
      & \sC(Gd',c)
    \end{tikzcd}
  \end{center}

  Returning to Definition 4.2.5, to avoid confusion, let us take
  $\sE:=\sC^\op$, and write $\func{\bar F}{\sE}{\sD}$ and
  $\func{\bar G}{\sD}{\sE}$ for $F$ and~$G$ thought of as
  contravariant functors between $\sD$ and~$\sE$.  We then have
  $\sD(\bar Fe,d)\cong \sE(e,\bar Gd)$ for all $e\in\sE$ and
  $d\in\sD$, natural in both variables, so $\bar F\ladj \bar G$.  We
  thus have natural transformations
  $\Func{\bar\eta}{1_\sE}{\bar G \bar F}$ and
  $\Func{\bar\epsilon}{\bar F \bar G}{1_\sD}$ satisfying the triangle
  identities:
  \begin{center}
    \begin{tikzcd}
      \bar F \ar[r, Rightarrow, "\bar F\bar\eta"] \ar[dr, Rightarrow,
      "1_{\bar F}"']
      & \bar F\bar G\bar F \ar[d, Rightarrow, "\bar\epsilon \bar F"]
      & \bar G \ar[r, Rightarrow, "\bar\eta \bar G"]
      \ar[dr, Rightarrow, "1_{\bar G}"']
      & \bar G\bar F\bar G \ar[d, Rightarrow, "\bar G\bar\epsilon"] \\
      & \bar F && \bar G
    \end{tikzcd}
  \end{center}
  Let us now translate these identities back from $\sE=\sC^\op$
  to~$\sC$, writing $\eta$ and~$\epsilon$ for the natural
  transformations $\bar\eta$ and~$\bar\epsilon$ expressed in terms of
  $\sC$ and~$\sD$.  We present the individual components of the
  translation in a table to make the steps clear.
  \begin{center}
    \renewcommand{\arraystretch}{1.4}
    \begin{tabular}{c|c}
      In $\sE=\sC^\op$ and $\sD$
          & In $\sC$ and $\sD$ \\ \hline
      $\func{\bar\eta_e}{e}{\bar G\bar F e}$ in $\sE$
          & $\func{\eta_c}{GFc}{c}$ in $\sC$ \\
      $\func{\bar\epsilon_d}{\bar F\bar G d}{d}$ in $\sD$
          & $\func{\epsilon_d}{FGd}{d}$ in $\sD$ \\
      $\func{(\bar F\bar\eta)_e}{\bar F e}{\bar F \bar G \bar F e}$ in
      $\sD$
          & $\func{(F\eta)_c}{Fc}{FGFc}$ in $\sD$ \\[-5pt]
          & because $F$ is contravariant \\
      $\func{(\bar\epsilon\bar F)_e}{\bar F \bar G \bar F e}{\bar F
      e}$ in $\sD$
          & $\func{(\epsilon F)_c}{FGFc}{Fc}$ in $\sD$ \\[-5pt]
          & because $(\epsilon F)_c=\epsilon_{Fc}$ by definition \\
      $(\bar\epsilon \bar F)_e\cdot (\bar F\bar\eta)_e = 1_{\bar F e}$
      in $\sD$
          & $(\epsilon F)_c\cdot (F\eta)_c = 1_{Fc}$ in $\sD$ \\
      $\func{(\bar\eta\bar G)_d}{\bar G d}{\bar G \bar F \bar G d}$ in
      $\sE$
          & $\func{(\eta G)_d}{GFGd}{Gd}$ in $\sC$ \\
      $\func{(\bar G\bar\epsilon)_d}{\bar G \bar F \bar G d}{\bar G
      d}$ in $\sE$
          & $\func{(G\epsilon)_d}{Gd}{GFGd}$ in $\sC$ \\[-5pt]
          & because $G$ is contravariant \\
      $(\bar G\bar\epsilon)_d\cdot (\bar\eta \bar G)_d = 1_{\bar G d}$
      in $\sE$
          & $(\eta G)_d\cdot (G\epsilon)_d = 1_{Gd}$ in $\sC$
    \end{tabular}
  \end{center}
  Therefore the definition becomes:
  \begingroup
  \renewcommand{\theHtheorem}{\theHsection.\arabic{theorem}.left}
  \settheorem{4}{2}{5}
  \begin{definition}[mutually left adjoint version]
    A pair of mutually left adjoint functors are contravariant
    functors $\func{F}{\sC^\op}{\sD}$ and $\func{G}{\sD^\op}{\sC}$
    together with natural transformations
    $\Func{\eta}{GF}{1_\sC}$ and $\Func{\epsilon}{FG}{1_\sD}$ that
    satisfy the triangle identities:
    \begin{center}
      \begin{tikzcd}
        F \ar[r, Rightarrow, "F\eta"] \ar[dr, Rightarrow, "1_F"']
        & FGF \ar[d, Rightarrow, "\epsilon F"]
        & G \ar[r, Leftarrow, "\eta G"] \ar[dr, Leftarrow, "1_G"']
        & GFG \ar[d, Leftarrow, "G\epsilon"] \\
        & F && G
      \end{tikzcd}
    \end{center}
  \end{definition}
  \popthm
  \endgroup

  Let us note that these triangle inequalities are symmetric, in the
  sense that swapping $F$ and~$\eta$ with $G$ and~$\epsilon$ does not
  change them: the pair still read $(\epsilon F)\cdot(F\eta)=1_F$ and
  $(\eta G)\cdot(G\epsilon)=1_G$.  This is to be expected, as the
  roles of $F$ and~$G$ are now perfectly symmetric, unlike the case of
  a normal adjunction.

  \bigskip

  Let us now repeat this for mutually right adjoint functors.  This
  time, we replace $\sD$ by $\sD^\op$.  For brevity, we just state the
  results rather than work through all the details, which are the dual
  of the above.

  The functors $\func{F}{\sC^\op}{\sD}$ and $\func{G}{\sD^\op}{\sC}$
  are mutually right adjoint functors if there are isomorphisms
  $$\sD(d,Fc)\cong \sC(c,Gd)$$
  for all $c\in\sC$ and $d\in\sD$, natural in both variables.

  Again, we clarify what naturality of this collection of isomorphisms
  means.  Naturality in~$c$ means that for any $\func{h}{c'}{c}$
  in~$\sC$, so $\func{Fh}{Fc}{Fc'}$, the following diagram commutes:
  \begin{center}
    \begin{tikzcd}
      \sD(d,Fc) \ar[r, "\cong"] \ar[d, "(Fh)_*"']
      & \sC(c,Gd) \ar[d, "h^*"] \\
      \sD(d,Fc') \ar[r, "\cong"']
      & \sC(c',Gd)
    \end{tikzcd}
  \end{center}
  Similarly, naturality in~$d$ means that for any $\func{k}{d'}{d}$
  in~$\sD$, so $\func{Gk}{Gd}{Gd'}$, the following diagram commutes:
  \begin{center}
    \begin{tikzcd}
      \sD(d,Fc) \ar[r, "\cong"] \ar[d, "k^*"']
      & \sC(c,Gd) \ar[d, "(Gk)_*"] \\
      \sD(d',Fc) \ar[r, "\cong"']
      & \sC(c,Gd')
    \end{tikzcd}
  \end{center}

  Definition 4.2.5 becomes:
  \begingroup
  \renewcommand{\theHtheorem}{\theHsection.\arabic{theorem}.right}
  \settheorem{4}{2}{5}
  \begin{definition}[mutually right adjoint version]
    A pair of mutually right adjoint functors are contravariant
    functors $\func{F}{\sC^\op}{\sD}$ and $\func{G}{\sD^\op}{\sC}$
    together with natural transformations
    $\Func{\eta}{1_\sC}{GF}$ and $\Func{\epsilon}{1_\sD}{FG}$ that
    satisfy the triangle identities:
    \begin{center}
      \begin{tikzcd}
        F \ar[r, Leftarrow, "F\eta"] \ar[dr, Leftarrow, "1_F"']
        & FGF \ar[d, Leftarrow, "\epsilon F"]
        & G \ar[r, Rightarrow, "\eta G"] \ar[dr, Rightarrow, "1_G"']
        & GFG \ar[d, Rightarrow, "G\epsilon"] \\
        & F && G
      \end{tikzcd}
    \end{center}
  \end{definition}
  \popthm
  \endgroup

  See also Section~\ref{ssec:contravariant-extension} in these
  solutions for the dualisations of Lemma 4.1.3 and Proposition 4.2.6.
\end{proof}

\end{document}

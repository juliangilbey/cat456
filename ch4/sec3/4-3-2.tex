\documentclass[../../solutions]{subfiles}
\title{Template}
\author{}
\begin{document}
\maketitle

% \settheorem{1}{2}{3}
% \begin{lemma}
%   
% \end{lemma}
% \popthm

\setexercise{4}{3}{2}
\begin{exercise}
  Prove Proposition 4.3.6.
\end{exercise}

\begin{proof}
  We begin by presenting a slight generalisation of Proposition 4.3.4.
  We note that the nature of the set $\sB(Fa,b)$ on the left of the
  isomorphism in that proposition is not really used beyond the fact
  that it is a functor of the two variables $a$ and~$b$.  We can
  therefore generalise the proposition as follows:

  \settheorem{4}{3}{4}
  \begingroup
  \renewcommand{\thetheorem}{\thesection.$4'$}
  \begin{proposition}
    Consider a functor $\func{K}{\sD^\op\times\sE}{\Set}$ so that for
    each $e\in\sE$ there exists an object $Le\in\sD$ together with an
    isomorphism
    \begin{equation}
      \label{4-3-2-prop}
      K(d,e)\cong \sD(d,Le),\qquad \text{natural in $d\in\sD$.}
    \end{equation}
    There there exists a unique way to extend the assignment
    $\func{L}{\ob\sE}{\ob\sD}$ to a functor $\func{L}{\sE}{\sD}$ so
    that the family of isomorphisms \eqref{4-3-2-prop} is also natural
    in $e\in\sE$.
  \end{proposition}
  \endgroup
  \popthm

  The proof is almost identical to that of the original proposition;
  we just need to modify the first couple of sentences as follows.

  Naturality of \eqref{4-3-2-prop} in $\func{f}{e}{e'}$ demands that
  the function
  $$\sD(d,Le)\cong K(d,e) \xrightarrow{K(1_d,f)} K(d,e') \cong
  \sD(d,Le')$$
  equals post-composition by the yet-to-be-defined morphism
  $\func{Lf}{Le}{Le'}$.  We claim that this composite function defines
  a natural transformation $\sD(-,Le)\Rightarrow \sD(-,Le')$.  Suppose
  $\func{h}{d'}{d}$.  Then we have the following diagram:
  \begin{center}
    \begin{tikzcd}
      \sD(d,Le) \ar[r, "\cong"] \ar[d, "h^*"']
      & K(d,e) \ar[r, "{K(1_d,f)}"] \ar[d, "{K(h,1_e)}"]
      & K(d,e') \ar[r, "\cong"] \ar[d, "{K(h,1_{e'})}"]
      & \sD(d,Le') \ar[d, "h^*"] \\
      \sD(d',Le) \ar[r, "\cong"']
      & K(d',e) \ar[r, "{K(1_{d'},f)}"']
      & K(d',e') \ar[r, "\cong"']
      & \sD(d',Le')
    \end{tikzcd}
  \end{center}
  The left and right squares commute by the naturality of the
  isomorphisms in~$d$.  The middle square commutes because both paths
  compose to give $K(h,f)$.  Therefore the whole rectangle commutes
  and the composite function gives a natural transformation
  $\sD(-,Le)\Rightarrow \sD(-,Le')$ as claimed.

  The rest of the proof of this proposition continues exactly as in
  the proof of the original Proposition 4.3.4.

  We can now prove Proposition 4.3.6, which we quote in stages.

  \settheorem{4}{3}{6}
  \begin{proposition}
    Suppose that $\func{F}{\sA\times\sB}{\sC}$ is a bifunctor so that
    for each object $a\in\sA$, the induced functor
    $\func{F(a,-)}{\sB}{\sC}$ admits a right adjoint
    $\func{G_a}{\sC}{\sB}$.  Then:
    \begin{enumerate}[label=(\roman*)]
    \item These right adjoints assemble into a unique bifunctor
      $\func{G}{\sA^\op\times\sC}{\sB}$, defined so that
      $G(a,c)=G_a(c)$ and so that the isomorphisms
      $$\sC(F(a,b),c)\cong\sB(b,G(a,c))$$
      are natural in all three variables.
    \end{enumerate}
  \end{proposition}
  \popthm

  The isomorphisms are natural in $b$ and $c$ by the definition of
  adjoints, so we only need to show that they are natural in~$a$.  In
  our above Proposition 4.3.$4'$, take $\sD=\sB$,
  $\sE=\sA^\op$, $K_c(b,a)=\sC(F(a,b),c)$ and $L_ca=G(a,c)$.  Then our
  isomorphisms can be expressed as $K_c(b,a)\cong \sB(b,L_ca)$,
  natural in~$b$, so there is a unique way to extend the assignment
  $\func{L_c}{\ob\sA^\op}{\ob\sB}$ to a functor
  $\func{L_c}{\sA^\op}{\sB}$ so that the family of isomorphisms is
  natural in~$a$.  This then gives us a unique functor
  $\func{G}{\sA^\op\times\sC}{\sB}$ making the isomorphisms natural in
  all three variables.  (To be more explicit: $G(a,c)$ is already
  defined on objects.  On a morphism $\func{k}{c}{c'}$, we have
  $G(1_a,k)=G_a(k)$ and on a morphism $\func{h}{a}{a'}$, we have
  $G(h,1_c)=L_c(h)$.)

  \settheorem{4}{3}{6}
  \begin{proposition}[continued]
    \leavevmode\\
    If furthermore for each $b\in\sB$, the induced functor
    $\func{F(-,b)}{\sA}{\sC}$ admits a right adjoint
    $\func{H_b}{\sC}{\sA}$, then:
    \begin{enumerate}[label=(\roman*), start=2]
    \item There is a unique bifunctor
      $\func{H}{\sB^\op\times\sC}{\sA}$ defined so that
      $H(b,c)=H_b(c)$ and the isomorphisms
      $$\sC(F(a,b),c)\cong \sB(b,G(a,c)) \cong \sA(a,H(b,c))$$
      are natural in all three variables.
    \item In this case, for each $c\in\sC$, the functors
      $\func{G(-,c)}{\sA^\op}{\sB}$ and $\func{H(-,c)}{\sB^\op}{\sA}$
      are mutual right adjoints.
    \end{enumerate}
  \end{proposition}
  \popthm

  The naturality of the first isomorphism was shown in~(i).  The
  naturality of the isomorphisms $\sC(F(a,b),c)\cong \sA(a,H(b,c))$
  follows exactly as in~(i); we have just swapped the roles of
  $a\in\sA$ and $b\in\sB$.

  Given these isomorphisms, (iii) immediately follows from the natural
  isomorphisms $\sB(b,G(a,c)) \cong \sA(a,H(b,c))$ by the definition
  of mutual right adjoint functors.
\end{proof}

\end{document}


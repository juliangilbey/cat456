\documentclass[../../solutions]{subfiles}
\title{Template}
\author{}
\begin{document}
\maketitle

% \settheorem{1}{2}{3}
% \begin{lemma}
%
% \end{lemma}
% \popthm

\setexercise{4}{3}{4}
\begin{exercise}
  Define pointwise adjoints to the following bifunctors, giving rise
  to examples of two-variable adjunctions.
  \begin{enumerate}[label=(\roman*)]
  \item There is a bifunctor
    $$\Setp^\op\times\Setp\xrightarrow{\Hom_*}\Setp,$$
    where $\Hom_*((X,x),(Y,y))$ is defined to be the set of pointed
    functions $(X,x)\to(Y,y)$, with the constant function at~$y$
    serving as the basepoint.  Define a two-variable adjunction
    determined by this bifunctor, the pointwise left adjoints to
    $\Hom_*((X,x),-)$.
  \item Describe the left adjoint bifunctor
    $\Setp\times\Setp\xrightarrow{\wedge}\Setp$ constructed in~(i) in
    a sufficiently categorical way so that $\Set$ can be replaced by
    any cartesian closed category with pushouts and pullbacks.
  \item The discussion in Example 4.3.11 extends to any category
    $\Mod{R}$ of modules over a \emph{commutative} ring~$R$.  If
    $R$~is not commutative, for a pair of $R$-modules $A$ and~$B$,
    $\Hom(A,B)$ is not necessarily an $R$-module, but it is still an
    abelian group.  This construction defines a bifunctor
    $$\Mod{R}^\op\times\Mod{R}\xrightarrow{\Hom}\Ab.$$
    Extend this data to a two-variable adjunction.
  \item In a similar fashion, define a two-variable adjunction
    determined by the hom bifunctor
    $$\cat{Ch}_R^\op\times\cat{Ch}_R\xrightarrow{\Hom}\Mod{R},$$
    where the ring $R$ is commutative and the set
    $\Hom(A_\bullet,B_\bullet)$ of chain homomorphisms $A_\bullet\to
    B_\bullet$ inherits an $R$-module structure with addition and
    scalar multiplication defined elementwise.
  \end{enumerate}
\end{exercise}

\begin{proof}
  \begin{enumerate}[label=(\roman*)]
  \item In parallel to the discussion in Example 4.3.14, we call the
    left adjoint the smash product.  To simplify the notation, we
    write $X_*=(X,x)$ and so on.  We thus require
    $$\Setp(X_*\wedge Y_*,Z_*) \cong \Setp(Y_*, \Hom_*(X_*,Z_*)) \cong
    \Setp(X_*, \Hom_*(Y_*,Z_*)).$$
    For a given function $\hat{f}$ in $\Setp(Y_*, \Hom_*(X_*,Z_*))$,
    the values of $\hat{f}(y')$ are arbitrary values in~$Z$ except when
    $y'=y$ or $\hat{f}(y')(x)$, when the value must be~$z$.  We thus
    define
    $$(X,x)\wedge (Y,y) = (X\times Y)/(X\times\{y\}\cup \{x\}\times
    Y)$$
    with basepoint $(X\times\{y\}\cup \{x\}\times Y)$.  For a function
    $f\in\Setp(X_*\wedge Y_*, Z_*)$, we have $\hat{f}\in \Setp(Y_*,
    \Hom_*(X_*,Z_*))$ defined by $\hat{f}(y')(x') = f(x',y')$, where
    $(x',y')$ is taken to be the image of $(x',y')\in X\times Y$ in
    $(X,x)\wedge (Y,y)$; this is well-defined, as $f(x',y')=z$
    whenever $x'=x$ or $y'=y$.

    The naturality of this first isomorphism in all three variables is
    straightforward.

    Since $(X,x)\wedge(Y,y)$ is symmetrical in $(X,x)$ and $(Y,y)$,
    the third isomorphism also holds, and we thus have a two-variable
    adjunction.

  \item We first note that we need a little more than pushouts and
    pullbacks to do this: we also need either an initial object~$i$ or
    at least binary coproducts.  Let us assume this from now on.  (A
    cartesian closed category has finite products by definition, so in
    particular it has the empty product, which is a terminal element.)
    So suppose that $\sC$ is a cartesian closed category with an
    initial element~$i$, a terminal element~$t$, pushouts and
    pullbacks.  (From Theorem 3.4.18, we can equivalently describe
    this as assuming that $\sC$ has all finite limits and finite
    colimits.)  We write $\sC_*$ for the slice category $t/\sC$.

    One challenge here is not only to replicate the quotient
    construction of part~(i), which can be achieved using a pushout,
    but also to specify the basepoint of this quotient, as the smash
    product must live in~$\sC_*$.

    We follow the construction in A.~D. Elmendorf and M.~A. Mandell,
    \emph{Permutative categories, multicategories, and algebraic
      K-theory}, Algebr.\ Geom.\ Topol.\ 9 (2009) 2391--2441, which
    can be found at \url{https://arxiv.org/abs/0710.0082}.  Their
    Construction 4.19 describes both the wedge product and the
    $\Hom_*$ functor, with a proof that the construction gives a
    two-variable adjunction appearing in Lemma 4.20.

    Let $\func{f}{t}{a}$ and $\func{g}{t}{b}$ be objects in~$\sC_*$.
    For simplicity of notation, we write these objects as $a$ and~$b$,
    which can be thought of as objects in either $\sC_*$ or of~$\sC$.
    The functions $f$ and~$g$ are called the structure maps for $a$
    and~$b$.

    We can construct the generalisation of the basepoint
    $X\times\{y\}\cup \{x\}\times Y$ from~(i) as follows: working
    in~$\sC$, we have the objects $a\times t$ and $t\times b$, and
    thus the object $(a\times t)\coprod(t\times b)$.  (This is why we
    need binary coproducts or an initial element.)  Using this, we can
    construct the following pushout in~$\sC$:
    \begin{center}
      \begin{tikzcd}
        (a\times t)\coprod(t\times b)
        \ar[r] \ar[d, "!"'] \ar[dr, phantom, "\ulcorner", at end]
        & a\times b \ar[d] \\
        t \ar[r]
        & a\wedge b
      \end{tikzcd}
    \end{center}
    The top morphism is induced by
    $\func{1_a\times g}{a\times t}{a\times b}$ and
    $\func{f\times 1_b}{t\times b}{a\times b}$.  In $\Set$, the
    resulting object $a\wedge b$ is isomorphic to $a\times b$ with all
    elements of the subset $(a\times t)\coprod(t\times b)$ identified
    (where $t$~is interpreted as the basepoint of $a$ or~$b$ as
    appropriate).

    Finally, to turn this into an object in $\sC_*$, we require a
    morphism $t\to a\wedge b$, but this is now easy: it is the bottom
    arrow in this pushout.

    Though the exercise did not ask for this, we also describe the
    construction of the functor
    $\func{\Hom_*}{\sC_*\times\sC_*}{\sC_*}$.  For this, we use the
    following pullback in~$\sC$, where $\Hom$ here is the hom object
    in~$\sC$, being the two-variable adjunct of the product functor:
    \begin{center}
      \begin{tikzcd}[column sep = large]
        \Hom_*(a,b)
        \ar[r] \ar[d] \ar[dr, phantom, "\lrcorner", at start]
        & t \ar[d] \\
        \Hom(a,b) \ar[r, "{\Hom(f,1_b)}"']
        & \Hom(t,b)
      \end{tikzcd}
    \end{center}
    The morphism from $t\to\Hom(t,b)$ is the composite
    $$t\xrightarrow{\cong} \Hom(t,t) \xrightarrow{\Hom(1_t,g)}
    \Hom(t,b).$$
    Since $\Hom(t,-)$ is a right adjoint, it preserves limits (Theorem
    4.5.2), and since $t$~is the limit of the empty diagram,
    $\Hom(t,t)$ is also a terminal object, hence there is a unique
    morphism $t\to \Hom(t,t)$, which is an isomorphism.

    This gives the object $\Hom_*(a,b)$ in $\sC$; we now need to give
    a morphism $t\to\Hom_*(a,b)$ to give the corresponding object
    in~$\sC_*$.  For this, we consider the following diagram,
    extending the pullback:
    \begin{center}
      \begin{tikzcd}[column sep = large]
        t \ar[drr, bend left, "1_t"] \ar[d, "\cong"']
        \ar[dr, dashed, "\exists!"] \\
        \Hom(t,t) \ar[d, "{\Hom(!,1_t)}"] & \Hom_*(a,b)
        \ar[r] \ar[d] \ar[dr, phantom, "\lrcorner", at start]
        & t \ar[d] \\
        \Hom(a,t) \ar[r, "{\Hom(1_a,g)}"']
        & \Hom(a,b) \ar[r, "{\Hom(f,1_b)}"']
        & \Hom(t,b)
      \end{tikzcd}
    \end{center}
    The composite of the morphisms from $\Hom(t,t)$ to $\Hom(t,b)$ along
    the outer-left path is
    $$\Hom(f,1_b)\cdot \Hom(1_a,g) \cdot \Hom(!,1_t) = \Hom(1_t,g)$$
    (remembering that $\Hom$ is contravariant in the first variable),
    so the two outer paths from the upper left~$t$ to $\Hom(t,b)$ are
    equal.  By the universal property of the pullback, there is thus a
    unique morphism $t\to\Hom_*(a,b)$, and this defines $\Hom_*(a,b)$
    as an object of~$\sC_*$.

  \item We seek an adjunction
    $$\Mod{R}(F(M,A),N) \cong \Ab(A,\Hom(M,N)) \cong
    \Mod{R}(M,H(A,N))$$
    natural in $M,N\in\Mod{R}$ and $A\in\Ab$.

    Parallelling the discussion in Example 4.3.11, we define $F(M,A)$
    to be a tensor product.  As $A$~is an abelian group, it can be
    considered as a $\ZZ$-module.  There is a unique ring homomorphism
    $\ZZ\to R$,so the $R$-module $M$ can also be considered as a
    $\ZZ$-module, and hence in an $R$-$\ZZ$ bimodule.  We can thus
    form the tensor product $M\otimes_\ZZ A$, which is an $R$-module;
    we let $F(M,A)$ be this tensor product.  We then have
    $$\Mod{R}(M\otimes_\ZZ A,N) \cong \Ab(A,\Hom(M,N))$$
    which sends $\func{f}{M\otimes_\ZZ A}{N}$ to
    $\func{\hat{f}}{A}{\Hom(M,N)}$ where $\hat{f}(a)(m)=f(m\otimes
    a)$.  This isomorphism is natural in all three variables.

    The other right adjoint is $H(A,N)$, which must be an $R$-module.
    The only reasonble candidate for this is an $R$-module of
    homomorphisms from $A$ to~$N$.  We have a $\ZZ$-module
    $\Hom(A,N)$ by regarding $N$ as a $\ZZ$-module, and we can
    consider it as an $R$-module by defining scalar multiplication
    pointwise (so if $\phi\in\Hom(A,N)$ and $r\in R$, then
    $(r\phi)(a)=r(\phi(a))$).  This does give the required natural
    isomorphism, sending $\func{f}{M\otimes_\ZZ A}{N}$ to
    $\func{\hat{f}}{M}{\Hom(A,N)}$ where $\hat{f}(m)(a)=f(m\otimes
    a)$.

    We have thus produced a two-variable adjunction
    $$\Mod{R}(M\otimes_\ZZ A,N) \cong \Ab(A,\Hom(M,N)) \cong
    \Mod{R}(M,\Hom(A,N)).$$

  \item This case is similar to (iii); the only difficulty is to give
    meaningful definitions to the tensor product and the other hom
    object.  We wish to have natural isomorphisms
    $$\cat{Ch}_R(M\otimes_R A_\bullet,B_\bullet) \cong
    \cat{Ch}_R(A_\bullet,\Hom(M,B_\bullet)) \cong
    \Mod{R}(M,\Hom(A_\bullet,B_\bullet)).$$

    We first define the tensor product $M\otimes_R A_\bullet$.  This
    is the chain complex with $n$th module $M\otimes_R A_n$, where the
    boundary map is given by $\func{M\otimes_R d}{M\otimes_R
      A_n}{M\otimes_R A_{n-1}}$.  Then given a chain morphism
    $\func{f}{M\otimes_R A_\bullet}{B_\bullet}$, we obtain the
    transpose $\func{\hat{f}}{M}{\Hom(A_\bullet,B_\bullet)}$ by
    $\hat{f}(m)(a_n)=f(m\otimes a_n)$ where $a_n\in A_n$.

    We next define the hom object $\Hom(M,B_\bullet)$.  This is an
    $R$-module chain complex~$C_\bullet$, where $C_n=\Hom(M,B_n)$ with
    $R$-module structure given by pointwise operations (which we can
    do, as $R$~is commutative).  The boundary operator $d^C$ is given
    by $d^C=d^B_*$, that is, post-composition with the boundary
    operator of~$B_\bullet$.  Then given a chain morphism
    $\func{f}{M\otimes_R A_\bullet}{B_\bullet}$, we obtain the
    transpose $\func{\hat{f}}{A_\bullet}{\Hom(M,B_\bullet)}$ by
    $\hat{f}(a_n)(m)=f(m\otimes a_n)$ where $a_n\in A_n$.

    With these definitions of the tensor product and other hom object,
    we obtain the desired two-variable adjunction.
  \end{enumerate}

\end{proof}

\end{document}

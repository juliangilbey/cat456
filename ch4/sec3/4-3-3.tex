\documentclass[../../solutions]{subfiles}
\title{Template}
\author{}
\begin{document}
\maketitle

% \settheorem{1}{2}{3}
% \begin{lemma}
%   
% \end{lemma}
% \popthm

\setexercise{4}{3}{3}
\begin{exercise}
  Show that the contravariant power set functor
  $\func{P}{\Set^\op}{\Set}$ is mutually right adjoint to itself.
\end{exercise}

\begin{proof}
  Letting $X,Y\in\Set$, we seek a natural bijection
  $$\isom{\alpha_{X,Y}}{\Set(X,PY)}{\Set(Y,PX)}.$$
  Letting $\func{f}{X}{PY}$, we define $\alpha_{X,Y}f$ by
  $$(\alpha_{X,Y}f)(y)=\{x\in X:y\in f(x)\}.$$
  We first claim that $\alpha_{X,Y}$ is a bijection, by showing that
  $\alpha_{Y,X}\alpha_{X,Y}f=f$; it then immediately follows by
  swapping $X$ and~$Y$ that $\alpha_{X,Y}\alpha_{Y,X}g=g$ for
  $g\in\Set(Y,PX)$.  We have
  \begin{align*}
    (\alpha_{Y,X}\alpha_{X,Y}f)(x)
    &= \{y\in Y:x\in(\alpha_{X,Y}f)(y)\} \\
    &= \{y\in Y:y\in f(x)\} \\
    &= f(x)
  \end{align*}
  so $\alpha_{Y,X}\alpha_{X,Y}f=f$, and $\alpha_{X,Y}$ is a bijection.

  Next, we must show that this bijection is natural in $X$ and $Y$;
  again, it suffices to show naturality in~$X$ as naturality in~$Y$
  will follow by symmetry.  Suppose $\func{\phi}{X'}{X}$.  We must
  show that this diagram commutes:
  \begin{center}
    \begin{tikzcd}
      \Set(X,PY) \ar[r, "\alpha_{X,Y}"] \ar[d, "\phi^*"']
      & \Set(Y,PX) \ar[d, "(P\phi)_*"] \\
      \Set(X',PY) \ar[r, "\alpha_{X',Y}"']
      & \Set(Y,PX')
    \end{tikzcd}
  \end{center}
  Let $f\in\Set(X,PY)$.  Then the right-down path gives
  \begin{align*}
    ((P\phi)\alpha_{X,Y}f)(y)
    &= (P\phi)(\{x\in X:y\in f(x)\}) \\
    &= \phi^{-1}(\{x\in X:y\in f(x)\}) \\
    &= \{x'\in X':y\in f(\phi(x'))\}).
  \end{align*}
  The down-right path gives
  \begin{align*}
    (\alpha_{X',Y}(f\phi))(y) = \{x'\in X':y\in f\phi(x')\}
  \end{align*}
  by the definition of $\alpha_{X',Y}$, so the diagram commutes and
  $\alpha_{X,Y}$~is natural in~$X$, and hence in~$Y$ too.

  Therefore $P$ is mutually right adjoint to itself.
\end{proof}

\end{document}

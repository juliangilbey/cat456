\documentclass[../../solutions]{subfiles}
\title{Template}
\author{}
\begin{document}
\maketitle

\subsection{More on contravariant adjoint functors}
\label{ssec:contravariant-extension}

In Exercise 4.3.i, we dualised Definition 4.2.5 to define mutually
left/right adjoint functors in terms of natural transformations and
triangle identities.

It may be useful to also translate Lemma 4.1.3 and the explicit
formulas for transpositions in the proof of Proposition 4.2.6 to this
setting.  All we need to do is replace $\sC$ by~$\sC^\op$ or $\sD$
by~$\sD^\op$ in the statement of the lemma, and then translate it into
the $\sC$ and~$\sD$ setting.

Thus for mutually left adjoint functors $\func{F}{\sC^\op}{\sD}$ and
$\func{G}{\sD^\op}{\sC}$, Lemma 4.1.3 becomes:
\begingroup
\renewcommand{\theHtheorem}{\theHsection.\arabic{theorem}.left}
\settheorem{4}{1}{3}
\begin{lemma}[mutually left adjoint version]
  Consider a pair of functors $\func{F}{\sC^\op}{\sD}$ and
  $\func{G}{\sD^\op}{\sC}$ equipped with isomorphisms
  $\sD(Fc,d)\cong \sC(Gd,c)$ for all $c\in\sC$ and $d\in\sD$.
  Naturality of this collection of isomorphisms is equivalent to the
  assertion that for any morphisms with domains and codomains as
  displayed below
  \begin{center}
    \begin{tikzcd}
      Fc \ar[r, "f^\sharp"] \ar[d, "Fh"']
      & d \ar[d, "k"]
      & {}\ar[d, phantom, "\leftrightsquigarrow"]
      & c \ar[r, leftarrow, "f^\flat"] \ar[d, leftarrow, "h"']
      & Gd \ar[d, leftarrow, "Gk"] \\
      Fc' \ar[r, "g^\sharp"']
      & d'
      & {}
      & c' \ar[r, leftarrow, "g^\flat"']
      & Gd'
    \end{tikzcd}
  \end{center}
  the left-hand square commutes in $\sD$ if and only if the right-hand
  square transposed commutes in~$\sC$.
\end{lemma}
\popthm
\endgroup

The explicit formulae for transpositions in Proposition 4.2.6 become
the following.  Suppose that the counits of the adjunction are
$\Func{\eta}{GF}{1_\sC}$ and $\Func{\epsilon}{FG}{1_\sD}$.  Then the
adjuncts of $\func{f^\sharp}{Fc}{d}$ and $\func{g^\flat}{Gd}{c}$ are
\begin{align*}
  f^\flat &:= Gd \xrightarrow{Gf^\sharp} GFc \xrightarrow{\eta_c} c \\
  g^\sharp &:= Fc \xrightarrow{Fg^\flat} FGd \xrightarrow{\epsilon_d} d.
\end{align*}
Note that this is symmetrical as expected: swapping all pairs $F$/$G$,
$c$/$d$, $\eta$/$\epsilon$, $f^\sharp$/$g^\flat$ and
$f^\flat$/$g^\sharp$ leaves everything unchanged.

\bigskip

Repeating this for mutually right adjoint functors, Lemma 4.1.3
becomes:
\begingroup
\renewcommand{\theHtheorem}{\theHsection.\arabic{theorem}.right}
\settheorem{4}{1}{3}
\begin{lemma}[mutually right adjoint version]
  Consider a pair of functors $\func{F}{\sC^\op}{\sD}$ and
  $\func{G}{\sD^\op}{\sC}$ equipped with isomorphisms
  $\sD(d,Fc)\cong \sC(c,Gd)$ for all $c\in\sC$ and $d\in\sD$.
  Naturality of this collection of isomorphisms is equivalent to the
  assertion that for any morphisms with domains and codomains as
  displayed below
  \begin{center}
    \begin{tikzcd}
      Fc \ar[r, leftarrow, "f^\sharp"] \ar[d, leftarrow, "Fh"']
      & d \ar[d, leftarrow, "k"]
      & {}\ar[d, phantom, "\leftrightsquigarrow"]
      & c \ar[r, "f^\flat"] \ar[d, "h"']
      & Gd \ar[d, "Gk"] \\
      Fc' \ar[r, leftarrow, "g^\sharp"']
      & d'
      & {}
      & c' \ar[r, "g^\flat"']
      & Gd'
    \end{tikzcd}
  \end{center}
  the left-hand square commutes in $\sD$ if and only if the
  right-hand square transposed commutes in~$\sC$.
\end{lemma}
\popthm
\endgroup

The explicit formulae for transpositions in Proposition 4.2.6 become
the following.  Suppose that the units of the adjunction are
$\Func{\eta}{1_\sC}{GF}$ and $\Func{\epsilon}{1_\sD}{FG}$.  Then the
adjuncts of $\func{f^\sharp}{d}{Fc}$ and $\func{g^\flat}{c}{Gd}$ are
\begin{align*}
  f^\flat &:= c \xrightarrow{\eta_c} GFc \xrightarrow{Gf^\sharp} Gd \\
  g^\sharp &:= d \xrightarrow{\epsilon_d} FGd \xrightarrow{Fg^\flat} Fc.
\end{align*}

\end{document}

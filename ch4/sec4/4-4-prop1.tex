\documentclass[../../solutions]{subfiles}
\title{Template}
\author{}
\begin{document}
\maketitle

\settheorem{4}{4}{1}
\begin{proposition}
  \addcontentsline{toc}{subsection}{Proposition 4.4.1}%
  If $F$ and $F'$ are left adjoint to $G$, then $F\cong F'$, and
  moreover there is a unique natural isomorphism
  $\isom{\theta}{F}{F'}$ commuting with the units and counits of the
  adjunctions:
  $$
  \begin{tikzcd}
    1_\sC
    \ar[r, Rightarrow, "\eta"] \ar[dr, Rightarrow, "\eta'"']
    & GF \ar[d, Rightarrow, "G\theta"] \\
    & GF'
  \end{tikzcd}
  \hspace{2cm}
  \begin{tikzcd}
    FG
    \ar[r, Rightarrow, "\epsilon"] \ar[d, Rightarrow, "\theta G"']
    & 1_\sD \\
    F'G \ar[ur, Rightarrow, "\epsilon'"']
  \end{tikzcd}
  $$
\end{proposition}
\popthm

\begin{proof}[Proof]
  We complete the argument of Proof 1, and make a brief comment on the
  end of Proof~2.

  To complete Proof 1, we show that the triangles of natural
  transformations displayed in the statement commute.  For the left
  hand diagram, the composite path is
  $$1_\sC \xRightarrow{\eta} GF \xRightarrow{G\theta} GF'.$$
  By the definition of $\theta$, this expands to
  $$1_\sC \xRightarrow{\eta} GF \xRightarrow{GF\eta'} GFGF'
  \xRightarrow{G\epsilon F'} GF'.$$
  By the naturality of $\eta$, this equals
  $$1_\sC \xRightarrow{\eta'} GF' \xRightarrow{\eta GF'} GFGF'
  \xRightarrow{G\epsilon F'} GF'.$$
  Using the triangle identities for $\eta$ and $\epsilon$, this
  reduces to
  $$1_\sC \xRightarrow{\eta'} GF' \xRightarrow{1_G F'} GF'$$
  which is just $\eta'$, as required.

  The second diagram is similar:
  $$FG \xRightarrow{\theta G} F'G \xRightarrow{\epsilon'}
  1_\sD$$
  expands to
  $$FG \xRightarrow{F\eta' G} FGF'G \xRightarrow{\epsilon F'G} F'G
  \xRightarrow{\epsilon'} 1_\sD$$
  which, by the naturality of $\epsilon$, is equal to
  $$FG \xRightarrow{F\eta' G} FGF'G \xRightarrow{FG\epsilon'} FG
  \xRightarrow{\epsilon} 1_\sD$$
  which simplifies to
  $$FG \xRightarrow{F1_G} FG \xRightarrow{\epsilon} 1_\sD$$
  using the triangle identities for $\eta'$ and $\epsilon'$, and this
  is just $\epsilon$.

  \bigskip

  For Proof 2, we expand on the final sentence.  Setting $d=F'c$ gives
  $$\sD(F'c,F'c)\cong \sC(c,GF'c) \cong \sD(Fc,F'c).$$
  In this isomorphism, $1_{F'c}\in\sD(F'c,F'c)$ maps to
  $\eta'_c\in\sC(c,GF'c)$ and to $\theta_c\in\sD(Fc,F'c)$.  Going back
  from $\theta_c$ to $\eta'_c$ using Proposition 4.2.6 gives
  $$\eta'_c = G\theta_c\cdot \eta_c,$$
  which proves that the left diagram commutes.  For the other diagram,
  setting $c=Gd$ gives
  $$\sD(F'Gd,d)\cong \sC(Gd,Gd) \cong \sD(FGd,d).$$
  Here $1_{Gd}\in\sC(Gd,Gd)$ maps to $\epsilon'_d\in\sD(F'Gd,d)$ and
  $\epsilon_d\in\sD(FGd,d)$.  Using the naturality of isomorphism, we
  have the commuting square
  $$
  \begin{tikzcd}
    \sD(F'Gd, F'Gd) \ar[r, "\cong"] \ar[d, "(\epsilon'_d)_*"']
    & \sD(FGd, F'Gd) \ar[d, "(\epsilon'_d)_*"] \\
    \sD(F'Gd, d) \ar[r, "\cong"']
    & \sD(FGd, d)
  \end{tikzcd}
  $$
  The image of $1_{F'Gd}$ on the path via the top-right corner is
  $\epsilon'_d\cdot \theta_{Gd}$ and via the bottom-left
  is~$\epsilon_d$, as we showed above.  Therefore
  $\epsilon_d=\epsilon'_d\cdot \theta_{Gd}$ as required.
\end{proof}

\end{document}


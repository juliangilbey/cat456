\documentclass[../../solutions]{subfiles}
\title{Template}
\author{}
\begin{document}
\maketitle

% \settheorem{1}{2}{3}
% \begin{lemma}
%   
% \end{lemma}
% \popthm

\setexercise{4}{4}{3}
\begin{exercise}
  Use the natural bijection between hom-sets to prove the first half
  of Proposition 4.4.6, that an adjunction $F\ladj G$ induces an
  adjunction $F_*\ladj G_*$ by post-composition.  (Hint: Exercise
  3.2.iv might help.)
\end{exercise}

\begin{proof}
  We recall Exercise 3.2.iv.  (Unfortunately, the solution in the
  answer key provided by Dr Pardue's class has some significant errors
  in it, so we give a fresh proof here.)

  Exercise 3.2.iv asks: Show that for any small category~$\sJ$, any
  locally small category~$\sC$, and any parallel pair of functors
  $\pfunc{F,G}{\sJ}{\sC}$, the set $\Hom(F,G)$ of natural
  transformations from $F$ to~$G$ can be defined as a small limit in
  $\Set$.

  To show this, let $J^\mathsection$ be the category whose objects are
  morphisms in~$\sJ$ and which has morphisms $\func{\sigma_f}{1_x}{f}$
  and $\func{\tau_f}{1_y}{f}$ for every morphism $\func{f}{x}{y}$
  in~$\sJ$.  We define a functor $\func{H}{J^\mathsection}{\Set}$ by
  defining, for each morphism $\func{f}{x}{y}$ in~$\sJ$,
  \begin{align*}
    Hf &= \sC(Fx,Gy)\\
    H\sigma_f &= (Gf)_* \quad\text{which sends $H1_x=\sC(Fx,Gx)$ to
                $Hf=\sC(Fx,Gy)$} \\
    H\tau_f &= (Ff)^* \quad\text{which sends $H1_y=\sC(Fy,Gy)$ to
              $Hf=\sC(Fx,Gy)$}
  \end{align*}
  The limit of $H$ is defined to be the set of cones over~$H$ with
  summit~1.  Let $\lambda$~be such a cone with leg $\lambda_f$ for
  each morphism $f$ in~$\sJ$.  In particular, for each $x\in\sJ$,
  $\lambda_{1_x}\in \sC(Fx,Gx)$.  We claim that setting
  $\alpha_x=\lambda_{1_x}$ gives the components of a natural
  transformation $\Func{\alpha}{F}{G}$.  To show this, we must show
  that for any morphism $\func{f}{x}{y}$ in~$\sJ$, the diagram
  $$
  \begin{tikzcd}
    Fx \ar[r, "\alpha_x"] \ar[d, "Ff"']
    & Gx \ar[d, "Gf"] \\
    Fy \ar[r, "\alpha_y"']
    & Gy
  \end{tikzcd}
  $$
  commutes.  This requires $Gf\cdot \alpha_x = \alpha_y\cdot Ff$.  But
  as $\lambda$~is a cone over~$H$, the following diagram commutes:
  $$
  \begin{tikzcd}
    & 1 \ar[dl, "\lambda_{1_x}"']
    \ar[d, "\lambda_f"]
    \ar[dr, "\lambda_{1_y}"] \\
    H1_x \ar[r, "H\sigma_f"']
    & Hf
    & H1_y \ar[l, "H\tau_f"]
  \end{tikzcd}
  $$
  Evaluating the applications of the functor~$H$, this becomes
  \begin{equation}
    \label{eq:4-4-3-1}
    \begin{tikzcd}
      & 1 \ar[dl, "\alpha_x"']
      \ar[d, "\lambda_f"]
      \ar[dr, "\alpha_y"] \\
      \sC(Fx,Gx) \ar[r, "(Gf)_*"']
      & \sC(Fx,Gy)
      & \sC(Fy,Gy) \ar[l, "(Ff)^*"]
    \end{tikzcd}
  \end{equation}
  showing that $Gf\cdot \alpha_x=\alpha_y\cdot Ff$ as required.
  Therefore every cone over~$H$ can be understood as a natural
  transformation from $F$ to~$G$.

  Conversely, given such a natural transformation $\alpha$, we can
  define a cone~$\lambda$ over~$H$ with summit~1 in exactly this way,
  with $\lambda_f$ being defined from~\eqref{eq:4-4-3-1}.

  Therefore $\Hom(F,G)$ is bijective with the cones in
  $\lim_{\sJ^\mathsection} H$.

  Now writing $H=H(F,G)$ to make the dependence on $F$ and $G$
  explicit, we can regard $\Hom(F,G)$ as a functor
  $\func{\Hom}{(\sC^\sJ)^\op\times \sC^\sJ}{\Set}$, and likewise
  $\lim_{\sJ^\mathsection}H(F,G)$ is also a functor between these two
  categories.  The bijection
  $\Hom(F,G)\cong\lim_{\sJ^\mathsection}H(F,G)$ is natural in $F$
  and~$G$ (which seems obvious, and can be proven carefully by using
  Proposition 3.6.1, though the details are somewhat tedious and will
  not be given here), so these two functors are naturally isomorphic.

  Therefore $\Hom(F,G)$ can be defined as
  $\lim_{\sJ^\mathsection} H(F,G)$, a small limit in $\Set$.  This
  completes the solution of Exercise 3.2.iv.

  \bigskip

  We now return to the current exercise.  We are given the natural
  isomorphisms $\sD(Fc,d)\cong \sC(c,Gd)$ and a small category~$\sJ$,
  and we wish to show $\sD^\sJ(F_*K,L)\cong \sC^\sJ(K,G_*L)$, that is,
  $\sD^\sJ(FK,L)\cong \sC^\sJ(K,GL)$, natural in $K\in\sC^\sJ$ and
  $L\in\sD^\sJ$.

  Expressing $\sD^\sJ(FK,L)$ as the limit of the diagram
  $\func{H_\sD(FK,L)}{\sJ^\mathsection}{\Set}$, where $H_\sD(FK,L)$ is
  as described above, and $\sC^\sJ(K,GL)$ as the limit of the diagram
  $H_\sC(K,GL)$, we wish to show that these two limits are isomorphic,
  natural in $K$ and~$L$.  Using Corollary 3.6.3, it suffices to show
  that the diagrams themselves are naturally isomorphic.  Using the
  definition of the diagrams $H_\sD(FK,L)$ and $H_\sC(K,GL)$, we need
  to show that the following diagram commutes for each
  $\func{f}{x}{y}$ in~$\sJ$:
  $$
  \begin{tikzcd}
    \sD(FKx,Lx) \ar[r, "\cong"] \ar[d, "(Lf)_*"']
    & \sC(Kx,GLx) \ar[d, "(GLf)_*"] \\
    \sD(FKx,Ly) \ar[r, "\cong"']
    & \sC(Kx,GLy) \\
    \sD(FKy,Ly) \ar[r, "\cong"'] \ar[u, "(FKf)^*"]
    & \sC(Ky,GLy) \ar[u, "(Kf)^*"']
  \end{tikzcd}
  $$
  We take the isomorphisms to be transposition across the adjunction
  $F\ladj G$.  The upper square commutes by the naturality of
  transposition in the second variable and the lower square commutes
  by naturality of transposition in the first variable.  Therefore the
  limits are isomorphic.

  Next, we show that the isomorphism
  $\sD^\sJ(FK,L)\cong \sC^\sJ(K,GL)$ is natural in $K\in\sC^\sJ$.
  Suppose $K,K'\in\sC^\sJ$ with $\Func{\beta}{K'}{K}$.  We must show
  that the following diagram commutes:
  $$
  \begin{tikzcd}
    \lim_{\sJ^\mathsection}H_\sD(FK,L)
    \ar[r, "\cong"] \ar[d, "\lim(F\beta)^*"']
    & \lim_{\sJ^\mathsection}H_\sC(K,GL)
    \ar[d, "\lim\beta^*"] \\
    \lim_{\sJ^\mathsection}H_\sD(FK',L)
    \ar[r, "\cong"]
    & \lim_{\sJ^\mathsection}H_\sC(K',GL)
  \end{tikzcd}
  $$
  As $\lim_{\sJ^\mathsection}$ is a functor, it suffices to show that
  \begin{equation}
    \label{eq:4-4-3-2}
    \begin{tikzcd}
      H_\sD(FK,L)
      \ar[r, Rightarrow, "\cong"] \ar[d, Rightarrow, "(F\beta)^*"']
      & H_\sC(K,GL) \ar[d, Rightarrow, "\beta^*"] \\
      H_\sD(FK',L) \ar[r, Rightarrow, "\cong"]
      & H_\sC(K',GL)
    \end{tikzcd}
  \end{equation}
  commutes.  This requires showing that it commutes at every object
  in~$\sJ^\mathsection$.  Let $\func{f}{x}{y}$ be a morphism in~$\sJ$;
  the diagram at~$f$ is
  $$
  \begin{tikzcd}
    \sD(FKx,Ly) \ar[r, "\cong"] \ar[d, "(F\beta_x)^*"']
    & \sC(Kx,GLy) \ar[d, "(\beta_x)^*"] \\
    \sD(FK'x,Ly) \ar[r, "\cong"]
    & \sC(K'x,GLy)
  \end{tikzcd}
  $$
  Take $g^\sharp\in\sD(FKx,Ly)$.  Via the top-right path, this is
  transposed to $Gg^\sharp\cdot \eta_{Kx}$ (where
  $\Func{\eta}{1_\sC}{GF}$ is the unit of the transposition) and then
  sent to the composite
  $$K'x \xrightarrow{\beta_x} Kx \xrightarrow{\eta_{Kx}} GFKx
  \xrightarrow{Gg^\sharp} GLy.$$
  Along the bottom-left path, $g^\sharp$~is first sent to
  $g^\sharp\cdot F\beta_x$ and then transposed to give the composite
  $$K'x \xrightarrow{\eta_{K'x}} GFK'x \xrightarrow{GF\beta_x} GFKx
  \xrightarrow{Gg^\sharp} GLy.$$
  By naturality of $\eta$, these two composites are equal, and so the
  diagram~\eqref{eq:4-4-3-2} commutes at~$f$.  Therefore the original
  isomorphism is natural in $K\in\sC^\sJ$.

  Similarly, to show naturality in $L\in\sD^\sJ$, suppose
  $\Func{\gamma}{L}{L'}$.  We must show that the following diagram
  commutes:
  $$
  \begin{tikzcd}
    \lim_{\sJ^\mathsection}H_\sD(FK,L)
    \ar[r, "\cong"] \ar[d, "\lim \gamma_*"']
    & \lim_{\sJ^\mathsection}H_\sC(K,GL)
    \ar[d, "\lim(G\gamma)_*"] \\
    \lim_{\sJ^\mathsection}H_\sD(FK,L')
    \ar[r, "\cong"]
    & \lim_{\sJ^\mathsection}H_\sC(K,GL')
  \end{tikzcd}
  $$
  Again, it suffices to show that
  \begin{equation}
    \label{eq:4-4-3-3}
    \begin{tikzcd}
      H_\sD(FK,L)
      \ar[r, Rightarrow, "\cong"] \ar[d, Rightarrow, "\gamma_*"']
      & H_\sC(K,GL) \ar[d, Rightarrow, "(G\gamma)_*"] \\
      H_\sD(FK,L') \ar[r, Rightarrow, "\cong"]
      & H_\sC(K,GL')
    \end{tikzcd}
  \end{equation}
  commutes, and at $\func{f}{x}{y}$, this becomes
  $$
  \begin{tikzcd}
    \sD(FKx,Ly) \ar[r, "\cong"] \ar[d, "(\gamma_y)_*"']
    & \sC(Kx,GLy) \ar[d, "(G\gamma_y)_*"] \\
    \sD(FKx,L'y) \ar[r, "\cong"]
    & \sC(Kx,GL'y)
  \end{tikzcd}
  $$

  Take $g^\sharp\in\sD(FKx,Ly)$.  Via the top-right path, this is
  transposed to $Gg^\sharp\cdot \eta_{Kx}$ and then sent to the
  composite
  $$Kx \xrightarrow{\eta_{Kx}} GFKx \xrightarrow{Gg^\sharp} GLy
  \xrightarrow{G\gamma_y} GL'y.$$
  Along the bottom-left path, $g^\sharp$~is first sent to
  $\gamma_y\cdot g^\sharp$ and then transposed to give the composite
  $$Kx \xrightarrow{\eta_{Kx}} GFKx \xrightarrow{Gg^\sharp} GLy
  \xrightarrow{G\gamma_y} GL'y.$$
  As these two composites are identical, the
  diagram~\eqref{eq:4-4-3-3} commutes at~$f$.  Therefore the original
  isomorphism is natural in $L\in\sD^\sJ$.

  Therefore $\sD^\sJ(FK,L)\cong \sC^\sJ(K,GL)$, natural in $K$
  and~$L$, so $F_* \ladj G_*$.
\end{proof}

\end{document}

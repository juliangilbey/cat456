\documentclass[../../solutions]{subfiles}
\title{Template}
\author{}
\begin{document}
\maketitle

% \settheorem{1}{2}{3}
% \begin{lemma}
%
% \end{lemma}
% \popthm

\setexercise{4}{4}{2}
\begin{exercise}
  Use the unit and counit associated to an adjunction to prove
  Proposition 4.4.6.
\end{exercise}

\begin{proof}
  Let $\Func{\eta}{1_\sC}{GF}$ and $\Func{\epsilon}{FG}{1_\sD}$ be the
  unit and counit of the original adjunction $F\ladj G$.

  For the post-composition statement, let $\sJ$ be a small category.
  We define natural transformations
  $$\Func{\bar\eta}{1_{\sC^\sJ}}{G_*F_*}\quad\text{and}\quad
  \Func{\bar\epsilon}{F_*G_*}{1_{\sD^\sJ}};$$ by showing that these
  satisfy the triangle identities, we prove that $F_*\ladj G_*$ (using
  Definition 4.2.5).  Let $\func{K}{\sJ}{\sC}$ be a functor in
  $\sC^\sJ$, so $\func{F_*K=FK}{\sJ}{\sD}$, and let
  $\func{L}{\sJ}{\sD}$ be a functor in $\sD^\sJ$, so
  $\func{G_*L=GL}{\sJ}{\sC}$.  We then define $\bar\eta$ and
  $\bar\epsilon$ as follows:
  \begin{alignat*}{2}
    &\Func{\bar\eta_K}{K}{GFK} &&\quad\text{is defined as}\
    \bar\eta_K:=\eta K\\
    &\Func{\bar\epsilon_L}{FGL}{L} &&\quad\text{is defined as}\
    \bar\epsilon_L:=\epsilon L\\
  \end{alignat*}

  We first show that $\bar\eta$ is a natural transformation.  Suppose
  $\Func{\alpha}{K}{K'}$ is a morphism in~$\sC^\sJ$.  We then require
  the following square to commute:
  $$
  \begin{tikzcd}
    K \ar[r, Rightarrow, "\bar\eta_K"] \ar[d, Rightarrow, "\alpha"']
    & GFK \ar[d, Rightarrow, "GF\alpha"] \\
    K' \ar[r, Rightarrow, "\bar\eta_{K'}"']
    & GFK'
  \end{tikzcd}
  $$
  This commutes if it commutes for every component; for $j\in\sJ$, we
  have
  $$
  \begin{tikzcd}
    Kj \ar[r, "\eta_{Kj}"] \ar[d, "\alpha_j"']
    & GFKj \ar[d, "GF\alpha_j"] \\
    K'j \ar[r, "\eta_{K'j}"']
    & GFK'j
  \end{tikzcd}
  $$
  which commutes by the naturality of $\eta$.  The proof that
  $\bar\epsilon$ is a natural transformation is identical.

  We next show that the triangle identities hold.  The first triangle
  identity is
  $$
  \begin{tikzcd}
    F_*
    \ar[r, Rightarrow, "F_*\bar\eta"] \ar[dr, Rightarrow, "1_{F_*}"']
    & F_*G_*F_* \ar[d, Rightarrow, "\bar\epsilon F_*"] \\
    & F_*
  \end{tikzcd}
  $$
  We show that this commutes by showing that it commutes at every
  component.  At $K\in\sC^\sJ$, the triangle becomes
  $$
  \begin{tikzcd}
    FK
    \ar[r, Rightarrow, "F\eta K"] \ar[dr, Rightarrow, "1_{FK}"']
    & FGFK \ar[d, Rightarrow, "\epsilon_{FK}"] \\
    & FK
  \end{tikzcd}
  $$
  But for each $j\in\sJ$, this becomes
  $$
  \begin{tikzcd}
    FKj
    \ar[r, "F\eta_{Kj}"] \ar[dr, "1_{FKj}"']
    & FGFKj \ar[d, "\epsilon_{FKj}"] \\
    & FKj
  \end{tikzcd}
  $$
  which commutes as it is the $Kj$ component of the first triangle
  identity for $\eta$ and~$\epsilon$.  Alternatively, we could argue
  that since $(F\eta)\cdot(\epsilon F)=1_F$, which is an identity
  in~$\sD^\sC$, applying the functor $\func{K^*}{\sD^\sC}{\sD^\sJ}$
  gives $(F\eta K)\cdot(\epsilon FK)=1_{FK}$.  Thus the first identity
  holds for $\bar\eta$ and~$\bar\epsilon$.

  The argument for the second triangle identity is very similar: at
  $L\in\sD^\sJ$, the triangle
  $$
  \begin{tikzcd}
    G_*
    \ar[r, Rightarrow, "\bar\eta G_*"] \ar[dr, Rightarrow, "1_{G_*}"']
    & G_*F_*G_* \ar[d, Rightarrow, "G_*\bar\epsilon"] \\
    & G_*
  \end{tikzcd}
  $$
  becomes
  $$
  \begin{tikzcd}
    GL
    \ar[r, Rightarrow, "\eta GL"] \ar[dr, Rightarrow, "1_{GL}"']
    & GFGL \ar[d, Rightarrow, "G\epsilon L"] \\
    & GL
  \end{tikzcd}
  $$
  As $(G\epsilon)\cdot(\eta G)=1_G$, applying $L^*$ shows that this
  commutes, and so the second identity holds for $\bar\eta$
  and~$\bar\epsilon$.  Therefore both identities hold, and $F_*$~is
  left adjoint to~$G_*$.

  \bigskip

  For the pre-composition statement, we define natural transformations
  $\Func{\hat\eta}{1_{\sE^\sC}}{F^*G^*}$ and
  $\Func{\hat\epsilon}{G^*F^*}{1_{\sE^\sD}}$ by $\hat\eta_K=K\eta$ and
  $\hat\epsilon_L=L\epsilon$, where $K\in\sE^\sC$ and $L\in\sE^\sD$.
  The proof that $\hat\eta$ and~$\hat\epsilon$ are natural
  transformations is similar to the above argument for $\bar\eta$
  and~$\bar\epsilon$.

  The first triangle identity states that this triangle commutes:
  $$
  \begin{tikzcd}
    G^*
    \ar[r, Rightarrow, "G_*\hat\eta"] \ar[dr, Rightarrow, "1_{G^*}"']
    & G^*F^*G^* \ar[d, Rightarrow, "\hat\epsilon G^*"] \\
    & G^*
  \end{tikzcd}
  $$
  At $K\in\sE^\sC$, this triangle becomes
  $$
  \begin{tikzcd}
    KG
    \ar[r, Rightarrow, "K\eta G"] \ar[dr, Rightarrow, "1_{KG}"']
    & KGFG \ar[d, Rightarrow, "KG\epsilon"] \\
    & KG
  \end{tikzcd}
  $$
  and this commutes by applying $K$ to the triangle identity
  $(G\epsilon)\cdot(\eta G)=1_G$, so the first triangle identity
  holds.  Similarly, the second triangle identity requires this
  triangle to commute:
  $$
  \begin{tikzcd}
    F^*
    \ar[r, Rightarrow, "\hat\eta F^*"] \ar[dr, Rightarrow, "1_{F^*}"']
    & F^*G^*F^* \ar[d, Rightarrow, "F^*\hat\epsilon"] \\
    & F^*
  \end{tikzcd}
  $$
  At $L\in\sE^\sD$, this triangle becomes
  $$
  \begin{tikzcd}
    LF
    \ar[r, Rightarrow, "LF\eta"] \ar[dr, Rightarrow, "1_{LF}"']
    & LFGF \ar[d, Rightarrow, "L\epsilon F"] \\
    & LF
  \end{tikzcd}
  $$
  which commutes by applying $L$ to the triangle identity $(\epsilon
  F)\cdot(F\eta)=1_F$, so the second triangle identity also holds.  As
  both triangle identities hold, $G^*$ is left adjoint to~$F^*$.
\end{proof}

\end{document}

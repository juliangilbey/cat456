\documentclass[../../solutions]{subfiles}
\title{Template}
\author{}
\begin{document}
\maketitle

\subsection{The calculus of contravariant adjoint functors}
\label{ssec:contravariant-calculus}

Following on from the discussion in Exercise~\ref{ex:4.3.i} and
Section~\ref{ssec:contravariant-extension} in these solutions, we show
how the results of Section~4.4 can be adapted to the case of mutually
left or mutually right adjoint functors.

\begingroup
\renewcommand{\theHtheorem}{\theHsection.\arabic{theorem}.mutual}
\settheorem{4}{4}{1}
\begin{proposition}[mutually left/right adjoint version]
  If $\pfunc{F,F'}{\sC^\op}{\sD}$ are both mutually left adjoint
  to~$\func{G}{\sD^\op}{\sC}$ or both mutually right adjoint to~$G$,
  then $F\cong F'$, and moreover there is a unique natural isomorphism
  $\isom{\theta}{F}{F'}$ commuting with the (co)units of the
  adjunctions.  In the mutually left adjoint case, with
  $\Func{\eta}{GF}{1_\sC}$ and $\Func{\epsilon}{FG}{1_\sD}$ being the
  counits, the following triangles commute:
  $$
  \begin{tikzcd}
    1_\sC \ar[r, Leftarrow, "\eta"] \ar[dr, Leftarrow, "\eta'"']
    & GF \ar[d, Leftarrow, "G\theta"]
    & FG
      \ar[r, Rightarrow, "\epsilon"] \ar[d, Rightarrow, "\theta G"']
    & 1_\sD \\
    & GF'
    & F'G \ar[ur, Rightarrow, "\epsilon'"']
  \end{tikzcd}
  $$
  In the mutually right adjoint case, with $\Func{\eta}{1_\sC}{GF}$
  and $\Func{\epsilon}{1_\sD}{FG}$ being the units, the following
  triangles commute:
  $$
  \begin{tikzcd}
    1_\sC \ar[r, Rightarrow, "\eta"] \ar[dr, Rightarrow, "\eta'"']
    & GF \ar[d, Leftarrow, "G\theta"]
    & FG
      \ar[r, Leftarrow, "\epsilon"] \ar[d, Rightarrow, "\theta G"']
    & 1_\sD \\
    & GF'
    & F'G \ar[ur, Leftarrow, "\epsilon'"']
  \end{tikzcd}
  $$
\end{proposition}
\popthm
\endgroup

\begin{proof}[Proof]
  For the mutually left adjoint case, we can simply replace $\sC$ by
  $\sC^\op$ in the proof of Proposition 4.4.1 and then translate back
  to~$\sC$; this reverses all the arrows in the first triangle.
  Alternatively, we can repeat Proof~2 in this context: we have
  $$\sD(Fc,d)\cong \sC(Gd,c)\cong \sD(F'c,d).$$
  Then we have $\isom{\theta}{F}{F'}$ where the component $\theta_c$
  is defined to be the image of $1_{Fc}$ under this bijection.
  Setting $d=Fc$ gives the first identity, and setting $c=Gd$ gives
  the second.
\end{proof}

There are several ways to adapt Proposition 4.4.4 for the case of
mutually left/right adjoint functors.  Here are a couple; other
possibilities (for example where one pair are normal adjoints) can be
deduced in the same way.

\begingroup
\renewcommand{\theHtheorem}{\theHsection.\arabic{theorem}.mutual}
\settheorem{4}{4}{4}
\begin{proposition}[a mutually left/right adjoint version]
  If $\func{F}{\sC^\op}{\sD}$ and $\func{G}{\sD^\op}{\sC}$ are
  mutually left adjoint and $\func{F'}{\sD^\op}{\sE}$ and
  $\func{G'}{\sE^\op}{\sD}$ are mutually right adjoint, then $GG'\ladj
  F'F$.  If $F$ and~$G$ are mutually right adjoint and $F'$ and~$G'$
  are mutually left adjoint, then $F'F\ladj GG'$.
\end{proposition}
\popthm
\endgroup

\begin{proof}[Proof]
  We can prove this using natural isomorphisms.  For the first case,
  we have
  $$\sC(GG'e,c)\cong \sD(Fc,G'e) \cong \sE(e,F'Fc)$$
  while for the second case
  \[\sE(F'Fc,e)\cong \sD(G'e,Fc) \cong \sC(c,GG'e).\qedhere\]
\end{proof}

It is clear from this proof why we cannot say anything interesting if
$F$ and~$G$ are mutually left adjoint and $F'$ and~$G'$ are also
mutually left adjoint.

\bigskip

Adapting Proposition 4.4.6 for the case of mutually left/right
adjoints (say $\func{F}{\sC^\op}{\sD}$ and $\func{G}{\sD^\op}{\sC}$)
has an interesting twist: although the definition is symmetrical in
$F$ and~$G$ (swapping $\sC$ and~$\sD$ at the same time), to consider
the adjoint nature of the pre- and post-composition functors requires
symmetry-breaking: we must select one of $\sC$ and~$\sD$ (or something
related to it) to be ``replaced'' by its opposite.  We arbitrarily
choose~$\sC$ for this role.  We break the proposition into two
separate parts here: the post- and pre-composition parts.

\begingroup
\renewcommand{\theHtheorem}{\theHsection.\arabic{theorem}.mutual.post}
\settheorem{4}{4}{6}
\begin{proposition}[mutually left/right adjoint post-composition
  version]
  \label{prop:4.4.6.mutual.post}
  If $\func{F}{\sC^\op}{\sD}$ and $\func{G}{\sD^\op}{\sC}$ are
  mutually left adjoint, then the post-composition functors
  $$\func{F_*}{(\sC^{\sJ^\op})^\op}{\sD^\sJ} \quad \text{and} \quad
  \func{G_*}{(\sD^\sJ)^\op}{\sC^{\sJ^\op}}$$
  are mutually left adjoint.

  Dually, if $F$ and~$G$ are mutually right adjoint, then $F_*$
  and~$G_*$ are mutually right adjoint.
\end{proposition}
\popthm
\endgroup

\begin{proof}[Proof]
  We prove the mutually left adjoint case first.  Let
  $\Func{\eta}{GF}{1_\sC}$ and $\Func{\epsilon}{FG}{1_\sD}$ be the
  counits of the given adjunction; recall from Exercise~\ref{ex:4.3.i}
  that these satisfy the mutually left adjoint triangle identities, so
  $\epsilon_{Fc}\cdot (F\eta_c)=1_{Fc}$ and
  $\eta_{Gd}\cdot(G\epsilon_d)=1_{Gd}$ for every $c\in\sC$ and
  $d\in\sD$.

  We define natural transformations
  $\Func{\bar\eta}{G_*F_*}{1_{\sC^{\sJ^\op}}}$ and
  $\Func{\bar\epsilon}{F_*G_*}{1_{\sD^\sJ}}$ as follows:
  \begin{alignat*}{4}
    &\text{For $K\in\sC^{\sJ^\op}$,}&\quad&\Func{\bar\eta_K}{GFK}{K} &
    \quad & \text{is defined by}& \quad \bar\eta_K&:=\eta K \\
    &\text{For $L\in\sD^\sJ$,}&\quad&\Func{\bar\epsilon_L}{FGL}{L} &
    \quad & \text{is defined by}& \bar\epsilon_L&:=\epsilon L
  \end{alignat*}
  It is straightforward to show that these do define natural
  transformations: suppose $\Func{\alpha}{K}{K'}$ with
  $K,K'\in\sC^{\sJ^\op}$.  Then the diagram
  $$
  \begin{tikzcd}
    GFK
    \ar[r, Rightarrow, "\bar\eta_K"] \ar[d, Rightarrow, "GF\alpha"']
    & K \ar[d, Rightarrow, "\alpha"] \\
    GFK' \ar[r, Rightarrow, "\bar\eta_{K'}"']
    & K'
  \end{tikzcd}
  $$
  commutes if and only if it commutes at each $j\in\sJ$; at $j$, the
  diagram is
  $$
  \begin{tikzcd}
    GFKj \ar[r, "\eta_{Kj}"] \ar[d, "GF\alpha_j"']
    & Kj \ar[d, "\alpha_j"] \\
    GFK'j \ar[r, "\eta_{K'j}"']
    & K'j
  \end{tikzcd}
  $$
  and this commutes by naturality of $\eta$.  The proof for
  $\bar\epsilon$ is identical.

  We now show that the (mutually left adjoint) triangle identities
  hold for $\bar\eta$ and $\bar\epsilon$.  They are
  $$
  \begin{tikzcd}
    F_*
    \ar[r, Rightarrow, "F_*\bar\eta"] \ar[dr, Rightarrow, "1_{F_*}"']
    & F_*G_*F_* \ar[d, Rightarrow, "\bar\epsilon F_*"]
    & G_*
    \ar[r, Leftarrow, "\bar\eta G_*"] \ar[dr, Leftarrow, "1_{G_*}"']
    & G_*F_*G_* \ar[d, Leftarrow, "G_*\bar\epsilon"] \\
    & F_*
    && G_*
  \end{tikzcd}
  $$
  These commute if and only if they commute at each
  $K\in\sC^{\sJ^\op}$ and $L\in\sD^\sJ$, respectively:
  $$
  \begin{tikzcd}
    FK
    \ar[r, Rightarrow, "F_*\bar\eta_K"] \ar[dr, Rightarrow, "1_{FK}"']
    & FGFK \ar[d, Rightarrow, "\bar\epsilon_{FK}"]
    & GL
    \ar[r, Leftarrow, "\bar\eta_{GL}"] \ar[dr, Leftarrow, "1_{GL}"']
    & GFGL \ar[d, Leftarrow, "G_*\bar\epsilon_L"] \\
    & FK
    && GL
  \end{tikzcd}
  $$
  In turn, these commute if and only if they commute at each
  $j\in\sJ$:
  $$
  \begin{tikzcd}
    FKj
    \ar[r, "F\eta_{Kj}"] \ar[dr, "1_{FKj}"']
    & FGFKj \ar[d, "\epsilon_{FKj}"]
    & GLj
    \ar[r, leftarrow, "\eta_{GLj}"] \ar[dr, leftarrow, "1_{GLj}"']
    & GFGLj \ar[d, leftarrow, "G\epsilon_{Lj}"] \\
    & FKj
    && GLj
  \end{tikzcd}
  $$
  These commute by the triangle identities for $\eta$ and $\epsilon$,
  so $\bar\eta$ and~$\bar\epsilon$ themselves satisfy the triangle
  identities.  Therefore $F_*$ and~$G_*$ are mutually left adjoint
  functors.

  The mutually right adjoint case is almost identical.  We take the
  same definitions of $\bar\eta$ and~$\bar\epsilon$ (where
  $\Func{\eta}{1_\sC}{GF}$ and $\Func{\epsilon}{1_\sD}{FG}$ are now
  the units), and we require the opposite triangle identities to hold:
  $$
  \begin{tikzcd}
    F_*
    \ar[r, Leftarrow, "F_*\bar\eta"] \ar[dr, Leftarrow, "1_{F_*}"']
    & F_*G_*F_* \ar[d, Leftarrow, "\bar\epsilon F_*"]
    & G_*
    \ar[r, Rightarrow, "\bar\eta G_*"] \ar[dr, Rightarrow, "1_{G_*}"']
    & G_*F_*G_* \ar[d, Rightarrow, "G_*\bar\epsilon"] \\
    & F_*
    && G_*
  \end{tikzcd}
  $$
  At $K\in\sC^{\sJ^\op}$, $L\in\sD^\sJ$ and $j\in\sJ$, these become
  $$
  \begin{tikzcd}
    FKj
    \ar[r, leftarrow, "F\eta_{Kj}"] \ar[dr, leftarrow, "1_{FKj}"']
    & FGFKj \ar[d, leftarrow, "\epsilon_{FKj}"]
    & GLj
    \ar[r, "\eta_{GLj}"] \ar[dr, "1_{GLj}"']
    & GFGLj \ar[d, "G\epsilon_{Lj}"] \\
    & FKj
    && GLj
  \end{tikzcd}
  $$
  and these commute by the mutually right adjoint triangle identities
  for $\eta$ and~$\epsilon$.  Hence $F_*$ and~$G_*$ are mutually right
  adjoint.
\end{proof}

The pre-composition case is more surprising: $F^*$ and $G^*$ are both
covariant functors and they are normal adjoints.  Which is the left
adjoint and which is the right depends on our choice of how to break
the symmetry when we define $F^*$ and $G^*$.  (We can define them
symmetrically, but then they would be functors between incompatible
categories.)

\begingroup
\renewcommand{\theHtheorem}{\theHsection.\arabic{theorem}.mutual.pre}
\settheorem{4}{4}{6}
\begin{proposition}[mutually left/right adjoint pre-composition version]
  Suppose $\func{F}{\sC^\op}{\sD}$ and $\func{G}{\sD^\op}{\sC}$ are
  mutually left adjoint.  Then pre-composition with $F$ and~$G$
  defines an adjunction
  $$
  \begin{tikzcd}
    \sE^{\sC^\op}
    \ar[r, shift left = 1ex, "G^*"]
    \ar[r, phantom, "\scriptstyle\bot"]
    \ar[r, leftarrow, shift right = 1ex, "F^*"']
    & \sE^{\sD}
  \end{tikzcd}
  $$

  Dually, if $F$ and~$G$ are mutually right adjoint, then
  pre-composition defines an adjunction
  $$
  \begin{tikzcd}
    \sE^{\sD}
    \ar[r, shift left = 1ex, "F^*"]
    \ar[r, phantom, "\scriptstyle\bot"]
    \ar[r, leftarrow, shift right = 1ex, "G^*"']
    & \sE^{\sC^\op}
  \end{tikzcd}
  $$
\end{proposition}
\popthm
\endgroup

\begin{proof}[Proof]
  The proof is very similar to the previous one.  Again we prove the
  mutually left adjoint case first.  Letting $\Func{\eta}{GF}{1_\sC}$
  and $\Func{\epsilon}{FG}{1_\sD}$ be the counits of the given
  adjunction, we define natural transformations
  $\Func{\bar\eta}{1_{\sE^{\sC^\op}}}{F^*G^*}$ and
  $\Func{\bar\epsilon}{G^*F^*}{1_{\sE^\sD}}$ as follows:
  \begin{alignat*}{4}
    &\text{For $K\in\sE^{\sC^\op}$,}&\quad&\Func{\bar\eta_K}{K}{KGF} &
    \quad & \text{is defined by}& \quad \bar\eta_K&:=K\eta \\
    &\text{For $L\in\sE^\sD$,}&\quad&\Func{\bar\epsilon_L}{LFG}{L} &
    \quad & \text{is defined by}& \bar\epsilon_L&:=L\epsilon
  \end{alignat*}
  Note that these definitions make sense because $K$ is contravariant
  but $L$~is covariant.
  
  We show that the triangle identities hold for $\bar\eta$ and
  $\bar\epsilon$.  They are
  $$
  \begin{tikzcd}
    G^*
    \ar[r, Rightarrow, "G^*\bar\eta"] \ar[dr, Rightarrow, "1_{G^*}"']
    & G^*F^*G^* \ar[d, Rightarrow, "\bar\epsilon G^*"]
    & F^*
    \ar[r, Rightarrow, "\bar\eta F^*"] \ar[dr, Rightarrow, "1_{F^*}"']
    & F^*G^*F^* \ar[d, Rightarrow, "F^*\bar\epsilon"] \\
    & G^*
    && F^*
  \end{tikzcd}
  $$
  These commute if and only if they commute at each
  $K\in\sE^{\sC^\op}$ and $d\in\sD$, or at each $L\in\sE^\sD$ and
  $c\in\sC$, respectively giving:
  $$
  \begin{tikzcd}
    KGd
    \ar[r, "(\bar\eta_K)_{Gd}"] \ar[dr, "1_{KGd}"']
    & KGFGd \ar[d, "(\bar\epsilon_{KG})_d"]
    & LFc
    \ar[r, "(\bar\eta_{LF})_c"] \ar[dr, "1_{LFc}"']
    & LFGFc \ar[d, "(\bar\epsilon_L)_{Fc}"] \\
    & KGd
    && LFc
  \end{tikzcd}
  $$
  Using the definitions of $\bar\eta$ and $\bar\epsilon$, the first
  triangle commutes if $(KG\epsilon_d)\cdot (K\eta_{Gd})=1_{KGd}$.  As
  $(\eta_{Gd})\cdot (G\epsilon_d)=1_{Gd}$ by the triangle identities
  for the mutually left adjoint $F$ and~$G$, and $K$~is contravariant,
  this shows that the first triangle commutes.  For the second
  triangle, we require $(L\epsilon_{Fc})\cdot (LF\eta_c)=1_{LFc}$.  As
  $(\epsilon_{Fc})\cdot (F\eta_c)=1_{Fc}$ by the triangle identities
  for $F$ and~$G$, and $L$~is covariant, the second triangle also
  commutes.  Therefore $G^*\ladj F^*$.

  The case of mutually right adjoint functors follows in an identical
  fashion.
\end{proof}

\end{document}

\documentclass[../../solutions]{subfiles}
\title{Template}
\author{}
\begin{document}
\maketitle

% \settheorem{1}{2}{3}
% \begin{lemma}
%   
% \end{lemma}
% \popthm

\setexercise{4}{6}{2}
\begin{exercise}
  Use Theorem 4.6.3 to prove that the inclusion
  $\mathsf{Haus}\hookrightarrow\Top$ of the full subcategory of
  Hausdorff spaces into the category of all spaces has a left adjoint.
  The left adjoint carries a space to its ``largest Hausdorff
  quotient.''  Conclude, by applying Proposition 4.5.15, that the
  category of Hausdorff spaces, as a reflective subcategory of a
  complete and cocomplete category, is cocomplete as well as complete.
\end{exercise}

\begin{proof}
  We check that the conditions of Theorem 4.6.3 (the General Adjoint
  Functor Theorem) apply to this inclusion.  First, we note that
  $\mathsf{Haus}$ is a full subcategory of $\Top$: every morphism between
  two Hausdorff spaces in $\Top$ is just a continuous function, and so
  is a morphism in $\mathsf{Haus}$.

  \begin{itemize}
  \item The inclusion is continuous: a limit (cone) in $\mathsf{Haus}$
    is a limit (cone) in~$\Top$.  We can see this using Theorem
    3.4.12 as follows.  Suppose that we have a diagram
    $\func{F}{\sJ}{\mathsf{Haus}}$.  Then $\lim_\sJ F$ is obtained as
    an equaliser
    $$
    \begin{tikzcd}
      \lim_\sJ F \ar[r, rightarrowtail]
      & \prod\limits_{j\in\ob\sJ} Fj
      \ar[r, shift left, "c"] \ar[r, shift right, "d"']
      & \prod\limits_{f\in\mor\sJ} F(\cod f)
    \end{tikzcd}
    $$
    If the products and equaliser are interpreted in $\Top$ rather
    than $\mathsf{Haus}$, they and the morphisms are unchanged (noting
    that the product of Hausdorff spaces in $\Top$ is still Hausdorff).
    Thus the limit in $\mathsf{Haus}$ is a limit in $\Top$.

    Note that this also shows that $\mathsf{Haus}$ is complete; see
    the proof of Theorem 3.5.2.

  \item $\mathsf{Haus}$ is a locally small subcategory as $\Top$ is.
  \item $\mathsf{Haus}$ satisfies the solution set condition.  Letting
    $X\in\Top$, we define the set $\Phi_X=\{\func{f_i}{X}{Y_i}\}$
    (where $Y_i\in\mathsf{Haus}$) as follows.  Every continuous
    function $\func{f}{X}{Y}$ with $Y$~Hausdorff factorises through
    the image: $X\to \im f\hookrightarrow Y$.  The image is
    homeomorphic to a subset of~$X$ endowed with some Hausdorff
    topology, say~$\hat X$, giving the following factorisation of~$f$:
    $$X\to \hat{X} \xrightarrow{\cong} \im f \hookrightarrow Y,$$
    so in particular, $f$ factorises through this morphism
    $X\to\hat{X}$.  To define $\Phi_X$, we therefore let the~$f_i$
    range over all continuous maps to all possible subsets of~$X$ with
    all possible Hausdorff topologies; this is a set as required.
  \end{itemize}

  As the conditions of the theorem are met, the inclusion admits a
  left adjoint~$L$, which is a reflector.

  We have already shown above that $\mathsf{Haus}$ is complete.  For
  cocompleteness, consider a diagram in $\mathsf{Haus}$.  Its image in
  $\Top$ admits a colimit as $\Top$ is cocomplete by Proposition
  3.5.2.  Therefore by Proposition 4.5.15(ii), the diagram also admits
  a colimit in $\mathsf{Haus}$ (obtained by applying the reflector~$L$
  to the colimit in $\mathsf{Top}$).  Therefore $\mathsf{Haus}$ is
  cocomplete.
\end{proof}

\end{document}


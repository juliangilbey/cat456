\documentclass[../../solutions]{subfiles}
\title{Template}
\author{}
\begin{document}
\maketitle

% \settheorem{1}{2}{3}
% \begin{lemma}
%   
% \end{lemma}
% \popthm

\setexercise{4}{6}{1}
\begin{exercise}
  Give a direct proof of Lemma 4.6.2 along the lines of the argument
  used in Proposition 3.3.8.
\end{exercise}

\begin{proof}
  To define a diagram $\func{(K,\kappa)}{\sJ}{s\downarrow U}$ is to
  define a functor $\func{K}{\sJ}{\sA}$ together with a cone
  $\Func{\kappa}{s}{UK}$ with summit~$s$.  We will show that if
  $$
  \begin{tikzcd}
    K\colon \sJ \ar[r, "{(K,\kappa)}"]
    & s\downarrow U \ar[r, "\Pi"]
    & \sA
  \end{tikzcd}
  $$
  admits a limit cone in $\sA$, then this cone lifts to define a
  unique cone over $(K,\kappa)$ in $s\downarrow U$, and moreover this
  cone is a limit cone.

  Given a limit cone $\Func{\lambda}{\ell}{K}$ in $\sA$, as this limit
  is preserved by~$U$ (by assumption), $\Func{U\lambda}{U\ell}{UK}$ is
  a limit cone in~$\sS$.  It follows that there is a unique
  factorisation of the cone~$\kappa$ through~$U\lambda$ along a map
  $\func{t}{s}{U\ell}$.

  Then $(\ell, t)$ is a limit for the diagram $(K,\kappa)$ in
  $s\downarrow U$ with limit cone $U\lambda\cdot t$.
\end{proof}

\end{document}

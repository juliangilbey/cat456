\documentclass[../../solutions]{subfiles}
\title{Template}
\author{}
\begin{document}
\maketitle

% \settheorem{1}{2}{3}
% \begin{lemma}
%   
% \end{lemma}
% \popthm

\setexercise{4}{6}{3}
\begin{exercise}
  Suppose $\sC$ is a locally small category with coproducts.  Show
  that a functor $\func{F}{\sC}{\Set}$ is representable if and only if
  it admits a left adjoint.
\end{exercise}

\begin{proof}
  Suppose that $F\cong \sC(c,-)$ for some $c\in\sC$.  We define a
  functor $\func{L}{\Set}{\sC}$ as follows.  For $X\in\Set$, let $LX:=
  \coprod_X c$.  We then have
  $$\sC(LX,c') = \sC({\textstyle\coprod_X c},c')
  \cong \sC(c,c')^X \cong (Fc')^X \cong \Set(X,Fc')$$
  where the first congruence follows from Example 3.4.9 and the final
  one holds because for any $A,B\in\Set$, $A^B\cong\Set(B,A)$.
  Therefore $L\ladj F$.

  Conversely, if $F$ admits a left adjoint, then it is representable
  by the proof of Corollary 4.6.14, which made no assumptions
  on~$\sC$.

  Note that the assumptions of this exercise were stronger than
  necessary; having copowers is sufficient.
\end{proof}

\end{document}


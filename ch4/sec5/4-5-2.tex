\documentclass[../../solutions]{subfiles}
\title{Template}
\author{}
\begin{document}
\maketitle

% \settheorem{1}{2}{3}
% \begin{lemma}
%   
% \end{lemma}
% \popthm

\setexercise{4}{5}{2}
\begin{exercise}
  Suppose the diagonal functor $\func{\Delta}{\sC}{\sC^\sJ}$ admits
  both left and right adjoints.  Describe the units and counits of
  these adjunctions.
\end{exercise}

\begin{proof}
  For the right adjoint, $\Delta\ladj\lim$, we have the natural
  isomorphism $\sC^\sJ(\Delta c, F) \cong \sC(c, \lim F)$.  This takes
  a cone~$\mu$ over~$F$ with summit~$c$ to the unique morphism
  $c\to\lim F$ through which the cone factorises through the limit
  cone $\Func{\lambda}{\lim F}{F}$.

  The unit of this adjunction, $\eta$, has $\func{\eta_c}{c}{\lim
    \Delta c}$ being the transpose of $1_{\Delta c}$.  If we take
  $\lim \Delta c$ to be~$c$ with limit cone $1_{\Delta c}$, then
  $\eta_c=1_c$.  (We could take $\lim \Delta c$ to be any object
  isomorphic to~$c$, but that would make the answer much messier for
  little benefit.)

  The counit of this adjunction, $\epsilon$, has
  $\Func{\epsilon_F}{\Delta \lim F}{F}$ being the transpose of
  $1_{\lim F}$.  This is the limit cone associated with $\lim F$.

  For the left adjoint, $\colim\ladj\Delta$, we have the dual results:
  for the unit, $\Func{\eta_F}{F}{\Delta\colim F}$ is the colimit cone
  associated with $\colim F$, and for the counit,
  $\func{\eta_c}{\colim\Delta c}{c}$ is the identity~$1_c$.
\end{proof}

\end{document}


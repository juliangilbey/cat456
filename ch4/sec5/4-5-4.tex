\documentclass[../../solutions]{subfiles}
\title{Template}
\author{}
\begin{document}
\maketitle

% \settheorem{1}{2}{3}
% \begin{lemma}
%
% \end{lemma}
% \popthm

\setexercise{4}{5}{4}
\begin{exercise}
  If $\func{G}{\sD}{\sC}$ has a left adjoint $F$ and if $\sD$ and
  $\sC$ admit all limits indexed by a category~$\sJ$, use Propositions
  4.5.1, 4.4.4, and 4.4.6 to argue that right adjoints preserve limits
  by considering the diagram of adjoint functors:
  $$
  \begin{tikzcd}[row sep = large, column sep = large]
    \sC
      \ar[r, leftarrow, shift left = 1ex, "G"]
      \ar[r, shift right = 1ex, "F"']
      \ar[r, phantom, "\scriptscriptstyle\top"]
      \ar[d, leftarrow, shift right = 1ex, "\lim"']
      \ar[d, shift left = 1ex, "\Delta"]
      \ar[d, phantom, "\scriptscriptstyle\radj"]
    & \sD
      \ar[d, shift right = 1ex, "\Delta"']
      \ar[d, leftarrow, shift left = 1ex, "\lim"]
      \ar[d, phantom, "\scriptscriptstyle\ladj"] \\
    \sC^\sJ
      \ar[r, shift left = 1ex, "F_*"]
      \ar[r, leftarrow, shift right = 1ex, "G_*"']
      \ar[r, phantom, "\scriptscriptstyle\bot"]
    & \sD^\sJ
  \end{tikzcd}
  $$
  The characterization of preservation of limits provided by Exercise
  3.3.i may prove useful.
\end{exercise}

\begin{proof}
  The dual of the solution to Exercise 3.3.i is the following (using
  notation which will be convenient for the question at hand): For any
  diagram $\func{K}{\sJ}{\sD}$ and any functor $\func{G}{\sD}{\sC}$,
  there is a canonical map $\func{f}{G\lim K}{\lim GK}$, assuming both
  limits exists, and $G$~preserves the limit of~$K$ if and only if
  $f$~is an isomorphism.  The map~$f$ is defined as follows.  Let
  $\Func{\lambda}{\lim GK}{GK}$ and $\Func{\mu}{\lim K}{K}$ be the
  respective limit cones.  The cone $G\mu$ factorises uniquely
  through~$\lambda$, and we define the $f$~to be that unique morphism;
  in other words, it is the unique morphism making the following
  diagram commute:
  \begin{equation}
    \label{eq:4-5-4-3.3.i}
    \begin{tikzcd}
      \Delta G\lim K
        \ar[rr, Rightarrow, "G\mu"]
        \ar[rd, Rightarrow, dashed, "\exists! \Delta f"']
      && GK \\
      & \Delta\lim GK \ar[ru, Rightarrow, "\lambda"]
    \end{tikzcd}
  \end{equation}
  where $\Delta$ is the diagonal functor.

  We now consider the diagram in the question.  We have
  $\Delta\ladj\lim$ by Proposition 4.5.1, and $F_*\ladj G_*$ by
  Proposition 4.4.6.  We can compose the pairs of adjoints using
  Proposition 4.4.4 to deduce that $\Delta F\ladj G\lim$ and
  $F_*\Delta\ladj \lim G_*$ (using the two paths from $\sC$
  to~$\sD^\sJ$).  We also note that $\Delta F=F_*\Delta$, and so there
  is a unique natural isomorphism $\isom{\theta}{G\lim}{\lim G_*}$
  commuting with the units and counits of the composite adjunctions,
  by the dual of Proposition 4.4.1.  We will use this isomorphism to
  obtain the required canonical map appearing in Exercise 3.3.i.

  We first need to determine the units and counits of the composite
  adjunctions; we do this using the proof of Proposition 4.4.4 and the
  answer to Exercise~\ref{ex:4.5.ii}.  We label the four units
  ($\eta$) and counits ($\epsilon$) in the square as follows:
  $$
  \begin{tikzcd}[row sep = large, column sep = large]
    \sC
      \ar[r, leftarrow, shift left = 1ex, "G"]
      \ar[r, shift right = 1ex, "F"']
      \ar[r, phantom, "\scriptstyle\eta" near start,
          "\scriptscriptstyle\top",
          "\scriptstyle\epsilon" near end]
      \ar[d, leftarrow, shift right = 1ex, "\lim"']
      \ar[d, shift left = 1ex, "\Delta"]
      \ar[d, phantom, "\scriptstyle\hat\eta" near start,
          "\scriptscriptstyle\radj",
          "\scriptstyle\hat\epsilon" near end]
    & \sD
      \ar[d, shift right = 1ex, "\Delta"']
      \ar[d, leftarrow, shift left = 1ex, "\lim"]
      \ar[d, phantom, "\scriptstyle\eta'" near start,
          "\scriptscriptstyle\ladj",
          "\scriptstyle\epsilon'" near end] \\
    \sC^\sJ
      \ar[r, shift left = 1ex, "F_*"]
      \ar[r, leftarrow, shift right = 1ex, "G_*"']
      \ar[r, phantom, "\scriptstyle\hat\eta'" near start,
          "\scriptscriptstyle\bot",
          "\scriptstyle\hat\epsilon'" near end]
    & \sD^\sJ
  \end{tikzcd}
  $$
  The composite adjunction $\Delta F\ladj G\lim$ has unit $\bar\eta$
  and counit~$\bar\epsilon$, while the composite
  $F_*\Delta\ladj \lim G_*$ has unit~$\bar{\hat\eta}$ and counit
  $\bar{\hat\epsilon}$.  From Exercise 4.5.ii and the proof of
  Proposition 4.4.4, we have, for $K\in\sD^\sJ$ and $L\in\sC^\sJ$:
  \begin{itemize}
  \item $\func{\epsilon'_K}{\Delta \lim K}{K}$ is the limit cone
    over~$K$
  \item $\func{\hat\epsilon_L}{\Delta \lim L}{L}$ is the limit cone
    over~$L$
  \item $\func{\hat\epsilon'_K}{FGK}{K}$ is given by
    $\hat\epsilon'_K=\epsilon K$ from the proof of Proposition 4.4.6
    (see Exercise~\ref{ex:4.4.ii})
  \item $\func{\hat\eta'_L}{L}{GFL}$ is given by
    $\hat\eta'_L=\eta L$ similarly
  \item
    $\bar\epsilon = \Delta FG\lim \xRightarrow{\Delta \epsilon \lim}
    \Delta \lim \xRightarrow{\epsilon'} 1_{\sD^\sJ}$ using the proof
    of Proposition 4.4.4
  \item
    $\bar{\hat\epsilon} = F_*\Delta \lim G_* \xRightarrow{F_*
      \hat\epsilon G_*} F_*G_* \xRightarrow{\hat\epsilon'}
    1_{\sD^\sJ}$ similarly
  \end{itemize}

  We now fix a diagram $\func{K}{\sJ}{\sD}$ for the remainder of the
  proof, and we take $\Func{\lambda}{\Delta\lim FK}{FK}$ and
  $\Func{\mu}{\Delta \lim K}{K}$ to be the limit cones over $FK$ and
  $K$ respectively.

  We noted above that the isomorphism $\isom{\theta}{G\lim}{\lim G_*}$
  commutes with the units and counits of the composite adjunctions;
  for the counits, this means that the following diagram commutes:
  $$
  \begin{tikzcd}
    \Delta FG\lim
      \ar[r, Rightarrow, "\bar\epsilon"]
      \ar[d, Rightarrow, "\Delta F\theta"']
    & 1_{\sD^\sJ} \\
    \Delta F\lim G_*
      \ar[ur, Rightarrow, "\bar{\hat\epsilon}"']
  \end{tikzcd}
  $$
  (as $\Delta F\lim G_* = F_*\Delta \lim G_*$), and at the
  component~$K$, this diagram becomes
  $$
  \begin{tikzcd}
    \Delta FG\lim K
      \ar[r, Rightarrow, "\bar\epsilon_K"]
      \ar[d, Rightarrow, "\Delta F\theta_K"']
    & K \\
    \Delta F\lim GK
      \ar[ur, Rightarrow, "\bar{\hat\epsilon}_K"']
  \end{tikzcd}
  $$
  Now from the above expressions for $\bar\epsilon$ and
  $\bar{\hat\epsilon}$, this diagram can be expanded to give
  $$
  \begin{tikzcd}
    \Delta FG\lim K
      \ar[r, Rightarrow, "\Delta\epsilon_{\lim K}"]
      \ar[d, Rightarrow, "\Delta F\theta_K"']
    & \Delta\lim K
      \ar[r, Rightarrow, "\epsilon'_K"]
    & K
      \ar[d, Rightarrow, "1_K"] \\
    \Delta F\lim GK
      \ar[r, Rightarrow, "F\hat\epsilon_{GK}"']
    & FGK
      \ar[r, Rightarrow, "\hat\epsilon'_K"']
    & K
  \end{tikzcd}
  $$
  Using $\epsilon'_K=\mu$, $\hat\epsilon_{GK}=\lambda$ and
  $\hat\epsilon'_K=\epsilon K$, this diagram can be rewritten as
  $$
  \begin{tikzcd}[column sep = large]
    F\Delta G\lim K
      \ar[r, Rightarrow, "\mu\cdot\Delta\epsilon_{\lim K}"]
      \ar[d, Rightarrow, "F\Delta\theta_K"']
    & K
      \ar[d, Rightarrow, "1_K"] \\
    F\Delta\lim GK
      \ar[r, Rightarrow, "\epsilon K\cdot F\lambda"']
    & K
  \end{tikzcd}
  $$
  This is a diagram in $\sD^\sJ$, which we can transpose across the
  adjunction $F_*\ladj G_*$ to give
  $$
  \begin{tikzcd}[column sep = large]
    \Delta G\lim K
      \ar[r, Rightarrow, "(\mu\cdot\Delta\epsilon_{\lim K})^\flat"]
      \ar[d, Rightarrow, "\Delta\theta_K"']
    & GK
      \ar[d, Rightarrow, "1_{GK}"] \\
    \Delta\lim GK
      \ar[r, Rightarrow, "(\epsilon K\cdot F\lambda)^\flat"']
    & GK
  \end{tikzcd}
  $$
  We can calculate the transposes of the two composites using the
  naturality of the transposition (see the note after Definition
  4.1.1, top of page 117; we take $f^\sharp=\Delta\epsilon_{\lim K}$,
  $k=\mu$, and then $f^\sharp=\epsilon K$ and $h=\lambda$).  We obtain
  $(\mu\cdot\Delta\epsilon_{\lim K})^\flat=G\mu\cdot
  (\Delta\epsilon_{\lim K})^\flat$ and
  $(\epsilon K\cdot F\lambda)^\flat=(\epsilon K)^\flat\cdot\lambda$.
  We can now use Proposition 4.2.6 to calculate the remaining two
  transposes.  For the first, we have
  $$(\Delta\epsilon_{\lim K})^\flat = \Delta G\lim K
  \xRightarrow{(\hat\eta')_{\Delta G \lim K}} GF\Delta G \lim K
  \xRightarrow{G\Delta\epsilon_{\lim K}} G\Delta \lim K$$
  As $\hat\eta'_L=\eta L$, this gives $(\Delta\epsilon_{\lim K})^\flat
  = G\Delta \epsilon_{\lim K}\cdot \eta\Delta G\lim K$.  Since
  $\Delta$~is the constant functor, the component
  of this at $j\in\sJ$ is $G\epsilon_{\lim K}\cdot \eta_{G\lim
    K}=1_{G\lim K}$ from the triangle identities, and therefore
  $(\Delta\epsilon_{\lim K})^\flat = \Delta 1_{G\lim K}=1_{\Delta
    G\lim K}$.  The second transposition is
  $$(\epsilon K)^\flat = GK \xRightarrow{(\hat\eta')_{GK}} GFGK
  \xRightarrow{G\epsilon K} GK$$
  so $(\epsilon K)^\flat = G\epsilon K\cdot \eta GK = (G\epsilon\cdot
  \eta G)K=1_G K=1_{GK}$.

  Therefore our transposed square simplifies to
  $$
  \begin{tikzcd}
    \Delta G\lim K
      \ar[r, Rightarrow, "G\mu"]
      \ar[d, Rightarrow, "\Delta\theta_K"']
    & GK
      \ar[d, Rightarrow, "1_{GK}"] \\
    \Delta\lim GK
      \ar[r, Rightarrow, "\lambda"']
    & GK
  \end{tikzcd}
  $$
  It follows that the unique morphism making the diagram
  \eqref{eq:4-5-4-3.3.i} commute is $f=\theta_K$, and this is a
  natural isomorphism.  Hence $G$~preserves the limit of~$K$ as we
  wished to show.
\end{proof}

\end{document}

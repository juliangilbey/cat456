\documentclass[../../solutions]{subfiles}
\title{Template}
\author{}
\begin{document}
\maketitle

% \settheorem{1}{2}{3}
% \begin{lemma}
%   
% \end{lemma}
% \popthm

\setexercise{4}{5}{7}
\begin{exercise}
  Consider a reflective subcategory inclusion $\sD\hookrightarrow\sC$
  with reflector $\func{L}{\sC}{\sD}$.

  \begin{enumerate}[label=(\roman*)]
  \item Show that $\eta L = L\eta$, and that these natural
    transformations are isomorphisms.

  \item Show that an object $c\in\sC$ is in the \textbf{essential
      image} of the inclusion $\sD\hookrightarrow\sC$, meaning that it
    is isomorphic to an object in the subcategory $\sD$, if and only
    if $\eta_c$~is an isomorphism.

  \item Show that the essential image of $\sD$ consists of those
    objects $c$ that are \textbf{local} for the class of morphisms
    that are inverted by~$L$.  That is, $c$~is in the essential image
    if and only if the pre-composition functions
    $$\sC(b,c) \xrightarrow{f^*} \sC(a,c)$$
    are isomorphisms for all maps $\func{f}{a}{b}$ in $\sC$ for which
    $Lf$ is an isomorphism in~$\sD$.  This explains why the reflector
    is also referred to as ``localization.''
  \end{enumerate}
\end{exercise}

\begin{proof}
  To clarify what is happening, we will sometimes write the inclusion
  $\func{i}{\sD}{\sC}$ explicitly in this solution.

  \begin{enumerate}[label=(\roman*)]
  \item Writing the inclusion explicitly, the question becomes to show
    that $\eta iL=iL\eta$ (as $\Func{\eta}{1_{\sC}}{iL}$).

    From the triangle identities, we have $\epsilon L\cdot L\eta=1_L$.
    Post-composition with~$i$ gives $i\epsilon L\cdot iL\eta=1_{iL}$.
    As $\epsilon$~is an isomorphism, so is $i\epsilon L$, and hence
    $iL\eta$~is the inverse of~$i\epsilon L$ and is an isomorphism.

    From the other triangle identity, $i\epsilon\cdot\eta i=1_i$ and
    so $\eta i$ is the inverse of $i\epsilon$ (and both are
    isomorphisms).  Pre-composition with~$L$ gives
    $i\epsilon L\cdot \eta iL=1_{iL}$ and so $\eta iL$ is the inverse
    of $i\epsilon L$.

    Combining these two results, we have $iL\eta=\eta iL$ and these
    are isomorphisms.

  \item If $\eta_c$ is an isomorphism, then $c\cong Lc\in\sD$, so is
    in the essential image of $\sD\hookrightarrow\sC$.

    Conversely, assume that $c$ is in the essential image, so
    $\isom{\phi}{c}{c'}$ for some $c'\in\sD$.  From the naturality
    square
    $$
    \begin{tikzcd}
      c \ar[r, "\eta_c"] \ar[d, "\phi"']
      & Lc \ar[d, "L\phi"] \\
      c' \ar[r, "\eta_{c'}"']
      & Lc'
    \end{tikzcd}
    $$
    it follows that $\eta_c$ is an isomorphism if and only if
    $\eta_{c'}$~is.  We therefore need only show that $\eta_c$~is an
    isomorphism in the case that $c\in\sD$.  But in this case, as
    $c=ic$ and $\eta i$~is an isomorphism from the proof of~(i),
    $\eta_c=\eta_{ic}$ is an isomorphism.

  \item We first prove that if $c$ is in the essential image, then $c$
    is local for this class of morphisms.  Suppose that $c\in\sD$.
    Then for any $\func{f}{a}{b}$ in~$\sC$, we have the commuting
    diagram
    $$
    \begin{tikzcd}
      \sD(Lb,c) \ar[r, "\cong"] \ar[d, "(Lf)^*"']
      & \sC(b,c) \ar[d, "f^*"] \\
      \sD(La,c) \ar[r, "\cong"']
      & \sC(a,c)
    \end{tikzcd}
    $$
    where the isomorphisms are given by the adjuction $L\ladj i$.  If
    $Lf$ is an isomorphism, then so is $(Lf)^*$ by Lemma 1.2.3, and
    therefore so is~$f^*$.

    Now suppose that $c$ is in the essential image (but not
    necessarily in~$\sD$), and that $\isom{\phi}{c}{c'}$ is an
    isomorphism to some $c'\in\sD$.  If $\func{f}{a}{b}$ and $Lf$~is
    an isomorphism, then $\func{f^*}{\sC(b,c')}{\sC(a,c')}$ is an
    isomorphism.  The following commutative diagram then shows that
    $\func{f^*}{\sC(b,c)}{\sC(a,c)}$ is too:
    $$
    \begin{tikzcd}
      \sC(b,c) \ar[r, "f^*"] \ar[d, "\phi_*"']
      & \sC(a,c) \ar[d, "\phi_*"] \\
      \sC(b,c') \ar[r, "f^*"']
      & \sC(a,c')
    \end{tikzcd}
    $$

    For the converse, suppose that $\sC(b,c) \xrightarrow{f^*}
    \sC(a,c)$ is an isomorphism for every map $\func{f}{a}{b}$
    in~$\sC$ for which $Lf$~is an isomorphism.  By~(ii), it suffices
    to show that $\eta_c$~is an isomorphism.  Now $L\eta_c$~is an
    isomorphism by~(i), so
    $$\sC(Lc,c) \xrightarrow{(\eta_c)^*} \sC(c,c)$$
    is an isomorphism.  In particular, there is some morphism
    $\func{h}{Lc}{c}$ with $h\eta_c=1_c$, so $h$~is a split
    epimorphism.

    Post-composing this last equation with $L$ gives $Lh\cdot
    L\eta_c=1_{Lc}$, and $L\eta_c$~is an isomorphism by~(i), so
    $Lh$~is also an isomorphism.  As $Lc$~is in the essential image,
    the first part of this argument now shows that $\sC(c,Lc)
    \xrightarrow{h^*} \sC(Lc,Lc)$ is an isomorphism.  Thus there is
    some morphism $\func{k}{c}{Lc}$ with $kh=1_{Lc}$.  Therefore
    $h$~is also a (split) monomorphism, and from Exercise 1.2.vi it
    follows that $h$~is an isomorphism, and hence so is~$\eta_c$.
  \end{enumerate}
\end{proof}

\end{document}


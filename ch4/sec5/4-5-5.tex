\documentclass[../../solutions]{subfiles}
\title{Template}
\author{}
\begin{document}
\maketitle

% \settheorem{1}{2}{3}
% \begin{lemma}
%   
% \end{lemma}
% \popthm

\setexercise{4}{5}{5}
\begin{exercise}
  Show that a morphism $\func{f}{x}{y}$ in $\sC$ is a monomorphism if
  and only if the square
  $$
  \begin{tikzcd}
    x \ar[r, "1_x"] \ar[d, "1_x"']
    & x \ar[d, "f"] \\
    x \ar[r, "f"']
    & y
  \end{tikzcd}
  $$
  is a pullback.  Conclude that right adjoints preserve monomorphisms,
  and that left adjoints preserve epimorphisms.
\end{exercise}

\begin{proof}
  Consider a cone over the diagram $x \xrightarrow{f} y \xleftarrow{f}
  x$:
  $$
  \begin{tikzcd}
    z
      \ar[rrd, bend left, "h"]
      \ar[rd, dashed, "g"]
      \ar[rdd, bend right, "k"'] \\
    & x \ar[r, "1_x"] \ar[d, "1_x"']
    & x \ar[d, "f"] \\
    & x \ar[r, "f"']
    & y
  \end{tikzcd}
  $$
  There exists a morphism $g$ making the whole diagram commute if and
  only if $h=k$, in which case $g=h=k$ is unique.  Therefore, this
  square is a pullback if and only if $fh=fk$ implies $h=k$, that is,
  if and only if $f$~is a monomorphism.

  Since right adjoints preserve limits, if $G$~is a right adjoint and
  $f$~is a monomorphism, then
  $$
  \begin{tikzcd}
    Gx \ar[r, "1_{Gx}"] \ar[d, "1_{Gx}"']
    & Gx \ar[d, "Gf"] \\
    Gx \ar[r, "Gf"']
    & Gy
  \end{tikzcd}
  $$
  is a pullback, and so $Gf$ is a monomorphism.  Therefore right
  adjoints preserve monomorphisms.  Dually, it follows that left
  adjoints preserve epimorphisms (and a morphism is an epimorphism if
  and only if the dual square is a pushout).
\end{proof}

\end{document}


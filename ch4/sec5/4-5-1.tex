\documentclass[../../solutions]{subfiles}
\title{Template}
\author{}
\begin{document}
\maketitle

% \settheorem{1}{2}{3}
% \begin{lemma}
%   
% \end{lemma}
% \popthm

\setexercise{4}{5}{1}
\begin{exercise}
  When does the unique functor $\func{!}{\sC}{\one}$ have a left
  adjoint?  When does it have a right adjoint?
\end{exercise}

\begin{proof}
  $!$ has a left adjoint $\func{L}{\one}{\sC}$ if and only if there is
  a natural isomorphism $\sC(L0,c)\cong \one(0,!c)$, where $0$~is the
  unique object in~$\one$.  But $\one(0,!c)$ is a singleton set
  (containing only $1_0$), so this holds if and only if $L0$ is an
  initial object in~$\sC$.  Therefore $!$~has a left adjoint if and
  only if $\sC$~has an initial object.

  Dually, $!$ has a right adjoint $\func{R}{\one}{\sC}$ if and only if
  $\sC$~has a terminal object, in which case $R0$~is such a terminal
  object.
\end{proof}

\end{document}


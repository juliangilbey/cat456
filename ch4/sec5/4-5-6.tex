\documentclass[../../solutions]{subfiles}
\title{Template}
\author{}
\begin{document}
\maketitle

% \settheorem{1}{2}{3}
% \begin{lemma}
%   
% \end{lemma}
% \popthm

\setexercise{4}{5}{6}
\begin{exercise}
  Prove Lemma 4.5.13.
\end{exercise}

\begin{proof}
  We only prove the first half of the lemma; the results about the
  left adjoint~$F$ follow immediately by duality.

  Statement (iii) follows immediately from statements (i) and (ii)
  using Exercise 1.2.vi, noting that (by definition) a natural
  transformation is an isomorphism if and only if every component is.
  It therefore remains to prove statements (i) and~(ii) for a right
  adjoint~$G$.

  It is possible to prove these directly as four separate results,
  showing for~(i), for example, that if $G$~is faithful, then
  $h\cdot\epsilon_d = k\cdot\epsilon_d$ implies that $h=k$.  But it is
  more efficient and instructive to prove both together, as is done in
  \cite[90--91]{catworking}; we present an expanded version of the
  argument given there.

  We first note that $G$ is faithful or full just when the morphism
  $\sD(x,y)\xrightarrow{G} \sC(Gx,Gy)$ is monic or epic respectively
  for all $x$ and~$y$.  This morphism is natural in $x$ and~$y$.  We
  can compose this with the adjunction isomorphism
  $\isom{\phi_{x,y}}{\sC(Gx,Gy)}{\sD(FGx,y)}$ to give the composite
  morphism
  $$\sD(x,y)\xrightarrow{G} \sC(Gx,Gy) \xrightarrow{\phi_{x,y}}
  \sD(FGx,y)$$
  which is natural in $x$ and $y$; $G$ is faithful or full just when
  this is monic or epic for all $x$ and~$y$.  For each $x\in\sD$, this
  composite gives a natural transformation $\sD(x,-)\Rightarrow
  \sD(FGx,-)$, which by the Yoneda lemma is given by $f^*$ for some
  $\func{f}{FGx}{x}$ (Corollary 2.2.8).  Taking $y=x$, and observing
  that $1_x$ is sent to~$\epsilon_x$, we see that this natural
  transformation is $(\epsilon_x)^*$.

  The next observation is that if $\sA$~is a small category and
  $\sB$~has all pullbacks, then a natural transformation between two
  functors $\pfunc{H,K}{\sA}{\sB}$ is a monomorphism (in the functor
  category~$\sB^\sA$) if and only if every component is a
  monomorphism, and similarly for epimorphisms (now requiring $\sB$~to
  have all pushouts).  This follows because the natural transformation
  $\Func{\alpha}{H}{K}$ is a monomorphism if and only if the diagram
  $$
  \begin{tikzcd}
    H \ar[r, Rightarrow, "1_H"] \ar[d, Rightarrow, "1_H"']
    & H \ar[d, Rightarrow, "\alpha"] \\
    H \ar[r, Rightarrow, "\alpha"']
    & K
  \end{tikzcd}
  $$
  is a pullback, by Exercise \ref{ex:4.5.v}, and by Proposition 3.3.9,
  as $\sB$~admits all pullbacks, this is a pullback if and only if
  $$
  \begin{tikzcd}
    Hx \ar[r, rightarrow, "1_{Hx}"] \ar[d, rightarrow, "1_{Hx}"']
    & Hx \ar[d, rightarrow, "\alpha_x"] \\
    Hx \ar[r, rightarrow, "\alpha_x"']
    & Kx
  \end{tikzcd}
  $$
  is a pullback for every component of~$\alpha$, so if and only if
  $\alpha_x$ is a monomorphism for every~$x$.

  In this case, as $\sD(x,-)$ lies in the functor category $\Set^\sD$,
  and $\Set$~is complete and cocomplete, it follows that $G$~is
  faithful or full if and only if $(\epsilon_x)^*$ is monic or epic
  for every $x\in\sD$.  This argument requires $\sD$ to be small;
  Mac~Lane, though, allows (large) functor categories $\sB^\sA$ with
  $\sA$~any category, and the equivalent of Proposition 3.3.9 still
  works in this case, so this argument still holds.

  We next show that if $\func{f}{x}{y}$ is a morphism in a
  category~$\sE$, then $f$~is an epimorphism if and only if
  $\Func{f^*}{\sE(y,-)}{\sE(x,-)}$~is a monomorphism, and $f$~is a
  split monomorphism if and only if $f^*$~is an epimorphism.  For the
  first part, for any $h\in\sE(y,z)$, $f^*h=hf$, so $f^*h=f^*k$ if and
  only if $hf=kf$.  Therefore $f$~is an epimorphism if and only if
  every component of~$f^*$ is a monomorphism, which holds if and only
  if $f^*$~is a monomorphism by the above.  For the second part, if
  $f^*$~is an epimorphism, then there is some $\hat{f}\in\sE(y,x)$
  with $f^*\hat{f}=1_x$, so $\hat{f}f=1_x$ and $f$~is a split
  monomorphism.  Conversely, if $f$~is a split monomorphism with
  $\hat{f}f=1_x$, then for any $h\in\sE(x,z)$, $f^*(h\hat{f})=h$, so
  $f^*$~is an epimorphism.

  Applying this to the above, we find that $G$ is faithful if and only
  if every component of~$\epsilon$ is an epimorphism, and $G$~is full
  if and only if every component of~$\epsilon$ is a split
  monomorphism.
\end{proof}

\end{document}


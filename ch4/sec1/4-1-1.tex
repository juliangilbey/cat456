\documentclass[../../solutions]{subfiles}
\title{Template}
\author{}
\begin{document}
\maketitle

% \settheorem{1}{2}{3}
% \begin{lemma}
%   
% \end{lemma}
% \popthm

\setexercise{4}{1}{1}
\begin{exercise}
Show that functors $F : C \to D$ and $G : D \to C$ and bijections $D(Fc, d) \cong C(c, Gd)$ for each $c \in C$ and $d \in D$ define an adjunction if and only if these bijections induce a bijection between commutative squares (4.1.4). That is, prove Lemma 4.1.3.
\end{exercise}

\begin{proof}

Let $C$ and $D$ be categories and let $F : C \to D , G : D \to C$ be functors.

Suppose that we are given, for any $c\in C, d \in D$, a bijection \[ \varphi_{c,d}: D(Fc, d) \cong C(c, Gd) \]
Denote its inverse by $\psi_{c,d}$. We will omit the subscripts if they are understood.

Recall that this collection of bijections forms an adjunction (i.e. is natural in $C$ and $D$) if and only if the following identities hold:
\begin{equation} \label{adj1}
\varphi(k\cdot f^\sharp) = Gk \cdot f^\flat
\end{equation}
for any $c \in C, d, d' \in D$, $k : d\to d'$ and morphisms as pictured below
\[
\xymatrix{
 Fc\ar[dr] \ar[r]^{f^\sharp} & d \ar[d]^{k}\\
  & d'
} \ \ \ \ \ \ \ \ \ 
\xymatrix{
 c\ar[dr] \ar[r]^{f^\flat} & Gd \ar[d]^{Gk}\\
 & Gd'
} 
\]
where $(f^\sharp, f^\flat)$ is an adjoint pair; and similarly
\begin{equation} \label{adj2}
\psi(g^\flat  h) = g^\sharp \cdot Fh
\end{equation}
for any $c,c'\in C, d \in D$, $h : c \to c'$ and morphisms as pictured below
\[
\xymatrix{
 Fc\ar[d]^{Fh} \ar[dr] & \\
 Fc'\ar[r]^{g^\sharp} & d'
} \ \ \ \ \ \ \ \ \ 
\xymatrix{
 c\ar[d]^{h} \ar[dr] & \\
 c'\ar[r]^{g^\flat} & Gd'
} 
\]
where $(g^\sharp, g^\flat)$ is an adjoint pair.

Our goal is to show that the equations in (\ref{adj1}) and (\ref{adj2}) hold if and only if the left square below commutes precisely when the right square commutes:
\[
\xymatrix{
 Fc\ar[d]^{Fh} \ar[r]^{f^\sharp} & d \ar[d]^{k}\\
 Fc'\ar[r]^{g^\sharp} & d'
} \ \ \ \ \ \ \ \ \ 
\xymatrix{
 c\ar[d]^{h} \ar[r]^{f^\flat} & Gd \ar[d]^{Gk}\\
 c'\ar[r]^{g^\flat} & Gd'
} 
\]
Indeed, suppose first that the equations in (\ref{adj1}) and (\ref{adj2}) hold.

Suppose that the left square commutes. Then we have

\[
Gk \cdot f^\flat = \varphi(k \cdot f^\sharp) = \varphi(g^\sharp Fh) = g^\flat h 
\]
Where we use commutativity of the left square and the identities in (\ref{adj1}), (\ref{adj2}). So the right square commutes. 

Similarly, if the right square commutes, then we have
\[
k \cdot f^\sharp = \psi(Gk \cdot f^\flat) = \psi (g^\flat h) = g^\sharp \cdot Fh
\]
so the left square commutes.

Conversely, suppose that the left square commutes exactly when the right square commutes. Then in particular, in the context of (\ref{adj1}) we can choose $c' = c, h = 1_{c}, g^\flat =Gk \cdot f^\flat$, making the right square commute by construction. Then the left square commutes as well. But $F(1_c)=1_{Fc}$ and $g^\sharp = \psi(g^\flat)=\psi(Gk\cdot f^\flat)$, so commutativity of the left square means $\psi(Gk\cdot f) = k\cdot f^\sharp$. Since $\varphi$ is an inverse of $\psi$, this yields equation (\ref{adj1}). 

Similarly, in the context of (\ref{adj2}) we can choose $d = d', k = 1_d$ and $f^\sharp = g^\sharp \cdot Fh$. Then the left square commutes by construction, hence the right square commutes, which precisely amounts to equation (\ref{adj2})
\end{proof}

\end{document}


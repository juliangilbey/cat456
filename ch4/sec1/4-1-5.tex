\documentclass[../../solutions]{subfiles}
\title{Template}
\author{}
\begin{document}
\maketitle

% \settheorem{1}{2}{3}
% \begin{lemma}
%   
% \end{lemma}
% \popthm

\setexercise{4}{1}{5}
\begin{exercise}
  Use Lemma 4.1.3 to show that if $\afunc{F}{G}{\sC}{\sD}$ are adjoint
  functors, with $F\ladj G$, then the comma categories $F\downarrow
  \sD$ and $\sC\downarrow G$, introduced in Exercise 1.3.vi, are
  isomorphic, via an isomorphism that commutes with the forgetful
  functors to $\sC\times\sD$.
\end{exercise}

\begin{proof}
  $F\downarrow\sD$ has objects
  $(c\in\sC, d\in\sD, \func{f^\sharp}{Fc}{d})$, and a morphism
  $(c,d,f^\sharp)\to(c',d',g^\sharp)$ is a pair of morphisms
  $(\func{h}{c}{c'}, \func{k}{d}{d'})$ such that this square commutes
  in~$\sD$:
  \begin{center}
    \begin{tikzcd}
      Fc \ar[r, "f^\sharp"] \ar[d, "Fh"'] & d \ar[d, "k"] \\
      Fc' \ar[r, "g^\sharp"'] & d'
    \end{tikzcd}
  \end{center}

  Similarly, $\sC\downarrow G$ has objects
  $(c\in\sC, d\in\sD, \func{f^\flat}{c}{Gd})$, and a morphism
  $(c,d,f^\flat)\to(c',d',g^\flat)$ is a pair of morphisms
  $(\func{h}{c}{c'}, \func{k}{d}{d'})$ such that this square commutes
  in~$\sC$:
  \begin{center}
    \begin{tikzcd}
      c \ar[r, "f^\flat"] \ar[d, "h"'] & Gd \ar[d, "Gk"] \\
      c' \ar[r, "g^\flat"'] & Gd'
    \end{tikzcd}
  \end{center}

  We define the isomorphism
  $\isom{\eta}{F\downarrow\sD}{\sC\downarrow G}$ by
  $\eta(c,d,f^\sharp)=(c,d,f^\flat)$ on objects, where $f^\flat$~is
  the transpose of~$f^\sharp$ under the given adjunction $F\ladj G$,
  and $\eta(h,k)=(h,k)$ on morphisms.  Then by Lemma 4.1.3,
  $\func{(h,k)}{(c,d,f^\sharp)}{(c',d',g^\sharp)}$ is a morphism in
  $F\downarrow\sD$ if and only if
  $\func{(h,k)}{(c,d,f^\flat)}{(c',d',g^\flat)}$ is a morphism in
  $\sD\downarrow G$, showing that $\eta$~is a well-defined functor.
  It clearly has an inverse defined by
  $\epsilon(c,d,f^\flat)=\epsilon(c,d,f^\sharp)$, where again
  $f^\sharp$~is the transpose of~$f^\flat$, and this is well-defined
  for the same reason.  Therefore $\eta$~is an isomorphism of
  categories.

  Finally, this isomorphism clearly commutes with the forgetful
  functors to $\sC\times \sD$ (in the sense that the forgetful
  functors~$U$ satisfy
  $U_{\sC\downarrow G}\cdot\eta = \hat{\eta}\cdot U_{F\downarrow
    \sD}$, where $\hat{\eta}$ is $\eta$ restricted to
  $\sC\times \sD$).
\end{proof}

\end{document}


\documentclass[../../solutions]{subfiles}
\title{Template}
\author{Julian Gilbey}
\begin{document}
\maketitle

% \settheorem{1}{2}{3}
% \begin{lemma}
%
% \end{lemma}
% \popthm

\setexercise{4}{1}{3}
\begin{exercise}
  Show that any triple of adjoint functors
  \[\begin{tikzcd}[column sep=2cm]
      \sC \ar[r, "U"{description}] & \sD
      \ar[l, "L", swap, bend right=40, "\tadj"{below=0.1cm}]
      \ar[l, "R", bend left=40, "\tadj"{above=0.1cm}]
  \end{tikzcd}\]
  gives rise to a canonical adjunction $LU\ladj RU$ between the
  induced endofunctors of $\sC$.
\end{exercise}

\begin{proof}
  We wish to show that $\sC(LUx, y)\cong \sC(x, RUy)$ for any pair
  $x,y\in\sC$, with the isomorphism natural in $x$ and~$y$.

  We have
  \[\sC(LUx, y) \cong \sD(Ux, Uy) \cong \sC(x, RUy).\]
  The first isomorphism is natural in $y$ and the second is natural
  in~$x$.  To show that the first isomorphism is natural in~$x$,
  suppose that $\func{h}{x}{x'}$ in~$\sC$ and consider the following
  diagram, where $\alpha_{-,-}$ is the transposition isomorphism:

  \[\begin{tikzcd}
      \sC(LUx', y)
        \ar[r, "\alpha_{Ux',y}"]
        \ar[d, "(LUh)^*", swap]
      & \sD(Ux',Uy)
        \ar[d, "(Uh)^*"] \\
      \sC(LUx, y)
        \ar[r, "\alpha_{Ux,y}"]
      & \sD(Ux,Uy)
  \end{tikzcd}\]

  As $\alpha_{c,d}$ is natural in $c$ and $\func{Uh}{Ux}{Ux'}$ is a
  morphism in~$\sD$, this square is commutative as required.  A
  similar argument shows that the second isomorphism is natural
  in~$y$, and we are done.
\end{proof}

\end{document}

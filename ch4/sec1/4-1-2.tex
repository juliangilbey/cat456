\documentclass[../../solutions]{subfiles}
\title{Template}
\author{Julian Gilbey}
\begin{document}
\maketitle

\setexercise{4}{1}{2}
\begin{exercise}
  Define left and right adjoints to
  \begin{enumerate}[(i)]
  \item $\func{\ob}{\Cat}{\Set}$,
  \item $\func{\text{Vert}}{\cat{Graph}}{\Set}$, and
  \item $\func{\text{Vert}}{\cat{DirGraph}}{\Set}$.
  \end{enumerate}
\end{exercise}

\begin{proof}
  We write $L$ and $R$ for the left and right adjoints, respectively.
  The arguments are very similar for all three cases, so we only give
  full details for~(i).

  \begin{enumerate}[(i)]
  \item $\func{L}{\Set}{\Cat}$ sends a set $X$ to the category $LX$
    with objects~$X$ and only the identity morphisms.  We must show
    that we have a natural isomorphism
    $$\Cat(LX, \sC) \cong \Set(X, \ob\sC).$$
    Every functor $\func{F}{LX}{\sC}$ is completely determined by the
    action on objects as there are no non-identity morphisms, so we
    obtain a function in $\Set(X, \ob\sC)$.  Every function in this
    set is obtained from a unique functor, so these sets are
    isomorphic.

    This is also a natural isomorphism: a function $\func{f}{X}{Y}$
    sends a functor $\func{F}{LY}{\sC}$ to $F\cdot Lf$, which sends the
    object $x\in X$ to $F(fx)$.  The underlying object morphism
    $\ob F$ is sent to $(\ob F)\cdot f$, and this also sends $x$ to
    $F(fx)$, so the isomorphism is natural in~$X$.  It is likewise
    natural in~$\sC$ by similar reasoning.  Thus $L$~is a left
    adjoint.

    We now define $\func{R}{\Set}{\Cat}$ to send a set $X$ to the
    category $RX$ with objects~$X$ and a unique morphism $x\to y$ for
    each $x,y\in X$.  We must show that we have a natural isomorphism
    $$\Set(\ob\sC, X) \cong \Cat(\sC, RX).$$
    Every functor $\func{F}{\sC}{RX}$ is completely determined by the
    action on objects, as every morphism $x\to y$ in~$\sC$ is sent to
    the unique morphism $Fx\to Fy$ in~$RX$.  This gives a function in
    $\Set(\ob\sC, X)$, and every such function gives rise to a unique
    functor.  So these sets are isomorphic.  In the same way as
    before, this collection of isomorphisms is natural in $X$
    and~$\sC$, so $R$~is a right adjoint for~$\ob$.

  \item We recall that a graph homomorphism $\func{\phi}{G_1}{G_2}$ is
    a function sending every vertex of~$G_1$ to a vertex of~$G_2$
    preserving incidence relations (so if $xy$ is an edge in~$G_1$,
    then $\phi(x)\phi(y)$ is an edge in~$G_2$, and we may have $x=y$);
    see page~4.  We let $\func{L}{\Set}{\cat{Graph}}$ send a set~$X$
    to the empty graph with vertex set~$X$.  In the same way as
    before, we have a natural isomorphism
    $$\cat{Graph}(LX, G) \cong \Set(X, \text{Vert}(G));$$
    a graph homomorphism $\func{\phi}{LX}{G}$ is completely determined
    by the action on vertices, and as $LX$~has no edges, every
    function from~$X$ to the vertices of~$G$ gives a homomorphism.

    Likewise, we define $\func{R}{\Set}{\cat{Graph}}$ to send a set
    $X$ to the complete graph with vertex set~$X$ (including a loop at
    every vertex).  We then have a natural isomorphism
    $$\Set(\text{vert}(G), X) \cong \cat{Graph}(G, RX),$$
    as every possible function from the vertices of~$G$ to~$X$ gives a
    graph homomorphism, as every pair $x,y\in X$ is connected by an
    edge.

  \item This case is almost identical to~(ii): we again let
    $\func{L}{\Set}{\cat{DirGraph}}$ send a set~$X$ to the empty
    directed graph with vertex set~$X$, and we now let
    $\func{R}{\Set}{\cat{DirGraph}}$ send a set $X$ to the complete
    directed graph with vertex set~$X$: for every pair of vertices
    $x,y\in X$, we let $(x,y)$ be a directed edge in $RX$.
  \end{enumerate}
\end{proof}

\end{document}


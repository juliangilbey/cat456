\documentclass[../../solutions]{subfiles}
\title{Template}
\author{}
\begin{document}
\maketitle

% \settheorem{1}{2}{3}
% \begin{lemma}
%   
% \end{lemma}
% \popthm

\setexercise{5}{3}{1}
\begin{exercise}
  Reading between the lines in the proof of Proposition 5.3.3, prove
  that the category of algebras for an idempotent monad on~$\sC$
  defines a reflective subcategory of~$\sC$ and moreover is equivalent
  to the Kleisli category for that monad.
\end{exercise}

\begin{proof}
  Suppose that $(T,\eta,\mu)$ is an idempotent monad on~$\sC$; by
  Exercise 5.1.iii, this means that $\mu$~is invertible and
  $\eta T=T\eta$.  The first paragraph of the proof of Proposition
  5.3.3 then follows unchanged (just replacing $L$ by~$T$).  Thus the
  map $\func{U^T}{\sC^T}{\sC}$ is injective on objects and morphisms
  (the latter by the definition of~$U^T$ on morphisms), so we may
  treat $\sC^T$ as a subcategory of~$\sC$.

  It is a full subcategory, as if
  $(A,\eta_A^{-1}), (B,\eta_B^{-1})\in\sC^T$ and $\func{f}{A}{B}$ is a
  morphism in~$\sC$, then $f$~is an algebra morphism, since the
  diagram
  $$
  \begin{tikzcd}
    TA
    \ar[r, "Tf"]
    \ar[d, "\eta_A^{-1}"']
    & TB
    \ar[d, "\eta_B^{-1}"]
    \\
    A
    \ar[r, "f"']
    & B
  \end{tikzcd}
  $$
  commutes by the naturality of $\eta$.  As $U^T$ has a left
  adjoint~$F^T$, it follows that $\sC^T$~is a reflective subcategory
  of~$\sC$.

  Now consider the canonical comparison functor from $\sC_T$
  to~$\sC^T$:
  $$
  \begin{tikzcd}[row sep=huge, column sep=large]
    \sC_T \ar[rr, "K"]
    \ar[dr, shift left=1ex, pos=0.4, "U_T" {inner sep=0mm}]
    \ar[dr, phantom, "\scriptstyle\top" {anchor=center, rotate=-45,
      inner sep=.5mm}]
    \ar[dr, shift right=1ex, leftarrow, pos=0.4, "F_T"' {inner sep=0mm}]
    && \sC^T
    \ar[dl, shift left=1ex, pos=0.4, "U^T" {inner sep=0mm}]
    \ar[dl, phantom, "\scriptstyle\bot" {anchor=center, rotate=45,
      inner sep=.5mm}]
    \ar[dl, shift right=1ex, leftarrow, pos=0.4, "F^T"' {inner sep=0mm}]
    \\
    & \sC
  \end{tikzcd}
  $$
  $K$ is full and faithful by Lemma 5.2.13, so we only need to show
  that it is essentially surjective on objects to demonstrate that
  $\sC_T$~is equivalent to~$\sC^T$ (Theorem 1.5.9).  By Lemma 5.2.13
  again, the image of~$\sC_T$ is the free $T$-algebras, $(Tc,\mu_c)$
  for $c\in\sC$.  But as $\sC^T$~is a reflective subcategory of~$\sC$,
  for every $(A,a)\in \sC^T$, the component of the counit
  $\epsilon_{(A,a)}\colon (TA,\mu_A)\xrightarrow{a}(A,a)$ is an
  isomorphism, so every object in~$\sC^T$ is isomorphic to~$Kc$ for
  some $c\in\sC_T$.  Thus $K$~defines an equivalence of categories.
\end{proof}

\end{document}


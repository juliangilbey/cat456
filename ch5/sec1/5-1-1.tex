\documentclass[../../solutions]{subfiles}
\title{Template}
\author{}
\begin{document}
\maketitle

% \settheorem{1}{2}{3}
% \begin{lemma}
%
% \end{lemma}
% \popthm

\setexercise{5}{1}{1}
\begin{exercise}
  Suppose $\sV$ is a monoidal category (see \S E.2), i.e., a category
  with a bifunctor $\func{\otimes}{\sV \times \sV}{\sV}$ that is
  associative up to coherent natural isomorphism and a unit object
  ${*}\in\sV$ with natural isomorphisms
  $v\otimes {*} \cong v \cong {*} \otimes v$.  Suppose also that
  $\sV$~has finite coproducts and that the bifunctor~$\otimes$
  preserves them in each variable.  Show that
  $T(X) = \coprod_{n\ge0} X^{\otimes n}$ defines a monad on~$\sV$ by
  defining natural transformations $\Func{\eta}{1_\sV}{T}$ and
  $\Func{\mu}{T^2}{T}$ that satisfy the required conditions.
\end{exercise}

\begin{proof}
  We first note that for the definition of the monad to make sense, we
  require $\sV$ to have countable coproducts, not just finite ones,
  and so we assume that it does, and that these are also preserved by
  the bifunctor~$\otimes$.

  We also observe that this is an extension of the free monoid monad
  of Example 5.1.4(ii), as $\Set$ can be thought of as a monoidal
  category, with monoidal product the cartesian product and monoidal
  unit a singleton set.

  We present two solutions: the first is a direct construction of the
  required natural transformations that parallel those for the free
  monoid monad, while the second is a longer construction that shows
  how this monad can be derived as the monad of an adjunction in a
  similar way to that of the free monoid monad.

  Both solutions require a little more understanding of the properties
  of monoidal categories than given in the footnote of this exercise;
  Appendix \S E.2 gives more details of the coherence result that we
  will require (Theorem E.2.2), though we will not make use of
  symmetors here.  A full statement and proof appears in \S VII.2
  of~\cite{catworking}.  At the end of \S E.2, Riehl notes that any
  two ``$n$-ary tensor product'' functors $\sV^{\times n}\to \sV$ that
  are built iteratively from the bifunctor $\otimes$ and the unit
  object~$*$, each permuting and parenthesising the inputs in some
  manner, are connected by a unique natural isomorphism that is built
  out of the given natural isomorphisms $\alpha$, $\lambda$ and~$\rho$
  (and their inverses).  Furthermore, any two morphisms built in this
  way between the same two products are equal.  Though we seldom need
  to emphasise these, says Riehl, it is necessary to do so for this
  proof.  The unique natural isomorphism is called the canonical
  isomorphism.  We will write `can' in the following to mean the
  relevant component of the relevant canonical isomorphism.

  In the rest of this solution, we will slightly simplify the notation
  used in the question: for any object $X\in\sV$, we write~$X^n$
  instead of~$X^{\otimes n}$, and we understand this to mean
  $((\cdots(X\otimes X)\otimes\cdots X)\otimes X$ (``all parentheses
  in front''), with $X^0={*}$ and $X^1=X$; we similarly understand
  other expressions such as $a\otimes b\otimes \cdots \otimes c$.  We
  also write $\iota_n$ to indicate the appropriate coproduct
  injection; we will not use different symbols for different
  coproducts.  \renewcommand{\qedsymbol}{}
\end{proof}

\bigskip
\textsc{First Approach}: a direct construction.

We mimic the free monoid monad construction, defining
$\func{\mu_X}{T^2(X)=\coprod_m\bigl(\coprod_n
  X^n\bigr)^m}{T(X)=\coprod_r X^r}$ componentwise by
$$
\begin{tikzcd}[column sep=7em, row sep=large]
  X^{n_1}\otimes \dots \otimes X^{n_m}
  \ar[r,"\iota_m(\iota_{n_1}\otimes \dots \otimes \iota_{n_m})"]
  \ar[d,"\text{can}"']
  & \coprod_m\bigl(\coprod_n X^n\bigr)^m
  \ar[d,dashed,"\mu_X"]
  \\
  X^{n_1+\dots+n_m}
  \ar[r,"\iota_{n_1+\dots+n_m}"']
  & \coprod_r X^r
\end{tikzcd}
$$
where the top morphism is the composite
$$\textstyle X^{n_1}\otimes \dots \otimes X^{n_m}
  \xrightarrow{\iota_{n_1}\otimes \dots \otimes \iota_{n_m}}
  \bigl(\coprod_n X^n\bigr)^m
  \xrightarrow{\iota_m}
  \coprod_m\bigl(\coprod_n X^n\bigr)^m,
$$
and the unit by $\func{\eta_X=\iota_1}{X}{\coprod_n X^n}$.

We now show that these natural transformations satisfy the required
identities.  We have the diagram
\begingroup
\small
$$
\begin{tikzcd}[column sep=small, row sep=large]
  \begin{aligned}
    &(X^{n_{11}}\otimes\dots\otimes X^{n_{1m_1}}) \otimes \cdots\\
    &\quad {}\otimes(X^{n_{k1}}\otimes\dots\otimes X^{n_{km_k}})
  \end{aligned}
  \ar[rrr, "\text{can}\otimes\dots\otimes\text{can}"]
  \ar[ddd, "\text{can}"']
  \ar[dr, "\phi_1"]
  &&&
  \begin{aligned}
    &X^{n_{11}+\dots+n_{1m_1}} \otimes \cdots\\
    &\quad {}\otimes X^{n_{k1}+\dots+n_{km_k}}
  \end{aligned}
  \ar[ddd, "\text{can}"]
  \ar[dl, "\phi_2"']
  \\
  & \coprod_k\bigl(\coprod_m\bigl(\coprod_n X^n\bigr)^m\bigr)^k
  \ar[r, "T\mu_X"]
  \ar[d, "\mu_{TX}"']
  & \coprod_k\bigl(\coprod_r X^r\bigr)^k
  \ar[d, "\mu_X"]
  \\
  & \coprod_s\bigl(\coprod_n X^n\bigr)^s
  \ar[r, "\mu_X"']
  & \coprod_t X^t
  \\
  \begin{aligned}
    &X^{n_{11}}\otimes\dots\otimes X^{n_{1m_1}} \otimes \cdots\\
    &\quad {} \otimes X^{n_{k1}}\otimes\dots\otimes X^{n_{km_k}}
  \end{aligned}
  \ar[rrr, "\text{can}"']
  \ar[ur, "\phi_3"]
  &&&
  X^{n_{11}+\dots+n_{1m_1}+\dots+n_{k1}+\dots+n_{km_k}}
  \ar[ul, "\phi_4"']
\end{tikzcd}
$$
\endgroup
where the $\phi_i$ are given by:
\begin{align*}
  \phi_1 &= \iota_k\bigl(\iota_{m_1}
           (\iota_{n_{11}}\otimes \dots\otimes \iota_{n_{1m_1}})
           \otimes \dots \otimes
           \iota_{m_k}
           (\iota_{n_{k1}}\otimes \dots\otimes \iota_{n_{km_k}})
           \bigr)\\
  \phi_2 &= \iota_k(\iota_{n_{11}+\dots+n_{1m_1}}
           \otimes \dots \otimes
           \iota_{n_{k1}+\dots+n_{km_k}})\\
  \phi_3 &= \iota_{m_1+\dots+m_k}
           (\iota_{n_{11}}\otimes \dots\otimes \iota_{n_{1m_1}}
           \otimes \dots \otimes
           \iota_{n_{k1}}\otimes \dots\otimes \iota_{n_{km_k}}) \\
  \phi_4 &= \iota_{n_{11}+\dots+n_{1m_1}+\dots+n_{k1}+\dots+n_{km_k}}
\end{align*}
The outer square commutes by the coherence theorem, and so the inner
square commutes as required.

For the other two identities involving $\eta$, we have the following
diagrams, using $\eta_X=\iota_1$ and similarly for $\eta_{TX}$:
$$
\begin{tikzcd}[column sep=normal, row sep=large]
  X^n
  \ar[rrr, equal]
  \ar[dr, "\iota_n"]
  \ar[ddd, equal]
  &&& X^n
  \ar[dl, "\iota_1 \iota_n"']
  \ar[dddlll, bend left, "\text{can}=1_{X^n}"]
  \\
  & \coprod_n X^n
  \ar[r, "\eta_{TX}"]
  \ar[d, "1_{\coprod X^n}"']
  & \coprod_m\bigl(\coprod_n X^n\bigr)^m
  \ar[dl, "\mu_X"]
  \\
  & \coprod_n X^n
  \\
  X^n
  \ar[ur, "\iota_n"']
\end{tikzcd}
$$

$$
\begin{tikzcd}[column sep=normal, row sep=large]
  X^n
  \ar[rrr, equal]
  \ar[dr, "\iota_n"]
  \ar[ddd, equal]
  &&& X^n
  \ar[dl, "\iota_n(\iota_1\otimes\dots\otimes\iota_1)"']
  \ar[dddlll, bend left, "\text{can}=1_{X^n}"]
  \\
  & \coprod_n X^n
  \ar[r, "T\eta_X"]
  \ar[d, "1_{\coprod X^n}"']
  & \coprod_n\bigl(\coprod_m X^m\bigr)^n
  \ar[dl, "\mu_X"]
  \\
  & \coprod_n X^n
  \\
  X^n
  \ar[ur, "\iota_n"']
\end{tikzcd}
$$
In each case, the squares commute by the definitions of $\eta$ and
$\mu$; as the outer triangle obviously commutes, so does the induced
inner one.

(Note that we have not demonstrated explicitly that the $\eta$ and
$\mu$ we have defined are natural transformations; it is clear from
their definitions that they are.)

\bigskip
\goodbreak

\textsc{Second Approach}: via an adjunction.

The free monoid monad is induced by the free $\ladj$ forgetful
adjunction between monoids and sets as described in Example 5.1.4(ii).
The unit and multiplication of the monad are then as described in
Lemma 5.1.3.  Therefore, if we can produce a similar adjunction in
this more general setting, we will automatically obtain the required
monad along with its unit and multiplication.

To do this, we will define the category of monoid objects in $\sV$,
denoted $\Monoid_\sV$, and describe a free $\ladj$ forgetful
adjunction $\afunc{F}{G}{\sV}{\Monoid_\sV}$.  The argument here is
that of Theorem VII.3.2 of~\cite{catworking}, with some details filled
in.

An object in $\Monoid_\sV$ (a monoid) is an object $X\in\sV$ together
with a unit $\func{\eta}{*}{X}$ and a multiplication
$\func{\mu}{X\otimes X}{X}$ such that the diagrams
$$
\begin{tikzcd}
  X\otimes(X\otimes X)
  \ar[r, "\alpha"]
  \ar[d, "1_X\otimes\mu"']
  & (X\otimes X)\otimes X
  \ar[r, "\mu\otimes 1_X"]
  & X\otimes X
  \ar[d, "\mu"]
  \\
  X\otimes X
  \ar[rr, "\mu"']
  && X
\end{tikzcd}
$$

$$
\begin{tikzcd}
  {*}\otimes X
  \ar[r, "\eta\otimes 1_X"]
  \ar[dr, "\lambda"']
  & X \otimes X
  \ar[d, "\mu"]
  & X\otimes{*}
  \ar[l, "1_X\otimes\eta"']
  \ar[dl, "\rho"]
  \\
  & X
\end{tikzcd}
$$
are commutative.  A morphism
$\func{f}{(X,\eta_X,\mu_X)}{(Y,\eta_Y,\mu_Y)}$ of monoids is a
morphism $\func{f}{X}{Y}$ that commutes with the units and
multiplications; explicitly, we require these diagrams to commute:
$$
\begin{tikzcd}
  X \otimes X
  \ar[r, "f\otimes f"]
  \ar[d, "\mu_X"']
  & Y \otimes Y
  \ar[d, "\mu_Y"]
  \\
  X
  \ar[r, "f"']
  & Y
\end{tikzcd}
\qquad
\begin{tikzcd}
  *
  \ar[r, "\eta_X"]
  \ar[dr, "\eta_Y"']
  & X
  \ar[d, "f"]
  \\
  & Y
\end{tikzcd}
$$

We will also have need below to refer to iterated multiplication.  For
the monoid $(X,\eta,\mu)$, we define a morphism
$\func{\mu^{(n)}}{X^n}{X}$ iteratively by $\mu^{(0)}=\eta$,
$\mu^{(1)}=1_X$, $\mu^{(2)}=\mu$ and $\mu^{(n+1)}=\mu(\mu^{(n)}\otimes
1_X)$.  Similarly, if $\func{f}{X}{Y}$ is a morphism of objects in~$\sV$,
then $\func{f^n}{X^n}{Y^n}$ is defined componentwise: $f^0=1_*$,
$f^1=f$, $f^{n+1}=f^n\otimes f$.

We are now in a position to state the required theorem (VII.3.2
of~\cite{catworking}):

\begin{mtheorem}
  If the monoidal category $(\sV,\otimes,{*})$ has countable
  coproducts, and if for each $X\in\sV$ the functors $X\otimes{-}$ and
  $\func{{-}\otimes X}{\sV}{\sV}$ preserve these coproducts, then the
  forgetful functor $\func{U}{\Monoid_\sV}{\sV}$ has a left adjoint.
\end{mtheorem}

The action of the left adjoint $\func{F}{\sV}{\Monoid_\sV}$ on objects
is given by $F(X)=\bigl(\coprod_{n\ge0}X^n, \eta_X, \mu_X\bigr)$,
where $\eta_X$ and $\mu_X$ will be described in the proof.  Therefore
the induced monad $T=UF$ has $T(X)=\coprod_{n\ge0}X^n$ as required.
The monad unit is then~$\eta$ and the monad multiplication is
$U\epsilon F$; these are identical to those defined in the first
approach.

\begin{proof}[Proof]
  We first note that for any sequence of objects $a_n\in\sV$ and any
  $b\in\sV$, there is an isomorphism
  $\func{\theta}{b\otimes \coprod_n a_n}{\coprod_n b\otimes a_n}$.
  This is because the functor $b\otimes{-}$ preserves countable
  coproducts, so the image of the colimit cone
  $\func{\iota_n}{a_n}{\coprod_n a_n}$ under this functor is a colimit
  cone $\func{\iota_n}{b\otimes a_n}{b\otimes \coprod_n a_n}$.  Since
  we also have the colimit cone
  $\func{\iota_n}{b\otimes a_n}{\coprod_n b\otimes a_n}$ as $\sV$~has
  countable coproducts, these two colimits are isomorphic, and there
  is a unique isomorphism commuting with the legs of the colimit cones
  by Proposition 3.1.7.  The same applies to the functor
  ${-}\otimes b$, and we again use $\theta$ to indicate the
  corresponding isomorphism.

  Given $X\in\sV$, our first task is to construct a monoid $F(X)$.  We
  take the object to be $\coprod_n X^n$, and we define the unit
  $\func{\eta_X}{*}{\coprod_X X^n}$ to be the injection
  $\func{\iota_0}{{*}=X^0}{\coprod_n X^n}$.

  We next define the multiplication
  $\func{\mu_X}{\coprod_m X^m\otimes \coprod_n X^n}{\coprod_r X^r}$
  componentwise by $\func{\text{can}}{X^m\otimes X^n}{X^{m+n}}$ using
  the canonical morphism.  More formally, we define $\mu_X$ by the
  following commutative diagram:
  $$
  \begin{tikzcd}
    \bigl(\coprod_m X^m\bigr) \otimes \bigl(\coprod_n X^n\bigr)
    \ar[r, "\theta\circ\theta"]
    \ar[dr, dashed, "\mu_X"']
    & \coprod_{m,n} X^m\otimes X^n
    \ar[d, dashed, "\phi"]
    & X^m\otimes X^n
    \ar[l, "\iota_{m,n}"']
    \ar[d, "\text{can}"]
    \\
    & \coprod_r X^r
    & X^{m+n}
    \ar[l, "\iota_{m+n}"]
  \end{tikzcd}
  $$
  where $\phi$ is the map induced by the coproduct injections,
  $\theta\circ\theta$ is the composite of two isomorphisms as
  described above (and implicitly also composed with the isomorphism
  $\coprod_m \coprod_n F(m,n)\cong \coprod_{m,n} F(m,n)$ of Theorem
  3.8.1), and $\mu_X=\phi(\theta\circ\theta)$.

  We must show that these $\eta_X$ and $\mu_X$ satisfy the monoid
  axioms; we do this below.

  We now use Lemma 4.6.1 to show that $F$ is indeed a left adjoint
  to~$U$.  For each $X\in\sV$, we define $\bar\eta_X$ (which will be
  the component at~$X$ of the unit of the adjunction) to be the
  morphism
  $$\func{\bar\eta_X=\iota_1}{X=X^1}{\textstyle \coprod_n X^n=UFX}$$
  and we claim that this is an initial object in the comma category
  $X\downarrow U$.

  Let $(Y,\eta_Y,\mu_Y)$ be a monoid in $\sV$ and
  $\func{f}{X}{Y=U(Y,\eta_Y,\mu_Y)}$ be a morphism in~$\sV$ (which is
  an object in $X\downarrow U$).  We need to show that there is a
  unique monoid morphism
  $\func{\bar f}{\bigl(\coprod_n
    X^n,\eta_X,\mu_X\bigr)}{(Y,\eta_Y,\mu_Y)}$ with
  $\bar f \bar \eta_X = f$.  We construct~$\bar f$ via the components
  $\func{\bar f_n=\bar f \iota_n}{X^n}{Y}$ for each~$n$ as
  follows.\footnote{This differs slightly from the argument
    in~\cite{catworking}, as the uniqueness of~$\bar f$ is not
    explicitly demonstrated there.}

  The second monoid morphism identity requires the diagram
  $$
  \begin{tikzcd}
    X^0={*}
    \ar[r, "\eta_X=\iota_0"]
    \ar[d, "\eta_Y"']
    & \coprod_n X^n
    \ar[dl, "\bar f"]
    \\
    Y
  \end{tikzcd}
  $$
  to commute, so we must have $\bar f_0=\eta_Y$.

  Since $f=\bar f\bar\eta_X=\bar f\iota_1$, we require $\bar f_1=f$.

  As the morphism must satisfy the first monoid morphism identity,
  consider the following diagram:
  $$
  \begin{tikzcd}
    X^m \otimes X^n
    \ar[dr, "\iota_{m,n}"']
    \ar[rr, "\text{can}"]
    \ar[dd, equals]
    && X^{m+n}
    \ar[dd, "\iota_{m+n}"]
    \\
    & \coprod_{m,n} X^m \otimes X^n
    \ar[dr, "\phi"]
    \\
    X^m \otimes X^n
    \ar[r, "\iota_m\otimes\iota_n"']
    & \coprod_m X^m \otimes \coprod_n X^n
    \ar[r, "\mu_X"']
    \ar[u, "\theta\circ\theta"]
    \ar[d, "\bar f\otimes \bar f"']
    & \coprod_r X^r
    \ar[d, "\bar f"]
    \\
    & Y \otimes Y
    \ar[r, "\mu_Y"']
    & Y
  \end{tikzcd}
  $$
  The top part of the diagram commutes by the definition of $\phi$
  and~$\mu_X$, with the left square commuting by Proposition 3.1.7.
  We require the bottom square to commute.  Taking $n=1$, this then
  requires
  $\bar f\circ \iota_{m+1}\circ\text{can} = \mu_Y\circ (\bar f\otimes
  \bar f)\circ(\iota_m\otimes \iota_1)$.  But $X^m\otimes X^1=X^{m+1}$
  by definition, so $\text{can}=1_{X^{m+1}}$ in this case, and the
  equation simplifies to
  $\bar f_{m+1} = \mu_Y(\bar f_m\otimes \bar f_1) = \mu_Y(\bar
  f_m\otimes f)$.  By induction, it then follows that $\bar
  f_m=\mu_Y^{(m)}\circ f^m$, and this holds for $m=0$ as well.

  We can summarise this as follows: $\bar f$ is the composite on the
  bottom of this commutative diagram:
  $$
  \begin{tikzcd}
    X^n
    \ar[r, "f^n"]
    \ar[d, "\iota_n"']
    & Y^n
    \ar[r, "\mu_Y^{(n)}"]
    \ar[d, "\iota_n"]
    & Y
    \ar[d, equals]
    \\
    \coprod_n X^n
    \ar[r, dashed]
    & \coprod_n Y^n
    \ar[r, dashed]
    & Y
  \end{tikzcd}
  $$
  where the arrows on the bottom are defined by the coproduct
  injections to make the two squares commute.

  We can slightly expand the above diagram to show that this
  definition of~$\bar f$ does give a monoid morphism:
  $$
  \begin{tikzcd}
    X^m \otimes X^n
    \ar[dr, "\iota_{m,n}"']
    \ar[rr, "\text{can}"]
    \ar[dd, equals]
    && X^{m+n}
    \ar[dd, "\iota_{m+n}"]
    \ar[dddr, bend left, "f^{m+n}"]
    \\
    & \coprod_{m,n} X^m \otimes X^n
    \ar[dr, "\phi"]
    \\
    X^m \otimes X^n
    \ar[r, "\iota_m\otimes\iota_n"']
    \ar[d, "f^m\otimes f^n"']
    %\ar[dr, phantom, "\tikz{\node[shape=circle,draw,inner sep=1pt]
    % {\scriptstyle 1};}"]
    \ar[dr, phantom, "\text{\textcircled{\scriptsize 1}}"]
    & \coprod_m X^m \otimes \coprod_n X^n
    \ar[r, "\mu_X"']
    \ar[u, "\theta\circ\theta"]
    \ar[d, dashed, shift right=5pt]
    \ar[dd, dashed, shift left=5pt, near start, "\bar f\otimes \bar f"]
    & \coprod_r X^r
    \ar[d, dashed, shift left=5pt]
    \ar[dd, dashed, shift right=5pt, near start, "\bar f"']
    \\
    Y^m \otimes Y^n
    \ar[r, "\iota_m\otimes\iota_n"']
    \ar[d, "\mu_Y^{(m)}\otimes\mu_Y^{(n)}"']
    \ar[dr, phantom, "\text{\textcircled{\scriptsize 2}}"]
    & \coprod_m Y^m \otimes \coprod_n Y^n
    \ar[d, dashed, shift right=5pt]
    \ar[r, phantom, "\text{\textcircled{\scriptsize 3}}"]
    & \coprod_r Y^r
    \ar[d, dashed, shift left=5pt]
    & Y^{m+n}
    \ar[l, "\iota_{m+n}"]
    \ar[d, "\mu_Y^{(m+n)}"]
    \\
    Y \otimes Y
    \ar[r, equals]
    & Y \otimes Y
    \ar[r, "\mu_Y"']
    & Y
    \ar[r, equals]
    & Y
  \end{tikzcd}
  $$
  The top part of the diagram commutes as above.  The dashed morphisms
  in the squares labelled 1 and~2 are induced by the coproduct
  injections, and their composite is $\bar f \otimes \bar f$ by
  construction.  Similarly, the right part of the diagram commutes by
  the construction of~$\bar f$.  It remains to show that square~3
  commutes.

  As the morphisms in square~3 are defined by the coproduct
  injections, it suffices to show that for every $m$ and~$n$, the
  outer part of the diagram commutes, that is, that the following
  diagram commutes:
  $$
  \begin{tikzcd}
    X^m \otimes X^n
    \ar[r, "\text{can}"]
    \ar[d, "f^m\otimes f^n"']
    & X^{m+n}
    \ar[d, "f^{m+n}"]
    \\
    Y^m \otimes Y^n
    \ar[r, dashed, "\text{can}"]
    \ar[d, "\mu_Y^{(m)}\otimes \mu_Y^{(n)}"']
    & Y^{m+n}
    \ar[d, "\mu_Y^{(m+n)}"]
    \\
    Y \otimes Y
    \ar[r, "\mu_Y"]
    & Y
  \end{tikzcd}
  $$
  The top square commutes by naturality of can, and the lower square
  commutes by the general associative law for monoid multiplication,
  which is Proposition VII.3.1 of~\cite{catworking}.  This completes
  the proof that $\bar f$~is a monoid morphism.

  Thus $\bar\eta_X$~is an initial object in $X\downarrow U$, proving
  $U$~has a left adjoint and $F(X)=(\coprod_n X^n, \eta_X, \mu_X)$ is
  a free monoid.

  It only remains to show that $\eta_X$ and $\mu_X$ satisfy the
  required monoid identities, which we now do.\footnote{Mac Lane
    leaves it as an exercise: ``A large but routine diagram
    (exercise!)''.  The following long argument is an expansion of
    that given by Stefan Hamcke on StackExchange:
    \url{https://math.stackexchange.com/questions/1269975}}

  The first identity we need to show is proven using the following
  diagram:
  \begin{center}
    \small
    \begin{tikzpicture}[commutative diagrams/every diagram]
      \node (P0) at (-2.5,5) {$\coprod_k X^k \otimes \bigl(\coprod_m X^m
        \otimes \coprod_n X^n\bigr)$};
      \node (P4) at (2.5,5) {$\bigl(\coprod_k X^k \otimes \coprod_m X^m
        \bigr) \otimes \coprod_n X^n$};
      \node (P1) at (-5.8,1.8) {$\coprod_k X^k \otimes \coprod_t X^t$};
      \node (P3) at (5.8,1.8) {$\coprod_r X^r \otimes \coprod_n X^n$};
      \node (P2) at (0,0) {$\coprod_s X^s$};
      \node (C0) at (-2.5,3) {$\coprod_{k,m,n} X^k\otimes (X^m \otimes
        X^n)$};
      \node (C1) at (2.5,3) {$\coprod_{k,m,n} (X^k\otimes X^m) \otimes
        X^n$};

      \path[commutative diagrams/.cd, every arrow, every label]
        (P0) edge node {$\alpha$} (P4)
        (P0) edge node[swap] {$1_{\coprod X^k}\otimes \mu_X$} (P1)
        (P1) edge node[swap] {$\mu_X$} (P2)
        (P4) edge node {$\mu_X\otimes 1_{\coprod X^n}$} (P3)
        (P3) edge node {$\mu_X$} (P2)
        (P0) edge node[swap] {$\cong$} (C0)
        (P4) edge node {$\cong$} (C1)
        (C0) edge node[swap] {$\coprod \alpha$} (C1)
        (C0) edge (P2)
        (C1) edge (P2);
    \end{tikzpicture}
  \end{center}
  We must show that the outer pentagon commutes: we do this by showing
  that the three inner squares and triangle all commute.

  We begin with the lower triangle.  This expands to the following
  diagram:
  $$
  \begin{tikzcd}[row sep=large]
    \coprod_{k,m,n} X^k\otimes (X^m \otimes X^n)
    \ar[rr, dashed, "\coprod_{k,m,n} \alpha_{X^k,X^m,X^n}"]
    \ar[rddd, dashed, bend right=30]
    && \coprod_{k,m,n} (X^k\otimes X^m) \otimes X^n
    \ar[lddd, dashed, bend left=30]
    \\
    X^k\otimes (X^m \otimes X^n)
    \ar[rr, "\alpha_{X^k,X^m,X^n}"]
    \ar[u, "\iota_{k,m,n}"]
    \ar[dr, "\text{can}"']
    && (X^k\otimes X^m) \otimes X^n
    \ar[u, "\iota_{k,m,n}"]
    \ar[dl, "\text{can}"]
    \\
    & X^{k+m+n}
    \ar[d, "\iota_{k+m+n}"]
    \\
    & \coprod_s X^s
  \end{tikzcd}
  $$
  The dashed arrows indicate maps induced from the coproduct
  injections, and the $\alpha$'s are associators.  The central
  triangle commutes because of the coherence theorem, and so the outer
  (dashed) triangle commutes.

  We next consider the upper square:
  $$
  \begin{tikzcd}[row sep=large, column sep=7em]
    \coprod_k X^k \otimes \bigl(\coprod_m X^m \otimes \coprod_n
    X^n\bigr)
    \ar[r, "\alpha_{\coprod X^k,\coprod X^m,\coprod X^n}"]
    \ar[d, "\theta^4"]
    & \bigl(\coprod_k X^k \otimes \coprod_m X^m \bigr) \otimes
    \coprod_n X^n
    \ar[d, "\theta^4"']
    \\
    \coprod_{k,m,n} X^k\otimes (X^m \otimes X^n)
    \ar[r, "\coprod_{k,m,n} \alpha_{X^k,X^m,X^n}"']
    & \coprod_{k,m,n} (X^k\otimes X^m) \otimes X^n
    \\
    X^k\otimes (X^m \otimes X^n)
    \ar[r, "\alpha_{X^k,X^m,X^n}"']
    \ar[u, "\iota_{k,m,n}"']
    \ar[uu, bend left=30, near end, "\iota_k\otimes(\iota_m\otimes
      \iota_n)"]
    & (X^k\otimes X^m) \otimes X^n
    \ar[u, "\iota_{k,m,n}"]
    \ar[uu, bend right=30, near end, "(\iota_k\otimes\iota_m)\otimes
      \iota_n"']
  \end{tikzcd}
  $$
  Here, $\theta^4$ indicates a composition of four isomorphisms
  between coproducts of the form described at the beginning of the
  proof, such that it commutes with the coproduct injections
  indicated.  The outer square commutes by the naturality of~$\alpha$,
  and so the upper square commutes as required.

  We now consider the right hand square; the left hand square is
  essentially identical.  This can be expanded as follows, with the
  four objects and morphisms in the original diagram coloured blue for
  clarity:
  \begingroup
  \small
  $$
  \begin{tikzcd}
    & \color{blue} \bigl(\coprod_k X^k \otimes \coprod_m X^m\bigr)
    \otimes \coprod_n X^n
    \ar[dd, blue, near end, "(\theta\circ\theta)\otimes 1_{\coprod X^n}"']
    \ar[ddr, blue, very near end, "\mu_X\otimes 1_{\coprod X^n}"]
    \\
    (X^k\otimes X^m) \otimes \coprod_n X^n
    \ar[dr, "\iota_{k,m} \otimes 1_{\coprod X^n}"']
    \ar[rrr, crossing over, pos=0.7, "\text{can} \otimes 1_{\coprod X^n}"]
    \ar[ddd, equals]
    && \null
    &
    X^{k+m} \otimes \coprod_n X^n
    \ar[dl, "\iota_{k+m} \otimes 1_{\coprod X^n}"]
    \ar[ddd, equals]
    \\
    & \color{blue} \coprod_{k,m} (X^k \otimes X^m) \otimes \coprod_n X^n
    \ar[r, "\phi\otimes 1_{\coprod X^n}"']
    \ar[d, blue, "\theta"']
    \ar[ur, phantom, pos=0.3, "\text{\textcircled{\scriptsize 1}}"]
    \ar[dr, phantom, "\text{\textcircled{\scriptsize 2}}"]
    & \color{blue} \coprod_r X^r \otimes \coprod_n X^n
    \ar[d, blue, "\theta"]
    \\
    & \coprod_{k,m} \bigl((X^k \otimes X^m) \otimes \coprod_n X^n\bigr)
    \ar[r]
    \ar[ddd, blue, "\theta"']
    \ar[dr, phantom, pos=0.35, "\text{\textcircled{\scriptsize 3}}"]
    & \coprod_r \bigl(X^r \otimes \coprod_n X^n\bigr)
    \ar[ddd, blue, "\theta"]
    \\
    (X^k\otimes X^m) \otimes \coprod_n X^n    
    \ar[rrr, crossing over, "\text{can} \otimes 1_{\coprod X^n}"']
    \ar[ur, "\iota_{k,m}"]
    && \null
    &
    X^{k+m} \otimes \coprod_n X^n
    \ar[ul, "\iota_{k+m}"']
    \\
    (X^k\otimes X^m) \otimes X^n    
    \ar[rrr, crossing over, "\text{can} \otimes 1_{\coprod X^n}"']
    \ar[u, "1_{X^k\otimes X^m}\otimes \iota_n"]
    \ar[dr, "\iota_{k,m,n}"']
    \ar[ddd, "\text{can}"']
    &&&
    X^{k+m} \otimes X^n
    \ar[u, "1_{X^{k+m}}\otimes \iota_n"']
    \ar[dl, "\iota_{k+m,n}"']
    \ar[dddlll, bend left=20, "\text{can}"]
    \\
    & \coprod_{k,m,n} (X^k\otimes X^m) \otimes X^n
    \ar[r]
    \ar[d, blue]
    \ar[dr, phantom, very near start, "\text{\textcircled{\scriptsize 4}}"]
    & \coprod_{r,n} X^r \otimes X^n
    \ar[dl, blue, "\phi"]
    \\
    & \color{blue} \coprod_s X^s
    & \null
    \\
    X^{k+m+n}
    \ar[ur, "\iota_{k,m,n}"]
  \end{tikzcd}
  $$
  \endgroup

  The morphisms between the coproducts are induced by the coproduct
  injections shown, and the morphisms labelled~$\theta$ are
  isomorphisms of the type described at the beginning of this proof.
  The numbered regions are all bounded on the sides by blue morphisms.
  Triangle~1 commutes by the definition of $\mu_X$.  For square~2, the
  outer square around it trivially commutes, so the inner square does
  too.  Square~3 is similar, with both paths around the ``outer''
  square composing to give $\text{can}\otimes 1_{\coprod X^n}$.  For
  triangle~4, the outer triangle commutes by the coherence theorem,
  and hence the inner one does too.

  Finally, we need to prove the remaining monoid identities involving
  the unit.  Recalling that $\eta_X=\iota_0$, we have
  \begingroup
  \small
  $$
  \begin{tikzcd}[row sep=large, column sep=large]
    {*}\otimes X^n
    \ar[rr, equals]
    \ar[dr, "1_*\otimes \iota_n"]
    \ar[dddrr, bend right=10, "\lambda_{X^n}"']
    && X^0 \otimes X^n
    \ar[rr, equals]
    \ar[d, "\iota_0\otimes \iota_n"]
    && X^0 \otimes X^n
    \ar[dl, "\iota_{0,n}"']
    \ar[dddll, bend left=10, "\text{can}"]
    \\
    & {*}\otimes \coprod_n X^n
    \ar[r, "\eta_X\otimes 1_{\coprod X^n}"]
    \ar[dr, "\lambda_{\coprod X^n}"']
    & \coprod_m X^m \otimes \coprod_n X^n
    \ar[r, "\theta\circ\theta"]
    \ar[d, "\mu_X"]
    & \coprod_{m,n} X^m \otimes X^n
    \ar[dl, "\phi"]
    \\
    && \coprod_r X^r
    \\
    && X^n
    \ar[u, "\iota_n"']
  \end{tikzcd}
  $$
  \endgroup
  The maps between the coproducts are induced as before (though the
  full generality for~$\phi$ is not shown).  The top left square
  commutes as $\eta_X=\iota_0$ (and $X^0={*}$); the top right commutes
  by the choice of $\theta\circ\theta$; the bottom left square
  commutes by naturality of~$\lambda$ and the bottom right commutes by
  the definition of~$\phi$.  The right central triangle commutes by
  the definition of~$\mu_X$ and the outer triangle commutes as
  $\func{\lambda_{X^n}}{X^0\otimes X^n}{X^n}$ is the canonical
  morphism.  A diagram chase then shows that $\mu_X\circ
  (\eta_X\otimes 1_{\coprod X^n}) \circ (1_*\otimes \iota_n)=
  \lambda_{\coprod X^n} \circ (1_*\otimes \iota_n)$, hence the left
  central triangle commutes as required.  The corresponding identity
  involving~$\rho$ is proved identically.
\end{proof}

\end{document}

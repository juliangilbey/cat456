\documentclass[../../solutions]{subfiles}
\title{Template}
\author{}
\begin{document}
\maketitle

% \settheorem{1}{2}{3}
% \begin{lemma}
%
% \end{lemma}
% \popthm

\setexercise{5}{1}{2}
\begin{exercise}
  Show that the functor $\func{\beta}{\Set}{\Set}$ that carries a set
  to the set of ultrafilters on that set is a monad by defining unit
  and multiplication natural transformations that satisfy the unit and
  associativity laws. (Hint: The ultrafilter monad is a submonad of
  the double power set monad $\func{P^2}{\Set}{\Set}$.)
\end{exercise}

\begin{proof}
  To use the hint, we must define \emph{submonad}.  For a
  category~$\sC$, we define the category $\cat{Monad}_\sC$ to have as
  objects monads on~$\sC$.  A morphism \footnote{This definition is
    taken from the nLab, Example 3.3 at
    \url{https://ncatlab.org/nlab/show/monad}, with slightly different
    notation.} from $(S, \eta^S, \mu^S)$ to $(T, \eta^T, \mu^T)$ is a
  natural transformation $\Func{\alpha}{S}{T}$ such that the following
  diagrams commute:
  $$
  \begin{tikzcd}[arrows=Rightarrow]
    1_\sC
    \ar[r, "\eta^S"]
    \ar[dr, "\eta^T"']
    & S
    \ar[d, "\alpha"]
    \\
    & T
  \end{tikzcd}
  \qquad
  \begin{tikzcd}[arrows=Rightarrow]
    SS
    \ar[r, "S\alpha"]
    \ar[d, "\mu^S"']
    & ST
    \ar[r, "\alpha T"]
    & TT
    \ar[d, "\mu^T"]
    \\
    S
    \ar[rr, "\alpha"']
    && T
  \end{tikzcd}
  $$

  A submonad is then a subobject in the category $\cat{Monad}_\sC$,
  that is, a submonad of $(T,\eta^T,\mu^T)$ is a monomorphism
  $\mono{\alpha}{(S,\eta^S,\mu^S)}{(T,\eta^T,\mu^T)}$.  A natural
  tranformation between two functors $\Set\to\Set$ is a monomorphism
  if every component is (see the solution to Exercise~\ref{ex:5.1.i}
  for a discussion of the converse).  Therefore, in
  $\cat{Monad}_\Set$, if every component of~$\alpha$ is a
  monomorphism, $\alpha$~is a monomorphism of monads (though the
  converse might not be the case).

  We now define the unit and multiplication of the monad
  $(\beta,\eta^\beta,\mu_\beta)$, and show that it is a submonad of
  $(P^2,\eta^{P^2},\mu^{P^2})$ by defining a natural monomorphism
  $\Func{\alpha}{\beta}{P^2}$ that satisfies the above conditions.
  The rest of this solution proceeds as follows: we first define the
  action of~$\beta$ on morphisms; we then define~$\alpha$ and show
  that it is a natural monomorphism; next we determine $\eta^\beta$
  and~$\mu^\beta$ from the defining conditions for a submonad and
  prove they are well-defined, and finally we show that they satisfy
  the conditions for the unit and multiplication of a monad.

  \emph{Action of $\beta$ on morphisms.}  We recall the conditions for
  a set $Z\subseteq PX$ to be an ultrafilter on~$X$:
  \begin{enumerate}[label=(\arabic*)]
  \item $Z$ is upward closed;
  \item $Z$ is closed under finite intersections, and
  \item for every subset $Y\subseteq X$, exactly one of $Y\in Z$ and
    $X\setminus Y\in Z$ holds.\footnote{Note that this implies that
      there are no ultrafilters on the empty set.}
  \end{enumerate}
  Note that if $Z$ is an ultrafilter on~$X$, then $\emptyset\notin Z$:
  if it were, then by~(1), we must have $Z=PX$, and this
  contradicts~(3), which requires that exactly one of $\emptyset\in Z$
  and $X\in Z$ holds.

  In the rest of this proof, we consistently use subscripts so that
  $A_i\in P^iA$, $B_i\in P^iB$ and so on; in particular, we will have
  $A_2\in\beta A$ and $A_4\in\beta(\beta A)$.

  For any non-empty subset $Y\subseteq PX$, we write $\bar Y$ to mean
  the smallest upward closed subset of~$PX$ containing~$Y$;
  explicitly,
  $$\bar Y=\{X'\subseteq X: Y'\subseteq X'\ \text{for some $Y'\in
    Y$}\}.$$
  For $\func{f}{A}{B}$, we define, for any $A_2\in\beta A$,
  $$\beta f(A_2)=\overline{\{f(A_1)\colon A_1\in A_2\}}$$
  and claim that this definition results in $\beta$~being
  functorial.\footnote{Note in particular that for a principal
    ultrafilter $\eta_A(a)$, its image under $\beta f$ is
    $\eta_B(f(a))$, which suggests that this is likely to be the
    `correct' definition.  We do not attempt to show that this
    definition is the same as that derived from the definition of the
    functor $\func{\beta}{\Top}{\cat{cHaus}}$ in Example 4.6.12.}  We
  show first that $\beta f(A_2)\in \beta B$ by checking the defining
  conditions:
  \begin{enumerate}[label=(\arabic*)]
  \item $\beta f(A_2)$ is upward closed by construction.
  \item Suppose $B_1^{\vphantom1}, B_1'\in \beta f(A_2)$.  Then there
    are sets $A_1^{\vphantom1},A_1'\in A_2$ with $f(A_1)\subseteq B_1$
    and $f(A_1')\subseteq B_1'$.  Since $A_2$~is closed under finite
    intersections, $f(A_1^{\vphantom1}\cap A_1')\in \beta f(A_2)$.  As
    $f(A_1^{\vphantom1}\cap A_1')\subseteq f(A_1^{\vphantom1})\cap
    f(A_1')$, it follows that
    $f(A_1^{\vphantom1}\cap A_1')\subseteq B_1^{\vphantom1}\cap B_1'$,
    so $B_1^{\vphantom1}\cap B_1'\in \beta f(A_2)$.
  \item Suppose $B_1\subseteq B$ and $B_1\notin \beta f(A_2)$; we must
    then show that $B\setminus B_1\in \beta f(A_2)$.  Let
    $B_1'=B_1^{\vphantom1}\cap f(A)$.  We must have
    $B_1'\notin\beta f(A_2)$ as $\beta f(A_2)$ is upward closed.  Let
    $A_1=f^{-1}(B_1')$, and note that $f(A_1)=B_1'$.  If $A_1\in A_2$,
    then $B_1'\in \beta f(A_2)$, which is not the case, so
    $A\setminus A_1\in A_2$, and thus
    $f(A)\setminus B_1'=f(A)\setminus f(A_1)=f(A\setminus A_1)\in
    \beta f(A_2)$.  (The second equality holds as if $x\in A\setminus
    A_1$, then $f(x)\notin B_1'=f(A_1)$.)  Finally, as $(B\setminus
    B_1)\cap f(A)= f(A)\setminus B_1'$, it follows that $B\setminus
    B_1\in \beta f(A_2)$.

    If we had both $B_1\in \beta f(A_2)$ and $B\setminus B_1\in\beta
    f(A_2)$, then by~(2), we would have $\emptyset\in\beta f(A_2)$.
    But as $\emptyset\notin A_2$, $\emptyset\notin \beta f(A_2)$.
  \end{enumerate}

  For functoriality, for any $A_2\in\beta A$, we have
  $$\beta 1_A(A_2)=\overline{\{A_1:A_1\in A_2\}}=A_2=1_{\beta
    A}(A_2)$$
  as $A_2$ is upward closed.  And if $\func{f}{A}{B}$ and
  $\func{g}{B}{C}$, then for $A_2\in\beta A$,
  \begin{align*}
    \beta(gf)(A_2)
      &= \overline{\{gf(A_1):A_1\in A_2\}} \\
      &= \{C_1:gf(A_1)\subseteq C_1\ \text{for some $A_1\in A_2$}\} \\
    \beta g(\beta f(A_2))
      &= \overline{\{g(B_1):B_1\in \beta f(A_2)\}} \\
      &= \{C_1:g(B_1)\subseteq C_1\ \text{for some $B_1\in \beta
        f(A_2)$}\} \\
      &= \{C_1:g(B_1)\subseteq C_1\ \text{and}\ f(A_1)\subseteq B_1\
        \text{for some $A_1\in A_2$, $B_1\subseteq B$}\}
  \end{align*}
  and these are clearly equal.  Therefore our definition of $\beta f$
  is valid.

  \emph{Definition of the natural monomorphism
    $\Func{\alpha}{\beta}{P^2}$.}  As described in Example 5.1.4(vii),
  $\eta^{P^2}_A=\eta_A$ is the principal ultrafilter function on~$A$,
  and $\mu^{P^2}_A=\eta_{PA}^{-1}$ is the inverse principal
  ultrafilter function.\footnote{To briefly expand on this, the power
    set adjunction, which we can denote as $P\ladj P^\op$, has
    unit~$\eta$ and, by symmetry, counit $\epsilon=\eta^\op$.  Then
    the induced double power set monad has multiplication
    $\mu=P^\op\epsilon P=P^\op\eta^\op P$, so the component at~$A$ is
    $\mu_A=P^\op\eta^\op_{PA}$.  As $P$~sends a function to its
    inverse image function, we have $\mu_A=\eta^{-1}_{PA}$.}  Since
  $\beta(A)\subset P^2A$ for all~$A$, we take $\alpha=\iota$ to be
  inclusion, every component of which is clearly a monomorphism.  To
  demonstrate naturality, we require the diagram
  $$
  \begin{tikzcd}
    \beta A
    \ar[r, "\iota_A"]
    \ar[d, "\beta f"']
    & P^2 A
    \ar[d, "P^2 f"]
    \\
    \beta B
    \ar[r, "\iota_B"']
    & P^2 f
  \end{tikzcd}
  $$
  to commute for $\func{f}{A}{B}$.  Recall that $\func{Pf}{PB}{PA}$ is
  the inverse image function, $f^{-1}$, so $\func{P^2f}{P^2A}{P^2B}$
  is given by $P^2f(A_2)=(f^{-1})^{-1}(A_2)$; more explicitly, if
  $B_1\subseteq B$, then $B_1\in P^2f(A_2)$ if and only if
  $f^{-1}(B_1)\in A_2$.  Now for $A_2\in \beta A$ and $B_1\subseteq
  B$, we have
  \begin{gather*}
    B_1\in\iota_B\cdot \beta f(A_2) = \overline{\{f(A_1):A_1\in A_2\}}
    \text{ if and only if } f(A_1)\subseteq B_1 \text{ for some $A_1\in
      A_2$}\\
    B_1\in P^2 f\cdot \iota_A(A_2) \text{ if and only if }
    f^{-1}(B_1)\in A_2
  \end{gather*}
  Therefore if $B_1\in\iota_B\cdot \beta f(A_2)$, then
  $f(A_1)\subseteq B_1$ for some~$A_1$, so $A_1\subseteq f^{-1}(B_1)$
  and hence $f^{-1}(B_1)\in A_2$ as $A_2$~is upward closed.
  Conversely, if $B_1\in P^2f\cdot\iota_A(A_2)$, then
  $A_1:=f^{-1}(B_1)\in A_2$, and $f(A_1)=f(f^{-1}(B_1))\subseteq
  B_1$.  This shows that $\iota$~is natural.

  \emph{Definition of $\eta^\beta$ and $\mu^\beta$.}  Returning to our
  desired morphism of monads, we thus require the following diagrams
  to commute for every $A\in\Set$:
  $$
  \begin{tikzcd}
    A
    \ar[r, "\eta^\beta_A"]
    \ar[dr, "\eta^{P^2}_A"']
    & \beta(A)
    \ar[d, "\iota_A"]
    \\
    & P^2(A)
  \end{tikzcd}
  \qquad
  \begin{tikzcd}
    \beta(\beta(A))
    \ar[r, "\beta\iota_A"]
    \ar[d, "\mu^\beta_A"']
    & \beta(P^2(A))
    \ar[r, "\iota_{P^2(A)}"]
    & P^2(P^2(A))
    \ar[d, "\mu^{P^2}_A"]
    \\
    \beta(A)
    \ar[rr, "\iota_A"']
    && P^2(A)
  \end{tikzcd}
  $$

  From the left hand diagram, it is clear that we must take
  $\eta^\beta_A=\eta_A$, the principal ultrafilter function, and this
  is clearly well-defined as $\eta_A(a)\in \beta(A)$ for every $a\in
  A$.  From the right-hand diagram, we must have $\iota_A\cdot
  \mu^\beta_A= \mu^{P^2}_A\cdot \iota_{P^2{A}}\cdot \beta\iota_A$; it
  thus suffices to show that $\mu^{P^2}_A\cdot \iota_{P^2{A}}\cdot
  \beta\iota_A(A_4)$ is an ultrafilter on~$A$ for every
  $A_4\in\beta(\beta(A))$.  If we do not write the inclusion morphisms
  explicitly, we have $\mu^\beta_A= \mu^{P^2}_A\cdot \beta\iota_A$ and
  must show that $\mu^{P^2}_A\cdot \beta\iota_A(A_4)$ is an
  ultrafilter on~$A$.

  We have $\beta\iota_A(A_4)=\overline{\{A_3:A_3\in A_4\}}$, where the
  closure is in $P^2(P^2A)$, and so
  $$\mu^\beta(A_4) = \mu^{P^2}_A\cdot \beta\iota_A(A_4)=
  \eta^{-1}_{PA}(\overline{\{A_3:A_3\in A_4\}}) =
  \eta^{-1}_{PA}(\bar A_4) \in P^2A.$$
  We can now check the conditions for an ultrafilter.  We first note
  that
  \begin{alignat*}{2}
    &&A_1&\in \eta^{-1}_{PA}(\bar A_4)\\
    &\iff \quad& \eta_{PA}(A_1) &\in \bar A_4 \\
    &\iff& A_3 &\subseteq \eta_{PA}(A_1) \quad\text{for some $A_3\in
      A_4$} \\
    &\iff& A_1 &\in A_2\quad\text{for all $A_2\in A_3$ for some
      $A_3\in A_4$}
  \end{alignat*}

  \begin{enumerate}[label=(\arabic*)]
  \item Suppose that $A_1\in\eta^{-1}_{PA}(\bar A_4)$ and
    $A_1\subseteq B_1$.  Then with the $A_3$ as above, $B_1\in A_2$
    for all $A_2\in A_3$ as $A_2\in\beta A$ is upward closed (and
    $A_2\in\beta A$ because $A_4\in \beta(\beta A)$).  Therefore
    $A_3\subseteq \eta_{PA}(B_1)$ and so $\eta_{PA}(B_1)\in\bar A_4$,
    showing that $\eta^{-1}_{PA}(\bar A_4)$ is upward closed.

  \item Suppose $A_1,B_1\in\eta^{-1}_{PA}(\bar A_4)$.  Then as above,
    there are $A_3,B_3\in A_4$ with $A_1\in A_2$ for all $A_2\in A_3$
    and $B_1\in B_2$ for all $B_2\in B_3$.  But as
    $A_4\in\beta(\beta A)$, it is closed under finite intersections,
    so $A_3\cap B_3\in A_4$.  Now for all $A_2\in A_3\cap B_3$, we
    have $A_1,B_1\in A_2$, and as $A_2\in\beta A$, it follows that
    $A_1\cap B_1\in A_2$, and hence
    $A_1\cap B_1\in \eta^{-1}_{PA}(\bar A_4)$.

  \item Let $A_1\in PA$.  Then $\beta A\cap \eta_{PA}(A_1)$ is the set
    of all ultrafilters on~$A$ containing~$A_1$, while
    $\beta A\cap \eta_{PA}(A\setminus A_1)$ is the set of all
    ultrafilters on~$A$ containing~$A\setminus A_1$.  But every
    ultrafilter on~$A$ contains exactly one of $A_1$ and
    $A\setminus A_1$, so $\beta A$~is the disjoint union of
    $\beta A\cap \eta_{PA}(A_1)$ and
    $\beta A\cap \eta_{PA}(A\setminus A_1)$.  Therefore every
    ultrafilter in $\beta(\beta A)$ contains exactly one of
    $\beta A\cap \eta_{PA}(A_1)$ and
    $\beta A\cap \eta_{PA}(A\setminus A_1)$, so either
    $A_1\in \eta^{-1}_{PA}(\bar A_4)$ or $A\setminus A_1\in
    \eta^{-1}_{PA}(\bar A_4)$ (using the earlier characterisation of
    $\eta^{-1}_{PA}(\bar A_4)$.

    Now suppose that both
    $A_1,A\setminus A_1\in \eta^{-1}_{PA}(\bar A_4)$.  Then we have
    some $A_3\in A_4$ with $A_1\in A_2$ for all $A_2\in A_3$ and some
    $B_3\in A_4$ with $A\setminus A_1\in B_2$ for all $B_2\in B_3$.
    As $A_4$~is an ultrafilter, $A_3\cap B_3\in A_4$ and
    $A_1, A\setminus A_1\in A_2$ for all $A_2\in A_3\cap B_3$.  If
    $A_3\cap B_3\ne\emptyset$, this is impossible, as $A_2$~is an
    ultrafilter.  But if $A_3\cap B_3=\emptyset$, then
    $B_3\subseteq \beta A\setminus A_3$, so
    $\beta A\setminus A_3\in A_4$ as $A_4$~is upward closed, but then
    $A_3, \beta A\setminus A_3\in A_4$, which is impossible as
    $A_4$~is an ultrafilter on~$\beta A$.  Thus we cannot have both
    $A_1,A\setminus A_1\in \eta^{-1}_{PA}(\bar A_4)$.
  \end{enumerate}

  Thus $\mu^\beta(A_4)=\mu^{P^2}_A\cdot \beta\iota_A(A_4)$ is
  well-defined.

  We next show that the components of $\eta^\beta$ and $\mu^\beta$ as
  defined assemble to give natural transformations.  Let
  $\func{f}{A}{B}$ be a morphism in $\Set$.  Then in the diagram
  $$
  \begin{tikzcd}
    A
    \ar[r, "\eta^\beta_A"]
    \ar[d, "f"']
    & \beta A
    \ar[r, "\iota_A"]
    \ar[d, "\beta f"]
    & P^2A
    \ar[d, "P^2f"]
    \\
    B
    \ar[r, "\eta^\beta_B"']
    &
    \beta B
    \ar[r, "\iota_B"']
    & P^2B
  \end{tikzcd}
  $$
  the outer rectangle commutes by the naturality of
  $\eta^{P^2}=\iota\cdot\eta^\beta$ and the right square commutes by
  the naturality of~$\iota$.  Thus
  $\iota_B\cdot \beta f\cdot \eta^\beta_A = P^2f\cdot \iota_A\cdot
  \eta^\beta_A = \iota_B\cdot \eta^\beta_B\cdot f$, so
  $\beta f\cdot \eta^\beta_A = \eta^\beta_B\cdot f$ as $\iota_B$~is a
  monomorphism.  The naturality of~$\mu^\beta$ follows in a similar
  fashion.

  \emph{Showing that $\eta^\beta$ and $\mu^\beta$ satisfy the unit and
    multiplication laws.}  For each $A\in\Set$, we require these
  diagrams to commute:
  $$
  \begin{tikzcd}
    \beta(\beta A)
    \ar[dr, "\mu^\beta_A"']
    & \beta A
    \ar[l, "\eta^\beta_{\beta A}"']
    \ar[r, "\beta\eta^\beta_A"]
    \ar[d, "1_{\beta A}"]
    & \beta(\beta A)
    \ar[dl, "\mu^\beta_A"]
    \\
    & \beta A
  \end{tikzcd}
  \qquad
  \begin{tikzcd}
    \beta(\beta(\beta A))
    \ar[r, "\beta\mu^\beta_A"]
    \ar[d, "\mu^\beta_{\beta A}"']
    & \beta(\beta A)
    \ar[d, "\mu^\beta_A"]
    \\
    \beta(\beta A)
    \ar[r, "\mu^\beta_A"']
    & \beta A
  \end{tikzcd}
  $$

  For the first identity, we have
  \begin{align*}
    &\begin{tikzcd}[ampersand replacement=\&]
      \beta A
      \ar[r, "\beta\eta^\beta_A"]
      \& \beta(\beta A)
      \ar[r, "\mu^\beta_A"]
      \& \beta A
      \ar[r, "\iota_A"]
      \& P^2A
    \end{tikzcd} \\
    =&\begin{tikzcd}[ampersand replacement=\&]
      \beta A
      \ar[r, "\beta\eta^\beta_A"]
      \& \beta(\beta A)
      \ar[r, "\beta\iota_A"]
      \& \beta(P^2A)
      \ar[r, "\iota_{P^2A}"]
      \& P^2(P^2A)
      \ar[r, "\mu^{P^2}_A"]
      \& P^2A
    \end{tikzcd} \\
    &\qquad\text{by the definition of $\mu^\beta$}\\
    =&\begin{tikzcd}[ampersand replacement=\&]
      \beta A
      \ar[r, "\beta\eta^{P^2}_A"]
      \& \beta(P^2A)
      \ar[r, "\iota_{P^2A}"]
      \& P^2(P^2A)
      \ar[r, "\mu^{P^2}_A"]
      \& P^2A
    \end{tikzcd} \\
    &\qquad\text{by the definition of $\eta^\beta$}\\
    =&\begin{tikzcd}[ampersand replacement=\&]
      \beta A
      \ar[r, "\iota_A"]
      \& P^2A
      \ar[r, "P^2\eta^{P^2}_A"]
      \& P^2(P^2A)
      \ar[r, "\mu^{P^2}_A"]
      \& P^2A
    \end{tikzcd} \\
    &\qquad\text{by naturality of $\iota$}\\
    =&\begin{tikzcd}[ampersand replacement=\&]
      \beta A
      \ar[r, "\iota_A"]
      \& P^2A
    \end{tikzcd} \\
    &\qquad\text{by the monad identities for $(P^2,\eta^{P^2},\mu^{P^2})$}
  \end{align*}
  so $\mu^\beta_A\cdot \beta\eta^\beta_A=1_{\beta A}$ as $\iota_A$ is
  a monomorphism.  Similarly,
  \begin{align*}
    &\begin{tikzcd}[ampersand replacement=\&]
      \beta A
      \ar[r, "\eta^\beta_{\beta A}"]
      \& \beta(\beta A)
      \ar[r, "\mu^\beta_A"]
      \& \beta A
      \ar[r, "\iota_A"]
      \& P^2A
    \end{tikzcd} \\
    =&\begin{tikzcd}[ampersand replacement=\&]
      \beta A
      \ar[r, "\eta^\beta_{\beta A}"]
      \& \beta(\beta A)
      \ar[r, "\beta\iota_A"]
      \& \beta(P^2A)
      \ar[r, "\iota_{P^2A}"]
      \& P^2(P^2A)
      \ar[r, "\mu^{P^2}_A"]
      \& P^2A
    \end{tikzcd} \\
    =&\begin{tikzcd}[ampersand replacement=\&]
      \beta A
      \ar[r, "\eta^\beta_{\beta A}"]
      \& \beta(\beta A)
      \ar[r, "\iota_{\beta A}"]
      \& P^2(\beta A)
      \ar[r, "P^2\iota_A"]
      \& P^2(P^2A)
      \ar[r, "\mu^{P^2}_A"]
      \& P^2A
    \end{tikzcd} \\
    &\qquad\text{by naturality of $\iota$}\\
    =&\begin{tikzcd}[ampersand replacement=\&]
      \beta A
      \ar[r, "\eta^{P^2}_{\beta A}"]
      \& P^2(\beta A)
      \ar[r, "P^2\iota_A"]
      \& P^2(P^2A)
      \ar[r, "\mu^{P^2}_A"]
      \& P^2A
    \end{tikzcd} \\
    &\qquad\text{by the definition of $\eta^\beta$}\\
    =&\begin{tikzcd}[ampersand replacement=\&]
      \beta A
      \ar[r, "\iota_A"]
      \& P^2A
      \ar[r, "\eta^{P^2}_{P^2A}"]
      \& P^2(P^2A)
      \ar[r, "\mu^{P^2}_A"]
      \& P^2A
    \end{tikzcd} \\
    &\qquad\text{by naturality of $\eta^{P^2}$}\\
    =&\begin{tikzcd}[ampersand replacement=\&]
      \beta A
      \ar[r, "\iota_A"]
      \& P^2A
    \end{tikzcd}
  \end{align*}
  so $\mu^\beta_A\cdot \eta^\beta_{\beta A}=1_{\beta A}$.

  Finally, for associativity of multiplication, expanding the
  definition of $\mu^\beta$ twice gives:
  \begin{align*}
    &\begin{tikzcd}[ampersand replacement=\&]
      \beta^3 A
      \ar[r, "\beta\mu^\beta_A"]
      \& \beta^2 A
      \ar[r, "\mu^\beta_A"]
      \& \beta A
      \ar[r, "\iota_A"]
      \& P^2A
    \end{tikzcd} \\
    =&\begin{tikzcd}[ampersand replacement=\&]
      \beta^3 A
      \ar[r, "\beta\mu^\beta_A"]
      \& \beta^2 A
      \ar[r, "\beta\iota_A"]
      \& \beta(P^2A)
      \ar[r, "\iota_{P^2A}"]
      \& P^2(P^2A)
      \ar[r, "\mu^{P^2}_A"]
      \& P^2A
    \end{tikzcd} \\
    =&\begin{tikzcd}[ampersand replacement=\&]
      \beta^3 A
      \ar[r, "\beta^2\iota_A"]
      \& \beta^2(P^2A)
      \ar[r, "\beta\iota_{P^2A}"]
      \& \beta(P^2(P^2A))
      \ar[r, "\beta\mu^{P^2}_A"]
      \& \beta(P^2A)
      \ar[r, "\iota_{P^2A}"]
      \& P^2(P^2A)
      \ar[r, "\mu^{P^2}_A"]
      \& P^2A
    \end{tikzcd}
  \end{align*}
  and likewise
  \begin{align*}
    &\begin{tikzcd}[ampersand replacement=\&]
      \beta^3 A
      \ar[r, "\mu^\beta_{\beta A}"]
      \& \beta^2 A
      \ar[r, "\mu^\beta_A"]
      \& \beta A
      \ar[r, "\iota_A"]
      \& P^2A
    \end{tikzcd} \\
    =&\begin{tikzcd}[ampersand replacement=\&]
      \beta^3 A
      \ar[r, "\mu^\beta_{\beta A}"]
      \& \beta^2 A
      \ar[r, "\beta\iota_A"]
      \& \beta(P^2A)
      \ar[r, "\iota_{P^2A}"]
      \& P^2(P^2A)
      \ar[r, "\mu^{P^2}_A"]
      \& P^2A
    \end{tikzcd} \\
    =&\begin{tikzcd}[ampersand replacement=\&]
      \beta^3 A
      \ar[r, "\beta^2\iota_A"]
      \& \beta^2(P^2A)
      \ar[r, "\mu^\beta_{P^2A}"]
      \& \beta(P^2A)
      \ar[r, "\iota_{P^2A}"]
      \& P^2(P^2A)
      \ar[r, "\mu^{P^2}_A"]
      \& P^2A
    \end{tikzcd} \\
    &\qquad\text{by naturality of $\mu^\beta$}\\
    =&\begin{tikzcd}[ampersand replacement=\&]
      \beta^3 A
      \ar[r, "\beta^2\iota_A"]
      \& \beta^2(P^2A)
      \ar[r, "\beta\iota_{P^2A}"]
      \& \beta(P^2(P^2A))
      \ar[r, "\iota_{P^2(P^2A)}"]
      \& P^2(P^2(P^2A))
      \ar[r, "\mu^{P^2}_{P^2A}"]
      \& P^2(P^2A)
      \ar[r, "\mu^{P^2}_A"]
      \& P^2A
    \end{tikzcd} \\
    =&\begin{tikzcd}[ampersand replacement=\&]
      \beta^3 A
      \ar[r, "\beta^2\iota_A"]
      \& \beta^2(P^2A)
      \ar[r, "\beta\iota_{P^2A}"]
      \& \beta(P^2(P^2A))
      \ar[r, "\iota_{P^2(P^2A)}"]
      \& P^2(P^2(P^2A))
      \ar[r, "P^2\mu^{P^2}_A"]
      \& P^2(P^2A)
      \ar[r, "\mu^{P^2}_A"]
      \& P^2A
    \end{tikzcd} \\
    &\qquad\text{by the monad identities for $\mu^{P^2}$}\\
    =&\begin{tikzcd}[ampersand replacement=\&]
      \beta^3 A
      \ar[r, "\beta^2\iota_A"]
      \& \beta^2(P^2A)
      \ar[r, "\beta\iota_{P^2A}"]
      \& \beta(P^2(P^2A))
      \ar[r, "\beta\mu^{P^2}_A"]
      \& \beta(P^2A)
      \ar[r, "\iota_{P^2A}"]
      \& P^2(P^2A)
      \ar[r, "\mu^{P^2}_A"]
      \& P^2A
    \end{tikzcd} \\
    &\qquad\text{by naturality of $\iota$}
  \end{align*}
  and so the two compositions are equal, proving the associativity
  of~$\mu^\beta$.

  In summary, $\eta^\beta_A=\eta_A$, the principal ultrafilter
  function, and $\mu^\beta_A=\mu^{P^2}_A\cdot \beta\iota_A$.  These
  definitions make $(\beta,\eta^\beta,\mu^\beta)$ a submonad of
  $(P^2,\eta^{P^2},\mu^{P^2})$.
\end{proof}

\end{document}

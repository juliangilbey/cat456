\documentclass[../../solutions]{subfiles}
\title{Template}
\author{}
\begin{document}
\maketitle

\settheorem{5}{2}{12}
\begin{proposition}
  \addcontentsline{toc}{subsection}{Proposition 5.2.12}%
  \label{prop:5.2.12}%
  The Kleisli category $\sC_T$ is initial in $\cat{Adj}_T$ and the
  Eilenberg--Moore category $\sC^T$ is terminal.  That is, for any
  adjunction $F\ladj U$ inducing the monad $(T,\eta,\mu)$ on~$\sC$,
  there exist unique functors
  $$
  \begin{tikzcd}[row sep=huge, column sep=large]
    \sC_T \ar[r, dashed, "J", "\exists!"']
    \ar[dr, shift left=1ex, pos=0.4, "U_T" {inner sep=0mm}]
    \ar[dr, phantom, "\scriptstyle\top" {anchor=center, rotate=-45,
      inner sep=.5mm}]
    \ar[dr, shift right=1ex, leftarrow, pos=0.4, "F_T"' {inner sep=0mm}]
    & \sD
    \ar[r, dashed, "K", "\exists!"']
    \ar[d, shift left=1ex, pos=0.5, "U"]
    \ar[d, phantom, "\scriptstyle\dashv"]
    \ar[d, shift right=1ex, leftarrow, pos=0.5, "F"']
    & \sC^T
    \ar[dl, shift left=1ex, pos=0.4, "U^T" {inner sep=0mm}]
    \ar[dl, phantom, "\scriptstyle\bot" {anchor=center, rotate=45,
      inner sep=.5mm}]
    \ar[dl, shift right=1ex, leftarrow, pos=0.4, "F^T"' {inner sep=0mm}]
    \\
    & \sC
  \end{tikzcd}
  $$
  commuting with the left and right adjoints.
\end{proposition}
\popthm

I found the justification of the uniqueness of $K$ in the book a
little unclear; the following is a rewrite of the second part of the
proof to clarify this.

\begin{proof}[Second half of proof]
  To define the functor $K$ on an object $d\in\sD$ so that $U^TK=U$,
  we seek a suitable $T$-object structure for the object $Ud\in\sC$;
  we denote this by $Kd=(Ud, \func{\gamma_d}{TUd}{Ud})$.  On
  morphisms, the condition $U^TK=U$ forces us to define the image
  under~$K$ of $\func{f}{d}{d'}$ to be
  $\func{Uf}{(Ud,\gamma_d)}{(Ud',\gamma_{d'})}$; that this is a
  morphism of algebras will follow once we have determined~$\gamma_d$
  (it is also given explicitly in the solution to
  \hyperref[ex:5.2.vi]{Exercise 5.2.vi}).  Functoriality of~$K$ is
  obvious.  For any algebra $(c,\func{\gamma}{Tc}{c})\in\sC^T$, the
  algebra structure map~$\gamma$ appears as $\epsilon^T_{(c,\gamma)}$,
  the component of the counit of $F^T\ladj U^T$ at the object
  $(c,\gamma)$; see the proof of Lemma 5.2.8.  Thus, $\gamma$~can be
  recognized as the morphism
  $\func{\gamma=\epsilon^T_{(c,\gamma)}}{(Tc,\mu_c)}{(c,\gamma)}$ that
  is the transpose of the identity on $c=U^T(c,\gamma)$.  The fact
  that $K$~preserves transposes now tells us that
  $K\epsilon_d=\epsilon^T_{(Ud,\gamma_d)}=\gamma_d$; since
  $K\epsilon_d=U\epsilon_d$, we must take $\gamma_d=U\epsilon_d$.
  Thus $Kd=(Ud,U\epsilon_d)$.  Finally, the proof that $U\epsilon_d$
  is an algebra structure map appears in Exercise 5.2.vi.
\end{proof}

\end{document}


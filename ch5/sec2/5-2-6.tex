\documentclass[../../solutions]{subfiles}
\title{Template}
\author{}
\begin{document}
\maketitle

% \settheorem{1}{2}{3}
% \begin{lemma}
%
% \end{lemma}
% \popthm

\setexercise{5}{2}{6}
\begin{exercise}
  Fill in the remaining details in the proof of Proposition 5.2.12:
  verify the functoriality of $J$ and~$K$ and show that the whiskered
  composites of the counits of these three adjunctions with $K$
  and~$J$ agree.
\end{exercise}

\begin{proof}
  For $K$, we need to show that $U\epsilon_d$ is an algebra structure
  map rather than that it is functorial (which has already been
  shown); we also show that $Kf$~is a morphism of algebras.  Note the
  \hyperref[prop:5.2.12]{rewrite at the start of this section} of the
  second half of the proof of this proposition.

  We write $\tilde f\colon c\rightsquigarrow c'$ for the morphism in
  $\sC_T$ represented by the morphism $\func{f}{c}{Tc'}$ in~$\sC$
  (these are transposes of each other).  Now $\tilde 1_c\colon
  c\rightsquigarrow c$ is represented by~$\eta_c$, so
  $$J\tilde 1_c =
  \begin{tikzcd}[ampersand replacement=\&]
    Fc
    \ar[r, "F\eta_c"]
    \& FUFc
    \ar[r, "\epsilon_{Fc}"]
    \& Fc
  \end{tikzcd}
  $$
  but $\epsilon F\cdot F\eta = 1_F$, so $J\tilde 1_c=1_{Fc}=1_{Jc}$.

  Now suppose $\tilde f\colon c\rightsquigarrow c'$ and $\tilde
  g\colon c'\rightsquigarrow c''$ are represented by
  $\func{f}{c}{Tc'}$ and $\func{g}{c'}{Tc''}$.  Then
  \begin{align*}
    J\tilde g\cdot J\tilde f
    &=\begin{tikzcd}[ampersand replacement=\&]
      Fc
      \ar[r, "Ff"]
      \& FUFc'
      \ar[r, "\epsilon_{Fc'}"]
      \& Fc'
      \ar[r, "Fg"]
      \& FUFc''
      \ar[r, "\epsilon_{Fc''}"]
      \& Fc''
    \end{tikzcd} \\
    &=\begin{tikzcd}[ampersand replacement=\&]
      Fc
      \ar[r, "Ff"]
      \& FUFc'
      \ar[r, "FUFg"]
      \& FUFUFc''
      \ar[r, "\epsilon_{FUFc''}"]
      \& FUFc''
      \ar[r, "\epsilon_{Fc''}"]
      \& Fc''
    \end{tikzcd} \\
    &\qquad\qquad\text{by naturality of $\epsilon$}\\
    &=\begin{tikzcd}[ampersand replacement=\&]
      Fc
      \ar[r, "Ff"]
      \& FUFc'
      \ar[r, "FUFg"]
      \& FUFUFc''
      \ar[r, "FU\epsilon_{Fc''}"]
      \& FUFc''
      \ar[r, "\epsilon_{Fc''}"]
      \& Fc''
    \end{tikzcd} \\
    &\qquad\qquad\text{by naturality of $\epsilon$ again}\\
    &=\begin{tikzcd}[ampersand replacement=\&]
      Fc
      \ar[r, "Ff"]
      \& FUFc'
      \ar[r, "FUFg"]
      \& FUFUFc''
      \ar[r, "F\mu_{c''}"]
      \& FUFc''
      \ar[r, "\epsilon_{Fc''}"]
      \& Fc''
    \end{tikzcd} \\
    &\qquad\qquad\text{as $\mu=U\epsilon F$}\\
    &=J(\tilde g \tilde f)
  \end{align*}
  showing that $J$ is functorial.

  We next show that $U\epsilon_d$ is an algebra structure map, showing
  that $Kd=(Ud,U\epsilon_d)$ is well-defined.  We must show that
  $$
  \begin{tikzcd}
    T^2Ud
    \ar[r, "\mu_{Ud}"]
    \ar[d, "TU\epsilon_d"']
    & TUd
    \ar[d, "U\epsilon_d"]
    \\
    TUd
    \ar[r, "U\epsilon_d"']
    & Ud
  \end{tikzcd}
  $$
  commutes.  But using $\mu=U\epsilon F$ and $T=UF$, this is
  $$
  \begin{tikzcd}
    UFUFUd
    \ar[r, "U\epsilon_{FUd}"]
    \ar[d, "UFU\epsilon_d"']
    & UFUd
    \ar[d, "U\epsilon_d"]
    \\
    UFUd
    \ar[r, "U\epsilon_d"']
    & Ud
  \end{tikzcd}
  $$
  which commutes by the naturality of $\epsilon$, and so $U\epsilon_d$
  is an algebra structure map.

  To show that for $\func{f}{d}{d'}$ in $\sD$,
  $\func{Uf}{(Ud,U\epsilon_d)}{(Ud',U\epsilon_{d'})}$ is a morphism of
  algebras, we note that the diagram
  $$
  \begin{tikzcd}[column sep=6em]
    UFUd=TUd
    \ar[r, "TUf=UFUf"]
    \ar[d, "U\epsilon_d"']
    & TUd' = UFUd'
    \ar[d, "U\epsilon_{d'}"]
    \\
    Ud
    \ar[r, "Uf"']
    & Ud'
  \end{tikzcd}
  $$
  commutes by the naturality of $\epsilon$.

  Next, to show that the whiskered composites agree, it suffices to
  show that $K$ and~$J$ commute with the left and right adjoints, as
  noted in the text after the statement of the proposition; we now do
  this.

  \begin{itemize}
  \item $JF_T=F$.  For $c\in\sC$, $Fc=JF_Tc$ by the definition
    of~$J$.  For $\func{f}{c}{c'}$ in~$\sC$,
    \begin{align*}
      JF_Tf
      &= \begin{tikzcd}[ampersand replacement=\&]
        Fc
        \ar[r, "Ff"]
        \& Fc'
        \ar[r, "F\eta_{c'}"]
        \& FTc' = FUFc'
        \ar[r, "\epsilon_{Fc'}"]
        \& Fc'
      \end{tikzcd} \\
      &=\begin{tikzcd}[ampersand replacement=\&]
        Fc
        \ar[r, "Ff"]
        \& Fc'
      \end{tikzcd} \\
      &\qquad\qquad\text{by the triangle identities}\\
      &=Ff
    \end{align*}
    so $JF_T=F$.

  \item $UJ=U_T$.  For $c\in\sC_T$, $UJc=UFc=Tc=U_Tc$.  For $\tilde
    f\colon c\rightsquigarrow c'$ in~$\sC_T$ represented by
    $\func{f}{c}{Tc'}$ in~$\sC$,
    \begin{align*}
      UJ\tilde f
      &= \begin{tikzcd}[ampersand replacement=\&]
        UFc
        \ar[r, "UFf"]
        \& UFUFc'
        \ar[r, "U\epsilon_{Fc'}"]
        \& UFc'
      \end{tikzcd} \\
      &=\begin{tikzcd}[ampersand replacement=\&]
        Tc
        \ar[r, "Tf"]
        \& T^2c'
        \ar[r, "\mu_{c'}"]
        \& Tc'
      \end{tikzcd} \\
      &\qquad\qquad\text{using $\mu=U\epsilon F$}\\
      &=U_Tf
    \end{align*}
    so $UJ=U_T$.

  \item $KF=F^T$.  For $c\in\sC$,
    $KFc=(UFc,U\epsilon_{Fc})=(Tc,\mu_c)=F^Tc$, using $\mu=U\epsilon
    F$.  For $\func{f}{c}{c'}$ in~$\sC$, $KFf=UFf=Tf=F^Tf$, so
    $KF=F^T$.

  \item $U^TK=U$.  For $d\in\sD$, $U^TKd=U^T(Ud,U\epsilon_d)=Ud$, and
    for $\func{f}{d}{d'}$ in~$\sD$, $U^TKf=U^TUf=Uf$, so $U^TK=U$.
  \end{itemize}
\end{proof}

\end{document}

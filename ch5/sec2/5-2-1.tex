\documentclass[../../solutions]{subfiles}
\title{Template}
\author{}
\begin{document}
\maketitle

% \settheorem{1}{2}{3}
% \begin{lemma}
%   
% \end{lemma}
% \popthm

\setexercise{5}{2}{1}
\begin{exercise}
  Pick your favorite monad on $\Set$ induced from a free $\dashv$
  forgetful adjunction involving some variety of algebraic structure
  borne by sets, identify its algebras, and show that algebra
  morphisms are precisely the homomorphisms, in the appropriate sense.
\end{exercise}

\begin{proof}
  Consider the free group monad $F$ of example 5.1.4(iv), induced by
  the free $\dashv$ forgetful adjunction $\begin{tikzcd}
    \Set \ar[r, shift left=1ex] \ar[r, leftarrow, shift right=1ex]
    \ar[r, phantom, "\scriptstyle\bot"] & \Group
  \end{tikzcd}$.  If $X\in\Set$, then $F(X)$ is the set of words
  in~$X$ and the formal inverse $a^{-1}$ for each $a\in X$, quotiented
  by the relations $aa^{-1}=a^{-1}a=1$ (the empty word) for each $a\in
  X$.

  An algebra for $F$ is a pair $(A\in\Set, \func{a}{FA}{A})$ such that
  the diagrams
  $$
  \begin{tikzcd}
    A
    \ar[r, "\eta_A"]
    \ar[dr, "1_A"']
    & FA
    \ar[d, "a"]
    \\
    & A
  \end{tikzcd}
  \qquad
  \begin{tikzcd}
    F^2A
    \ar[r, "\mu_A"]
    \ar[d, "Fa"']
    & FA
    \ar[d, "a"]
    \\
    FA
    \ar[r, "a"']
    & A
  \end{tikzcd}
  $$
  both commute.

  Every group $(G,*,e)$ gives rise to an algebra for $F$: let $X=UG$,
  the underlying set of~$G$, then $(X, \func{\text{ev}}{FX}{X})$ is an
  algebra, where $\text{ev}$ evaluates a word in the group~$G$
  (explicitly, $\text{ev}(x_1\ldots x_n)=x_1*\cdots*x_n$, where some
  of the $x_i$ might be of the form $x_i^{-1}$).  The left diagram
  commutes as $\text{ev}(x)=x$ for every $x\in X$, and the square
  commutes by the general associative law for~$*$.

  Conversely, any algebra $(A,a)$ for the free group monad defines a
  group $(A,*,e)$, where $e=a(1)$, $x*y=a(xy)$ and the inverse of~$x$
  is $a(x^{-1})$, since
  $$x*a(x^{-1})=a(xa(x^{-1}))=a(a(x)a(x^{-1}))=a(xx^{-1})=a(1)=e$$
  and likewise for $a(x^{-1})*x$.  The associativity condition implied
  by the square shows that $*$~is associative as required.

  Now consider an algebra morphism $\func{f}{(A,a)}{(B,b)}$, with the
  corresponding groups being $(A,*_A,e_A)$ and $(B,*_B,e_B)$:
  $$
  \begin{tikzcd}
    TA
    \ar[r, "Tf"]
    \ar[d, "a"']
    & TB
    \ar[d, "b"]
    \\
    A
    \ar[r, "f"']
    & B
  \end{tikzcd}
  $$
  Then $b\cdot Tf(1) = b(1) = e_B$, while $f\cdot a(1)=f(e_A)$, so
  $f(e_A)=e_B$.  If $x,y\in A$, then
  $$f(x*_A y)=f(a(xy))=b(Tf(ab))=b(f(x)f(y))=f(x)*_B f(y)$$
  so an algebra morphism is a group homomorphism.
\end{proof}

\end{document}


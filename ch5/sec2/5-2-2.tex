\documentclass[../../solutions]{subfiles}
\title{Template}
\author{}
\begin{document}
\maketitle

% \settheorem{1}{2}{3}
% \begin{lemma}
%   
% \end{lemma}
% \popthm

\setexercise{5}{2}{2}
\begin{exercise}
  Fill in the remaining details in the proof of Lemma 5.2.8 to show
  that the free and forgetful functors relating a category~$\sC$ with a
  monad~$T$ to the category $\sC^T$ of $T$-algebras are adjoints,
  inducing the given monad~$T$.
\end{exercise}

\begin{proof}
  To show the first triangle law holds,
  $$
  \begin{tikzcd}
    F^T
    \ar[r, Rightarrow, "F^T\eta"]
    \ar[dr, Rightarrow, "1_{F^T}"']
    & F^TU^TF^T
    \ar[d, Rightarrow, "\epsilon F^T"]
    \\
    & F^T
  \end{tikzcd}
  $$
  we show that it holds at an arbitrary $A\in\sC$, so we must show
  that
  $$
  \begin{tikzcd}
    F^TA
    \ar[r, "F^T\eta_A"]
    \ar[dr, "1_{F^TA}"']
    & F^TU^TF^TA
    \ar[d, "\epsilon_{F^TA}"]
    \\
    & F^TA
  \end{tikzcd}
  $$
  commutes.  Now $\eta_A$ is the component of the unit of the monad,
  $\epsilon_{F^TA}=\epsilon_{(TA,\mu_A)}=\mu_A$, and $U^TF^T=T$, so
  the triangle becomes
  $$
  \begin{tikzcd}
    F^TA
    \ar[r, "F^T\eta_A"]
    \ar[dr, "1_{F^TA}"']
    & F^TTA
    \ar[d, "\mu_A"]
    \\
    & F^TA
  \end{tikzcd}
  $$
  which expands to
  $$
  \begin{tikzcd}
    (TA,\mu_A)
    \ar[r, "T\eta_A"]
    \ar[dr, "1_{TA}"']
    & (T^2A, \mu_{TA})
    \ar[d, "\mu_A"]
    \\
    & (TA,\mu_A)
  \end{tikzcd}
  $$
  This commutes by the monad laws.

  The other identity reads at $(A,a)\in\sC^T$:
  $$
  \begin{tikzcd}
    U^T(A,a)
    \ar[r, "\eta_{U^T(A,a)}"]
    \ar[dr, "1_{U^T(A,a)}"']
    & U^TF^TU^T(A,a)
    \ar[d, "U^T\epsilon_{(A,a)}"]
    \\
    & U^T(A,a)
  \end{tikzcd}
  $$
  which simplifies to
  $$
  \begin{tikzcd}
    A
    \ar[r, "\eta_A"]
    \ar[dr, "1_A"']
    & TA
    \ar[d, "a"]
    \\
    & A
  \end{tikzcd}
  $$
  which commutes as $(A,a)$ is a $T$-algebra.
\end{proof}

\end{document}


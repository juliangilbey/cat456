\documentclass[../../solutions]{subfiles}
\title{Template}
\author{}
\begin{document}
\maketitle

\addcontentsline{toc}{subsection}{Definition 5.4.8 and Corollary 5.4.10(i)}%
\label{def:5.4.8}
Definition 5.4.8 is somewhat unclear (for example, in the definition
of creates coequalizers, is it the fork that is isomorphic, or just
the coequalizer?).  We present Beck's original definitions~\cite{beck}
and then a proof of Corollary 5.4.10(i) using this.  (The proof in the
book is unclear on this same point.)  At the start of the next
section, we will present a slightly amended proof of Theorem 5.5.1,
also taking this into account.  Note, incidentally, that the reference
given at the start of Section 5.5 to this theorem is incorrect; it
does not appear in Beck's Ph.D. thesis, but was, according to
Mac~Lane~\cite[159]{catworking}, presented at a conference in 1966.
Furthermore, as Corollary 5.4.10(ii) is used as a critical step in
Theorem 5.5.1, we have to rewrite it to remove the assumption that
$U$~is monadic; we call the new version Lemma 5.4.11.

\settheorem{5}{4}{8}
\begin{definition}
  Given a functor $\func{U}{\sD}{\sC}$:
  \begin{itemize}
  \item A \textbf{$U$-split coequalizer} is a parallel pair
    $\pfunc{f,g}{x}{y}$ in~$\sD$ together with an extension of the
    pair $\pfunc{Uf,Ug}{Ux}{Uy}$ to a split coequalizer diagram
    $$
    \begin{tikzcd}
      Ux
      \ar[r, shift left=0.5ex, "Uf"]
      \ar[r, shift right=0.5ex, "Ug"']
      & Uy \ar[r, "h"]
      \ar[l, bend left=45, pos=0.45, "t"]
      & z
      \ar[l, bend left, pos=0.55, "s"]
    \end{tikzcd}
    $$
    in~$\sC$.
  \item \textbf{$\sD$ has $U$-split coequalizer pairs} if every
    $U$-split coequalizer $\pfunc{f,g}{x}{y}$ in~$\sD$ has a
    coequalizer in~$\sD$.
  \item \textbf{$U$ preserves $U$-split pairs} if whenever
    $\pfunc{f,g}{x}{y}$ is $U$-split, and $x\rightrightarrows y\to w$
    is a coequaliser diagram, then its image under~$U$ is a
    coequaliser diagram.  Equivalently, if the split coequaliser
    diagram has underlying fork $Ux\rightrightarrows Uy\to z$, then
    the unique commuting map $z\to Uw$ in~$\sC$ is an isomorphism.
  \item \textbf{$U$ reflects $U$-split pairs} if whenever a fork
    $x\rightrightarrows y\to z$ in~$\sD$ maps to a fork in~$\sC$ that
    can be extended to a split coequalizer diagram, then
    $x\rightrightarrows y\to z$ is a coequalizer diagram in~$\sD$.
  \item \textbf{$U$ creates coequalizers of $U$-split pairs} if
    $\sD$~has $U$-split coequalizer pairs, and $U$ preserves and
    reflects $U$-split pairs.
  \item \textbf{$U$ strictly creates coequalizers of $U$-split pairs}
    if whenever $\pfunc{f,g}{x}{y}$ is $U$-split, with a split
    coequaliser diagram with underlying fork
    $Ux\rightrightarrows Uy\to z$, then this lifts to a unique fork
    $x\rightrightarrows y\to w$ in~$\sD$, and moreover, this unique
    fork is a coequalizer diagram in~$\sD$.
  \end{itemize}
\end{definition}

It is straightforward to see that if $U$ strictly creates coequalizers
of $U$-split pairs, then it creates coequalizers of $U$-split pairs.

With this clarified definition, the proof of Proposition 5.4.9 follows
unchanged, but the proof of Corollary 5.4.10(i) requires some changes.
We present this, together with the easy extension to the strict case.

\settheorem{5}{4}{10}
\begin{corollary}
  \label{corr:5.4.10}%
  If $\begin{tikzcd}\sC
    \ar[r, shift left=0.7ex, "F"]
    \ar[r, shift right=0.7ex, leftarrow, "U"']
    \ar[r, phantom, "\scriptstyle\bot"]
    & \sD
  \end{tikzcd}$ is a monadic adjunction, then
  \begin{enumerate}[label=(\roman*)]
  \item $\func{U}{\sD}{\sC}$ creates coequalizers of $U$-split pairs.
    If the adjunction is strictly monadic, then $U$ strictly creates
    coequalizers of $U$-split pairs.
  \end{enumerate}
\end{corollary}
\popthm

\begin{proof}[Proof]
  To say that $\func{U}{\sD}{\sC}$ is monadic is to say that there is
  an equivalence of categories
  $$
  \begin{tikzcd}
    \sD
    \ar[rr, "K", "\simeq"']
    \ar[dr, "U"']
    && \sC^T
    \ar[dl, "U^T"]
    \\
    & \sC
  \end{tikzcd}
  $$
  where $T$ is the monad $UF$.  In the strict case, $K$~is an
  isomorphism of categories.

  If $\pfunc{f,g}{A}{B}$ is a $U$-split pair in $\sD$, then
  commutativity $U=U^TK$ implies that $\pfunc{Kf,Kg}{KA}{KB}$ is a
  $U^T$-split pair in~$\sC^T$.  Proposition 5.4.9 lifts the fork of
  the $U$-split coequalizer diagram in~$\sC$, say
  $UA\rightrightarrows UB\to C$, to a unique coequalizer diagram
  $$
  \begin{tikzcd}
    KA
    \ar[r, shift left=0.5ex, "Kf"]
    \ar[r, shift right=0.5ex, "Kg"']
    & KB
    \ar[r, "h"]
    & (C,\gamma)
  \end{tikzcd}
  $$
  in~$\sC^T$.  As an equivalence, $K$~creates colimits, so $f$,~$g$
  have a coequalizer in~$\sD$, showing that $\sD$~has $U$-split
  coequalizer pairs.  Since $K$~preserves colimits, if the pair
  $f$,~$g$ has a coequalizer $y\to z_0$ in~$\sD$, then the image of
  this coequalizer under~$K$ is a coequalizer of $Kf$,~$Kg$ in~$\sD$,
  and so the canonical map $(C,\gamma)\to Kz_0$ in~$\sC^T$ is an
  isomorphism (Proposition 3.1.7), and hence the canonical map
  $C\to Uz_0$ is too.  Finally, as $K$~reflects colimits, if the fork
  $x \rightrightarrows y\to z$ in~$\sD$ maps to a fork in~$\sC$ that
  can be extended to a split coequalizer diagram, then its image
  under~$K$ in~$\sC^T$ is a coequalizer diagram, and so the fork
  in~$\sD$ is a coequalizer diagram.  Thus $U$~creates coequalizers of
  $U$-split pairs.

  In the case that the adjunction is strictly monadic, $K$~is an
  isomorphism.  Thus if $\pfunc{f,g}{A}{B}$ is a $U$-split pair in
  $\sD$, the uniqueness of the lift of the underlying fork of the
  split coequalizer diagram to~$\sC^T$ and it being a coequalizer
  diagram carries over directly to~$\sD$ via~$K$, so $U$~strictly
  creates coequalizers of $U$-split pairs.
\end{proof}

As noted above, we need to replace Corollary 5.4.10(ii) by the
following slight variant:

\settheorem{5}{4}{11}
\begin{lemma}
  \label{lemma:5.4.11}%
  Suppose that
  $\begin{tikzcd}\sC
    \ar[r, shift left=0.7ex, "F"]
    \ar[r, shift right=0.7ex, leftarrow, "U"']
    \ar[r, phantom, "\scriptstyle\bot"]
    & \sD
  \end{tikzcd}$ is an adjunction such that $U$~reflects $U$-split
  coequalisers.  Then for any $D\in\sD$, there is a coequaliser
  diagram
  $$
  \begin{tikzcd}
    FUFUD
    \ar[r, shift left=0.5ex, "FU\epsilon_D"]
    \ar[r, shift right=0.5ex, "\epsilon_{FUD}"']
    & FUD
    \ar[r, "\epsilon_D"]
    & D
  \end{tikzcd}
  $$
  involving the counit $\Func{\epsilon}{FU}{1_\sD}$ of the
  adjunction.
\end{lemma}
\popthm

Note that by Corollary 5.4.10(i), the hypotheses of this lemma hold if
$F\ladj U$ is a monadic adjunction, giving the original Corollary
5.4.10(ii).  The proof is exactly as given in the book, except that
now reflecting is an assumption:

\begin{proof}[Proof]
  This is a fork by the naturality of $\epsilon$.  It is $U$-split by
  the diagram
  $$
  \begin{tikzcd}
    UFUFUD
    \ar[r, shift left=0.5ex, "UFU\epsilon_D"]
    \ar[r, shift right=0.5ex, "U\epsilon_{UFD}"']
    & UFUD \ar[r, "U\epsilon_D"]
    \ar[bend left=45]{l}{\eta_{UFUD}}
    & UD
    \ar[bend left]{l}{\eta_{UD}}
  \end{tikzcd}
  $$
  in $\sC$.  Since $U$ reflects $U$-split coequalisers, the fork
  defines a coequaliser diagram in~$\sD$.
\end{proof}

\end{document}

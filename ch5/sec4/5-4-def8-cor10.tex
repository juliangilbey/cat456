\documentclass[../../solutions]{subfiles}
\title{Template}
\author{}
\begin{document}
\maketitle

\addcontentsline{toc}{subsection}{Definition 5.4.8 and Corollary 5.4.10(i)}%
\label{def:5.4.8}
Definition 5.4.8 is somewhat unclear (for example, in the definition
of creates coequalizers, is it the fork that is isomorphic, or just
the coequalizer?).  We present Beck's original
definitions~\cite{beck} and then a proof of Corollary 5.4.10(i) using
this.  (The proof in the book is unclear on this same point.)  At the
start of the next section, we will present a slightly amended proof of
Theorem 5.5.1, also taking this into account.  Note, incidentally,
that the reference given at the start of Section 5.5 to this theorem
is incorrect; it does not appear in Beck's Ph.D. thesis, but was,
according to Mac~Lane~\cite[159]{catworking}, presented at a
conference in 1966.

\settheorem{5}{4}{8}
\begin{definition}
  Given a functor $\func{U}{\sD}{\sC}$:
  \begin{itemize}
  \item A \textbf{$U$-split coequalizer} is a parallel pair
    $\pfunc{f,g}{x}{y}$ in~$\sD$ together with an extension of the
    pair $\pfunc{Uf,Ug}{Ux}{Uy}$ to a split coequalizer diagram
    $$
    \begin{tikzcd}
      Ux
      \ar[r, shift left=0.5ex, "Uf"]
      \ar[r, shift right=0.5ex, "Ug"']
      & Uy \ar[r, "h"]
      \ar[l, bend left=45, pos=0.45, "t"]
      & z
      \ar[l, bend left, pos=0.55, "s"]
    \end{tikzcd}
    $$
    in~$\sC$.
  \item \textbf{$\sD$ has $U$-split coequalizer pairs} if every
    $U$-split coequalizer $\pfunc{f,g}{x}{y}$ in~$\sD$ has a
    coequalizer in~$\sD$.
  \item \textbf{$U$ preserves $U$-split pairs} if whenever
    $\pfunc{f,g}{x}{y}$ is $U$-split, with the underlying fork of the
    split coequalizer diagram being $Ux\rightrightarrows Uy\to z$, and
    the pair $f$,~$g$ has a coequalizer $y\to z_0$ in~$\sD$, then the
    canonical map $z\to Uz_0$ in~$\sC$ is an isomorphism.
  \item \textbf{$U$ reflects $U$-split pairs} if whenever a fork
    $x\rightrightarrows y\to z$ in~$\sD$ maps to a fork in~$\sC$ that
    can be extended to a split coequalizer diagram, then
    $x\rightrightarrows y\to z$ is a coequalizer diagram in~$\sD$.
  \item \textbf{$U$ creates coequalizers of $U$-split pairs} if
    $\sD$~has $U$-split coequalizer pairs, and $U$ preserves and
    reflects $U$-split pairs.
  \item \textbf{$U$ strictly creates coequalizers of $U$-split pairs} if
    whenever $\pfunc{f,g}{x}{y}$ is $U$-split, this extends to a unique
    fork $x\rightrightarrows y\to z$ in~$\sD$ whose image is the
    underlying fork of the split coequalizer diagram in~$\sC$, and
    moreover, this unique fork is a coequalizer diagram in~$\sD$.
  \end{itemize}
\end{definition}

It is straightforward to see that if $U$ strictly creates coequalizers
of $U$-split pairs, then it creates coequalizers of $U$-split pairs.

With this clarified definition, the proof of Proposition 5.4.9 follows
unchanged, but the proof of Corollary 5.4.10(i) requires some changes.
We present this, together with the easy extension to the strict case.

\settheorem{5}{4}{10}
\begin{corollary}
  \label{corr:5.4.10}%
  If $\begin{tikzcd}\sC
    \ar[r, shift left=0.7ex, "F"]
    \ar[r, shift right=0.7ex, leftarrow, "G"']
    \ar[r, phantom, "\scriptstyle\bot"]
    & \sD
  \end{tikzcd}$ is a monadic adjunction, then
  \begin{enumerate}[label=(\roman*)]
  \item $\func{U}{\sD}{\sC}$ creates coequalizers of $U$-split pairs.
    If the adjunction is strictly monadic, then $U$ strictly creates
    coequalizers of $U$-split pairs.
  \end{enumerate}
\end{corollary}
\popthm

\begin{proof}
  To say that $\func{U}{\sD}{\sC}$ is monadic is to say that there is
  an equivalence of categories
  $$
  \begin{tikzcd}
    \sD
    \ar[rr, "K", "\simeq"']
    \ar[dr, "U"]
    && \sC^T
    \ar[dl, "U^T"]
    \\
    & \sC
  \end{tikzcd}
  $$
  where $T$ is the monad $UF$.  In the strict case, $K$~is an
  isomorphism of categories.

  If $\pfunc{f,g}{A}{B}$ is a $U$-split pair in $\sD$, then
  commutativity $U=U^TK$ implies that $\pfunc{Kf,Kg}{KA}{KB}$ is a
  $U^T$-split pair in~$\sC^T$.  Proposition 5.4.9 lifts the fork of
  the $U$-split coequalizer diagram in~$\sC$, say
  $UA\rightrightarrows UB\to C$, to a unique coequalizer diagram
  $$
  \begin{tikzcd}
    KA
    \ar[r, shift left=0.5ex, "Kf"]
    \ar[r, shift right=0.5ex, "Kg"']
    & KB
    \ar[r, "h"]
    & (C,\gamma)
  \end{tikzcd}
  $$
  in~$\sC^T$.  As an equivalence, $K$~creates colimits, so $f$,~$g$
  have a coequalizer in~$\sD$, showing that $\sD$~has $U$-split
  coequalizer pairs.  Since $K$~preserves colimits, if the pair
  $f$,~$g$ has a coequalizer $y\to z_0$ in~$\sD$, then the image of
  this coequalizer under~$K$ is a coequalizer of $Kf$,~$Kg$ in~$\sD$,
  and so the canonical map $(C,\gamma)\to Kz_0$ in~$\sC^T$ is an
  isomorphism (Proposition 3.1.7), and hence the canonical map
  $C\to Uz_0$ is too.  Finally, as $K$~reflects colimits, if the fork
  $x \rightrightarrows y\to z$ in~$\sD$ maps to a fork in~$\sC$ that
  can be extended to a split coequalizer diagram, then its image
  under~$K$ in~$\sC^T$ is a coequalizer diagram, and so the fork
  in~$\sD$ is a coequalizer diagram.  Thus $U$~creates coequalizers of
  $U$-split pairs.

  In the case that the adjunction is strictly monadic, $K$~is an
  isomorphism.  Thus if $\pfunc{f,g}{A}{B}$ is a $U$-split pair in
  $\sD$, the uniqueness of the lift of the underlying fork of the
  split coequalizer diagram to~$\sC^T$ and it being a coequalizer
  diagram carries directly to~$\sD$ via~$K$, so $U$~strictly creates
  coequalizers of $U$-split pairs.
\end{proof}

\end{document}


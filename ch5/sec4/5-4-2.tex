\documentclass[../../solutions]{subfiles}
\title{Template}
\author{}
\begin{document}
\maketitle

% \settheorem{1}{2}{3}
% \begin{lemma}
%   
% \end{lemma}
% \popthm

\setexercise{5}{4}{2}
\begin{exercise}
  Prove Proposition 5.4.3 directly by a diagram chase.
\end{exercise}

\begin{proof}
  Let $(T,\eta,\mu)$ be a monad on $\sC$ and let
  $(A,\func{\alpha}{TA}{A})$ be a $T$-algebra.  We must show that
  $$
  \begin{tikzcd}
    (T^2A,\mu_{TA})
    \ar[r, shift left=0.5ex, "T\alpha"]
    \ar[r, shift right=0.5ex, "\mu_A"']
    & (TA,\mu_A)
    \ar[r, "\alpha"]
    & (A,\alpha)
  \end{tikzcd}
  $$
  is a coequaliser diagram in $\sC^T$.  Note that it is certainly a
  fork because of the algebra identities for $(A,\alpha)$.

  Suppose $(B,\beta)$ is a $T$-algebra with
  $\func{f}{(TA,\mu_A)}{(B,\beta)}$ giving the fork
  \begin{equation}
    \label{eq:5-4-2-1}
    \begin{tikzcd}
      (T^2A,\mu_{TA})
      \ar[r, shift left=0.5ex, "T\alpha"]
      \ar[r, shift right=0.5ex, "\mu_A"']
      & (TA,\mu_A)
      \ar[r, "\alpha"]
      \ar[rd, "f"']
      & (A,\alpha)
      \ar[d, dashed, "\exists!"', "h"]
      \\
      && (B,\beta)
    \end{tikzcd}
  \end{equation}
  We need to show that there is a unique algebra morphism $h$, as
  shown, making the diagram commute.

  We first claim existence by showing that $h=f\cdot \eta_A$ is an
  algebra morphism that makes the diagram commute.  We have
  \begin{alignat*}{2}
    f\cdot \eta_A\cdot \alpha
    &= f\cdot T\alpha \cdot \eta_{TA} &\qquad&\text{by naturality of
      $\eta$}\\
    &= f\cdot \mu_A \cdot \eta_{TA} &&\text{by assumed fork}\\
    &= f &\qquad&\text{by the monad identities}
  \end{alignat*}
  so $h\cdot\alpha=f$ as morphisms in $\sC$.

  To show $h$ is an algebra morphism, consider the following diagram;
  our task is to show that the outer rectangle commutes.
  $$
  \begin{tikzcd}[row sep=large, column sep=large]
    TA
    \ar[r, "T\eta_A"]
    \ar[d, "\alpha"']
    & T^2A
    \ar[r, "Tf"]
    \ar[d, "T\alpha"]
    & TB
    \ar[d, "\beta"]
    \\
    A
    \ar[r, "\eta_A"']
    & TA
    \ar[r, "f"']
    & B
  \end{tikzcd}
  $$
  We have
  \begin{alignat*}{2}
    \beta\cdot Tf\cdot T\eta_A
    &= f\cdot T\alpha\cdot T\eta_A &\qquad&\text{$f$ is an algebra morphism}\\
    &= f &&\text{by the monad identities} \\
    &= f\cdot \eta_A\cdot\alpha &&\text{by the above}
  \end{alignat*}
  and so $h$ is an algebra morphism.

  Finally, suppose that $\func{h'}{(A,\alpha)}{(B,\beta)}$ is an
  algebra morphism also satisfying $h'\cdot \alpha=f$, so
  $h'\cdot\alpha=h\cdot\alpha$.  But as $\alpha\cdot\eta_A=1_A$,
  $\alpha$~is a split epimorphism, hence $h'=h$.  Thus the original
  fork is a coequaliser diagram in~$\sC^T$.
\end{proof}

\end{document}


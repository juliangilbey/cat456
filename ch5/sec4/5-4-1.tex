\documentclass[../../solutions]{subfiles}
\title{Template}
\author{}
\begin{document}
\maketitle

% \settheorem{1}{2}{3}
% \begin{lemma}
%   
% \end{lemma}
% \popthm

\setexercise{5}{4}{1}
\begin{exercise}
  The coequalizer of a parallel pair of morphisms $f$ and $g$ in the
  category $\Ab$ is equally the cokernel of the map $f-g$.  Explain
  how the canonical presentation of an abelian group described in
  Proposition 5.4.3 defines a presentation of that group, in the usual
  sense.
\end{exercise}

\begin{proof}
  The presentation of an abelian group in the usual sense is
  $A=\langle G\mid R\rangle$, where $G$~is a set of generators and
  $R$~is a set of relations, which are elements of $\ZZ[G]$.  In this
  case, for Proposition 5.4.3, $T=\ZZ[-]$, the abelian group monad on
  $\Set$, and
  $$
  \begin{tikzcd}
    \ZZ[\ZZ[A]]
    \ar[r, shift left=0.5ex, "{\ZZ[\alpha]}"]
    \ar[r, shift right=0.5ex, "\mu_A"']
    & \ZZ[A]
    \ar[r, twoheadrightarrow, "\alpha"]
    & A
  \end{tikzcd}
  $$
  is a coequaliser, where $\alpha$ is the canonical evaluation
  homomorphism.  But as noted in the question, the coequaliser of
  $\ZZ[\alpha]$ and~$\mu_A$ is the cokernel of the map
  $\ZZ[\alpha]-\mu_A$, so $A\cong \ZZ[A]/\im(\ZZ[\alpha]-\mu_A)$ (as
  abelian groups).

  Let us explore the map
  $\func{\ZZ[\alpha]-\mu_A}{\ZZ[\ZZ[A]]}{\ZZ[A]}$.  A finite sum
  $\sum_i\lambda_i \sum_j \mu_{ij}a_{ij}\in \ZZ[\ZZ[A]]$, where
  $\lambda_i, \mu_{ij}\in\ZZ$, $a_{ij}\in A$, maps under $\ZZ[\alpha]$
  to $\sum_i\lambda_i\alpha\bigl(\sum_j\mu_{ij}a_{ij}\bigr)$, where
  $\alpha\bigl(\sum_j\mu_{ij}a_{ij}\bigr)$ is a single element of~$A$,
  and it maps under~$\mu_A$ to
  $\sum_{i,j}(\lambda_i\mu_{ij})a_{ij}\in\ZZ[A]$ (simplified by
  collapsing terms with identical $a_{ij}$).

  In `usual' algebra, we have $A=\ZZ[G]/\ZZ[R]$, but we can take $G=A$
  and we can instead take~$R$ to be all relations of the form
  $1\cdot \sum\lambda_i a_i - \sum \lambda_i a_i$ (instead of just a
  small generating set), where the first sum is a single element
  of~$A$ and the second is an element of $\ZZ[A]$; we can write this
  relation as
  $1\cdot \alpha\bigl(\sum\lambda_i a_i\bigr) - \sum \lambda_i a_i$ to
  make this explicit.

  Now consider the element $1\cdot \sum\lambda_i a_i\in\ZZ[\ZZ[A]]$.
  We have
  $\ZZ[\alpha]\bigl(1\cdot \sum\lambda_i
  a_i\bigr)=1\cdot\alpha\bigl(\sum \lambda_i a_i\bigr)$ while
  $\mu_A\bigl(1\cdot \sum \lambda_i a_i\bigr)=\sum \lambda_i a_i$, and
  so $1\cdot \alpha\bigl(\sum\lambda_i a_i\bigr) - \sum \lambda_i a_i$
  is in the image of $\ZZ[\alpha]-\mu_A$.  As every element of
  $\ZZ[\ZZ[A]]$ is a linear combination of elements of the form
  $1\cdot \sum\lambda_i a_i\in\ZZ[\ZZ[A]]$, the image of
  $\ZZ[\alpha]-\mu_A$ will consist of linear combinations of elements
  of the form
  $1\cdot \alpha\bigl(\sum\lambda_i a_i\bigr) - \sum \lambda_i a_i$,
  which is $\ZZ[R]$.
\end{proof}

\end{document}


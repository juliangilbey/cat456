\documentclass[../../solutions]{subfiles}
\title{Template}
\author{}
\begin{document}
\maketitle

% \settheorem{1}{2}{3}
% \begin{lemma}
%   
% \end{lemma}
% \popthm

\setexercise{5}{5}{4}
\begin{exercise}
  For any group $G$, the forgetful functor $\Set^{\sB G}\to\Set$
  admits a left adjoint that sends a set~$X$ to the $G$-set
  $G\times X$, with $G$~acting on the left.  Prove that this
  adjunction is monadic by appealing to the monadicity theorem.
\end{exercise}

\begin{proof}
  We show that the forgetful functor $\func{U}{\Set^{\sB G}}{\Set}$ is
  strictly monadic by showing that it strictly creates coequalisers of
  $U$-split pairs.  Assume that $\pfunc{p,q}{X}{Y}$ is a $U$-split
  pair in $\Set^{\sB G}$, and that the underlying set functions admit
  a split coequaliser diagram
  $$
  \begin{tikzcd}
    X
    \ar[r, shift left=0.5ex, "p"]
    \ar[r, shift right=0.5ex, "q"']
    & Y
    \ar[r, twoheadrightarrow, "r"]
    \ar[l, bend left=50, "t"]
    & Z
    \ar[l, bend left=30, "s"]
  \end{tikzcd}
  $$

  For $Z$ to be a $G$-set, we need to define the action of $G$ on
  $Z$.  For each $g\in G$, we have the diagram
  $$
  \begin{tikzcd}
    X
    \ar[r, shift left=0.5ex, "p"]
    \ar[r, shift right=0.5ex, "q"']
    \ar[d, "g"']
    & Y
    \ar[r, twoheadrightarrow, "r"]
    \ar[d, "g"]
    & Z
    \ar[d, dashed, "\exists!\vphantom{g}"', "g\vphantom{!}"]
    \\
    X
    \ar[r, shift left=0.5ex, "p"]
    \ar[r, shift right=0.5ex, "q"']
    & Y
    \ar[r, twoheadrightarrow, "r"']
    & Z    
  \end{tikzcd}
  $$
  As both rows are coequaliser diagrams, there is a unique map
  $\func{g}{Z}{Z}$ induced.  By uniqueness, this gives a $G$-action
  on~$Z$ (i.e., $g\cdot(h\cdot z)=(gh)\cdot z$ and $1\cdot z=z$ for
  all $g,h\in G$).  This gives~$Z$ the structure of a $G$-set and also
  shows that $r$~is a $G$-set morphism.

  To show that this defines the coequaliser of $p$ and $q$ in
  $\Set^{\sB G}$, suppose that we have a fork 
  $
  \begin{tikzcd}
    X
    \ar[r, shift left=0.5ex, "p"]
    \ar[r, shift right=0.5ex, "q"']
    & Y
    \ar[r, "s"]
    & W
  \end{tikzcd}
  $
  in $\Set^{\sB G}$.  Then there is a unique factorisation in $\Set$:
  $$
  \begin{tikzcd}
    X
    \ar[r, shift left=0.5ex, "p"]
    \ar[r, shift right=0.5ex, "q"']
    & Y
    \ar[r, twoheadrightarrow, "r"]
    \ar[dr, "s"']
    & Z
    \ar[d, dashed, "\exists!\vphantom{j}"', "j\vphantom{!}"]
    \\
    && W
  \end{tikzcd}
  $$
  by the universal property of the coequaliser.  To show that $j$
  lifts to a morphism in $\Set^{\sB G}$, we must show that for each
  $g\in G$, we have $j\cdot g=g\cdot j$.  For each $y\in Y$, we have
  $$j(g\cdot(ry)) = j(r(g\cdot y)) = s(g\cdot y) = g\cdot(sy) =
  g\cdot(jry) = g\cdot(j(ry))$$
  so $j(g\cdot(ry))=g\cdot(j(ry))$ for all $y\in Y$.  As $r$~is
  surjective, it follows that $j(g\cdot z)=g\cdot(jz)$ for all $z\in
  Z$, so $j\cdot g=g\cdot j$ as required.
\end{proof}

\end{document}


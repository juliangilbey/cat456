\documentclass[../../solutions]{subfiles}
\title{Template}
\author{}
\begin{document}
\maketitle

% \settheorem{1}{2}{3}
% \begin{lemma}
%   
% \end{lemma}
% \popthm

\setexercise{5}{5}{3}
\begin{exercise}
  Prove Proposition 5.5.8.
\end{exercise}

\begin{proof}
  The proof of Theorem 5.5.1 (in the version given
  \hyperref[thm:5.5.1]{above}) only uses the assumption that
  $U$~creates coequalisers of $U$-split pairs in three places; we can
  replace these using the assumptions of this proposition.

  The first occasion is to show that for any algebra
  $(A,\alpha)\in\sC^T$, the parallel pair $F\alpha$, $\epsilon_{FA}$
  has a coequaliser in~$\sD$.  Consider the diagram
  $$
  \begin{tikzcd}
    FA
    \ar[r, "F\eta_A"]
    &
    FUFA
    \ar[r, shift left=0.5ex, "F\alpha"]
    \ar[r, shift right=0.5ex, "\epsilon_{FA}"']
    & FA
  \end{tikzcd}
  $$
  We have $\epsilon_{FA}\cdot F\eta_A=1_{FA}$ by the triangle
  identities for the adjunction $F\ladj U$, and $F\alpha\cdot F\eta_A=
  F1_A=1_{FA}$ by the algebra laws, so $F\alpha$, $\epsilon_{FA}$ is a
  reflexive pair.  By condition~(i) of the proposition, this pair has
  a coequaliser in~$\sD$ as required.

  The second use of the original assumption is that $U$~preserves
  $U$-split pairs, but this is only used for the same specific
  reflexive pair $F\alpha$, $\epsilon_{FA}$, so it follows that
  $U$~preserves the coequaliser of this pair by condition~(ii) of the
  proposition.

  The third occasion is at the end of the proof, where
  \hyperref[lemma:5.4.11]{Lemma 5.4.11} is used.  The pair considered
  is reflexive:
  $$
  \begin{tikzcd}
    FUD
    \ar[r, "F\eta_{UD}"]
    &
    FUFUD
    \ar[r, shift left=0.5ex, "FU\epsilon_D"]
    \ar[r, shift right=0.5ex, "\epsilon_{FUD}"']
    & FUD
  \end{tikzcd}
  $$
  with both composites being identities by the triangle identities,
  and so the conclusion of this lemma would follow if we can show that
  $U$~reflects coequalisers of reflexive pairs.  But $\sD$~has
  coequalisers of reflexive pairs and $U$~preserves them; $U$~also
  reflects isomorphisms, and so by a simple adaptation of Exercise
  3.3.iv, $U$~creates coequalisers of reflexive pairs, and in
  particular, reflects them.  (Explicitly, if $x\rightrightarrows y$
  is a reflexive pair in~$\sD$, then it has a coequaliser
  $x\rightrightarrows y\twoheadrightarrow z$ by~(i), whose image
  under~$U$ is a coequaliser by~(ii).  If $x\rightrightarrows y\to w$
  is a fork whose image under~$U$ is a coequaliser, then there is a
  unique morphism $\func{k}{z}{w}$ making the diagram on the left
  $$
  \begin{tikzcd}
    x
    \ar[r, shift left=0.5ex]
    \ar[r, shift right=0.5ex]
    & y
    \ar[r, twoheadrightarrow]
    \ar[dr]
    & z
    \ar[d, dashed, "\exists!\strut"', "k\strut"]
    \\
    && w
  \end{tikzcd}
  \qquad
  \begin{tikzcd}
    Ux
    \ar[r, shift left=0.5ex]
    \ar[r, shift right=0.5ex]
    & Uy
    \ar[r, twoheadrightarrow]
    \ar[dr, twoheadrightarrow]
    & Uz
    \ar[d, "Uk"]
    \\
    && Uw
  \end{tikzcd}  
  $$
  commute.  The image under~$U$ is shown on the right; as there is a
  unique morphism between the two coequalisers making this diagram
  commute, $Uk$~must be an isomorphism.  As $U$~reflects isomorphisms
  by~(iii), $k$~is an isomorphism, and hence $y\to w$ is a coequaliser
  of the original reflexive pair.)
\end{proof}

\end{document}

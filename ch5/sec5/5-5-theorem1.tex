\documentclass[../../solutions]{subfiles}
\title{Template}
\author{}
\begin{document}
\maketitle

\addcontentsline{toc}{subsection}{Theorem 5.5.1}%
\label{thm:5.5.1}
Following on from our discussion and modification of
\hyperref[def:5.4.8]{Definition 5.4.8} above, we present a proof of
Theorem 5.5.1 that closely follows the one presented in the text, but
that aligns better to the modified definition.  It also clarifies and
corrects the final paragraph of the argument.  The first two
paragraphs of the argument are almost unchanged, but for ease, they
are reproduced here.  We also incorporate Exercise 5.5.i into the
proof for simplicity.

\settheorem{5}{5}{1}
\begin{theorem}[monadicity theorem]
  A right adjoint functor $\func{U}{\sD}{\sC}$ is monadic
  (respectively, strictly monadic) if and only if it creates
  (respectively, strictly creates) coequalizers of $U$-split pairs.
\end{theorem}

\begin{proof}[Proof]
  \renewcommand*{\theequation}{%
    \thesection.\arabic{equation}%
  }
  Recalling Definition 5.4.8, the meaning of this result is the
  following.  Given an adjunction $F\ladj U$ inducing a monad~$T$
  on~$\sC$, Proposition 5.2.12 constructs the canonical comparison
  functor:
  $$
  \begin{tikzcd}[row sep=large, column sep=normal]
    \sD \ar[rr, "K"]
    \ar[dr, shift left=1ex, pos=0.5, "U"]
    \ar[dr, phantom, "\scriptstyle\top" {anchor=center, rotate=-45,
      inner sep=.5mm}]
    \ar[dr, shift right=1ex, leftarrow, pos=0.5, "F"']
    && \sC^T
    \ar[dl, shift left=1ex, pos=0.4, "U^T" {inner sep=0mm}]
    \ar[dl, phantom, "\scriptstyle\bot" {anchor=center, rotate=45,
      inner sep=.5mm}]
    \ar[dl, shift right=1ex, leftarrow, pos=0.4, "F^T"' {inner sep=0mm}]
    \\
    & \sC
  \end{tikzcd}
  $$
  Then by definition $U$ is monadic (respectively, strictly monadic)
  if $K$~is an equivalence (respectively, isomorphism) of categories.

  Corollary 5.4.10 (with the proof \hyperref[def:5.4.8]{above}) proves
  the implication (i)$\Rightarrow$(ii), so we assume~(ii) and use it
  to construct an inverse equivalence $\func{L}{\sC^T}{\sD}$ to the
  functor~$K$.  We have $U^TK=U$ and $KF=F^T$, so if~$L$ is to define
  an inverse equivalence to~$K$, we must have $U^T\cong UL$ and
  $F\cong LF^T$, conditions that we use to guide the definition
  of~$L$.
  
  In fact, we define $L$ so that the second of these conditions holds
  strictly, namely, for each $A\in\sC$, define
  $$L(TA,\mu_A) := FA$$
  and define $L$ to carry a free map
  $\func{Uf}{(TA,\mu_A)}{(TB,\mu_B)}$ between free algebras to the map
  $\func{f}{FA}{FB}$.  (This means, in particular, that $L$ carries a
  free map $Tf$ to~$Ff$.)  It remains to extend the definition of~$L$
  to non-free algebras and maps.

  An equivalence of categories preserves all limits and colimits.
  With this in mind, for any algebra $(A,\alpha)\in\sC^T$, consider
  the parallel pair $\pfunc{F\alpha,\epsilon_{FA}}{FUFA}{FA}$
  in~$\sD$.  By Example 5.4.7, this parallel pair is $U$-split, with
  the underlying fork of the split coequaliser diagram being
  \begin{equation}
    \label{eq:5-5-thm1-1}
    \begin{tikzcd}[column sep=huge]
      T^2A=UFUFA
      \ar[r, shift left=0.5ex, "UF\alpha=T\alpha"]
      \ar[r, shift right=0.5ex, "U\epsilon_{FA}=\mu_A"']
      & TA=UFA
      \ar[r, twoheadrightarrow, "\alpha"]
      & A
    \end{tikzcd}
  \end{equation}
  By the hypothesis that $U$ creates $U$-split coequalisers,
  $F\alpha$, $\epsilon_{FA}$ has a coequaliser in~$\sD$, and we define
  $\func{h_{(A,\alpha)}}{FA}{L(A,\alpha)}$ to be such a coequaliser:
  \begin{equation}
    \label{eq:5-5-thm1-2}
    \begin{tikzcd}
      FUFA
      \ar[r, shift left=0.5ex, "F\alpha"]
      \ar[r, shift right=0.5ex, "\epsilon_{FA}"']
      & FA
      \ar[r, twoheadrightarrow, "h_{(A,\alpha)}"]
      & L(A,\alpha)
    \end{tikzcd}
  \end{equation}
  This construction is arranged so that the coequaliser (5.4.4) is
  preserved by~$L$.  (In the strict case, we define
  $\func{h_{(A,\alpha)}}{FA}{L(A,\alpha)}$ to be the unique lift of
  the underlying fork of the coequaliser diagram to~$\sD$.)
  As $U$~preserves $U$-split pairs, the unique morphism $A\to
  UL(A,\alpha)$ making the diagram
  \begin{equation}
    \label{eq:5-5-thm1-3}
    \begin{tikzcd}
      UFUFA
      \ar[r, shift left=0.5ex, "UF\alpha"]
      \ar[r, shift right=0.5ex, "U\epsilon_{FA}"']
      & UFA
      \ar[r, twoheadrightarrow, "\alpha"]
      \ar[rd, "Uh_{(A,\alpha)}"']
      & A
      \ar[d, dashed, "\exists!"']
      \\
      && UL(A,\alpha)
    \end{tikzcd}
  \end{equation}
  commute is an isomorphism, and so
  $\func{Uh_{(A,\alpha)}}{UFA}{UL(A,\alpha)}$ is a coequaliser.  (In
  the strict case, $UL(A,\alpha)=A$ and this isomorphism is the
  identity.)

  The action of $\func{L}{\sC^T}{\sD}$ on a morphism
  $\func{f}{(A,\alpha)}{(B,\beta)}$ is defined to be the unique map
  between coequalisers induced by a commutative diagram of parallel
  pairs:
  $$
  \begin{tikzcd}[row sep=large]
    FUFA
    \ar[r, shift left=0.5ex, "F\alpha"]
    \ar[r, shift right=0.5ex, "\epsilon_{FA}"']
    \ar[d, "FUFf"']
    & FA
    \ar[r, twoheadrightarrow, "h_{(A,\alpha)}"]
    \ar[d, "Ff"]
    & L(A,\alpha)
    \ar[d, dashed, "\exists!"', "Lf"]
    \\
    FUFB
    \ar[r, shift left=0.5ex, "F\beta"]
    \ar[r, shift right=0.5ex, "\epsilon_{FB}"']
    & FB
    \ar[r, twoheadrightarrow, "h_{(B,\beta)}"]
    & L(B,\beta)
  \end{tikzcd}
  $$
  Uniqueness of factorisations through colimit cones implies that this
  definition is functorial; see Proposition 3.6.1.

  The functor $K$ carries \eqref{eq:5-5-thm1-2} to
  \begin{equation}
    \label{eq:5-5-thm1-4}
    \begin{tikzcd}
      (T^2A,\mu_{TA})
      \ar[r, shift left=0.5ex, "T\alpha"]
      \ar[r, shift right=0.5ex, "\mu_A"']
      & (TA,\mu_A)
      \ar[r, "Kh_{(A,\alpha)}"]
      & KL(A,\alpha)
    \end{tikzcd}
  \end{equation}
  since, writing $\epsilon'$ for the counit of $F^T\ladj U^T$, we have
  $K\epsilon=\epsilon' K$, and so
  $K\epsilon_{FA}= \epsilon'_{KFA}= \epsilon'_{F^TA}=
  \epsilon'_{(TA,\mu_A)}= \mu_A$.  Furthermore, $U^T$~sends this fork
  to the lower coequaliser in~\eqref{eq:5-5-thm1-3}, as $U=U^TK$.  As
  $U^T$ strictly creates coequalisers of $U^T$-split pairs,
  \eqref{eq:5-5-thm1-4}~is a coequaliser diagram and the unique map
  $\gamma_{(A,\alpha)}$ making the diagram
  $$
  \begin{tikzcd}
    (T^2A,\mu_{TA})
    \ar[r, shift left=0.5ex, "T\alpha"]
    \ar[r, shift right=0.5ex, "\mu_A"']
    & (TA,\mu_A)
    \ar[r, twoheadrightarrow, "Kh_{(A,\alpha)}"]
    \ar[rd, twoheadrightarrow, "\alpha"']
    & KL(A,\alpha)
    \ar[d, dashed, "\exists!"', "\gamma_{(A,\alpha)}"]
    \\
    && (A,\alpha)
  \end{tikzcd}
  $$
  commute is an isomorphism.  (In the strict case,
  $KL(A,\alpha)=(A,\alpha)$ and $\gamma_{(A,\alpha)}=1_{(A,\alpha)}$.)

  We now show that the components $\gamma_{(A,\alpha)}$ assemble into
  a natural isomorphism $\gamma\colon KL\cong 1_{\sC^T}$.  (In the
  strict case, we trivially have $KL=1_{\sC^T}$.)  Suppose
  $\func{f}{(A,\alpha)}{(B,\beta)}$.  We then have the diagram
  $$
  \begin{tikzcd}[row sep=large]
    &&& (A,\alpha)
    \ar[ddd, dashed, "\exists!"']
    \\
    (T^2A,\mu_{TA})
    \ar[r, shift left=0.5ex, "T\alpha"]
    \ar[r, shift right=0.5ex, "\mu_A"']
    \ar[d, "T^2f"']
    & (TA,\mu_A)
    \ar[r, twoheadrightarrow, "Kh_{(A,\alpha)}"']
    \ar[d, "Tf"]
    \ar[urr, twoheadrightarrow, "\alpha"]
    & KL(A,\alpha)
    \ar[d, "KLf"']
    \ar[ur, "\gamma_{(A,\alpha)}"']
    \\
    (T^2B,\mu_{TB})
    \ar[r, shift left=0.5ex, "T\beta"]
    \ar[r, shift right=0.5ex, "\mu_B"']
    & (TB,\mu_B)
    \ar[r, twoheadrightarrow, "Kh_{(B,\beta)}"]
    \ar[drr, twoheadrightarrow, "\beta"']
    & KL(B,\beta)
    \ar[dr, "\gamma_{(B,\beta)}"]
    \\
    &&& (B,\beta)
  \end{tikzcd}
  $$
  It is easy to show that
  $(\beta\cdot Tf)\cdot T\alpha = (\beta\cdot Tf)\cdot \mu_A$, so
  there is a unique arrow from the coequaliser (shown dashed)
  $\func{j}{(A,\alpha)}{(B,\beta)}$ with
  $\beta\cdot Tf=j\cdot \alpha$.  But $j=f$ satisfies this, as $f$~is
  an algebra morphism.  We are thus reduced to showing that the
  right-hand trapezium commutes.  We have
  $$f\cdot \gamma_{(A,\alpha)} \cdot Kh_{(A,\alpha)}
  = f\cdot\alpha = \beta\cdot Tf = \gamma_{(B,\beta)}\cdot
  Kh_{(B,\beta)} \cdot Tf = \gamma_{(B,\beta)}\cdot KLf \cdot
  Kh_{(A,a)}.$$
  Since $Kh_{(A,\alpha)}$ is an epimorphism, we obtain $f\cdot
  \gamma_{(A,\alpha)}= \gamma_{(B,\beta)}\cdot KLf$, and hence
  $\gamma$~is a natural transformation, proving that $KL\cong
  1_{\sC^T}$.

  It remains to prove that $LK\cong 1_\sD$ (and in the strict case,
  that $LK=1_\sD$).  Given $D\in\sD$, we have $KD=(UD,U\epsilon_D)$
  by the proof of Proposition \hyperref[prop:5.2.12]{5.2.12}, so the
  object $LKD$ is a coequaliser
  $$
  \begin{tikzcd}
    FUFUD
    \ar[r, shift left=0.5ex, "FU\epsilon_D"]
    \ar[r, shift right=0.5ex, "\epsilon_{FUD}"']
    & FUD
    \ar[r, twoheadrightarrow, "h_{KD}"]
    & LKD
  \end{tikzcd}
  $$
  But by \hyperref[lemma:5.4.11]{Lemma 5.4.11} (our rewritten variant
  of Corollary 5.4.10(ii)), this parallel pair has a coequaliser~$D$,
  so we have a unique isomorphism $\delta_D$ between coequalisers
  making this diagram commute:
  $$
  \begin{tikzcd}
    FUFUD
    \ar[r, shift left=0.5ex, "FU\epsilon_D"]
    \ar[r, shift right=0.5ex, "\epsilon_{FUD}"']
    & FUD
    \ar[r, twoheadrightarrow, "h_{KD}"]
    \ar[rd, twoheadrightarrow, "\epsilon_D"']
    & LKD
    \ar[d, dashed, "\exists!"', "\delta_D"]
    \\
    && D
  \end{tikzcd}
  $$
  (Note that Lemma 5.4.11 assumes that $U$~reflects coequalisers of
  $U$-split pairs.)  An almost identical diagram and argument to the
  above now shows that if $\func{g}{D}{D'}$, then
  $g\cdot \delta_D=\delta_{D'}\cdot LKg$ (though this time the dashed
  arrow equals~$g$ by naturality of~$\epsilon$), showing that the
  isomorphisms~$\epsilon_D$ assemble into a natural isomorphism
  $\delta\colon LK\cong 1_\sD$.  (In the case that $U$ strictly
  creates $U$-split pairs, our above definition gives $LKD=D$ and
  $h_{KD}=\epsilon_D$, so $\delta_D=1_D$ and $LK=1_\sD$.)
\end{proof}

\end{document}


\documentclass[../../solutions]{subfiles}
\title{Template}
\author{}
\begin{document}
\maketitle

% \settheorem{1}{2}{3}
% \begin{lemma}
%   
% \end{lemma}
% \popthm

\setexercise{5}{5}{6}
\begin{exercise}
  Describe a more general class of functors $\func{K}{\sI}{\sJ}$
  between small categories so that for any cocomplete $\sC$ the
  restriction functor $\func{\res_K}{\sC^\sJ}{\sC^\sI}$ strictly
  creates colimits of $\res_K$-split parallel pairs.  All such
  functors admit left adjoints and are therefore monadic.  Challenge:
  describe the left adjoint (or see Theorem 6.2.1).
\end{exercise}

\begin{proof}
  These functors admit left adjoints by Proposition 6.1.5 and Theorem
  6.2.1 (or Corollary 6.2.6).  We do not give a solution to the
  Challenge here.
  
  We claim that when $K$ is a bijection on objects, then the resulting
  adjunction is strictly monadic.  We suggest that this is the largest
  general class satisfying this condition.

  Note first that $\res_K=K^*$, so for any functor $F\in\sC^\sJ$,
  $\res_K F=FK$.

  Suppose that $\pfunc{\alpha,\beta}{F}{G}$ is a $\res_K$-split pair
  in~$\sC^\sJ$, so we have a split coequaliser diagram in~$\sC^\sI$
  with underlying fork
  $$
  \begin{tikzcd}
    FK
    \ar[r, shift left=0.5ex, "\alpha K"]
    \ar[r, shift right=0.5ex, "\beta K"']
    & GK
    \ar[r, twoheadrightarrow, "\gamma"]
    & H
  \end{tikzcd}
  $$
  Then at each $i\in\sI$, we can apply the evaluation functor $\ev_i$
  to get a coequaliser in~$\sC$:
  $$
  \begin{tikzcd}
    FKi
    \ar[r, shift left=0.5ex, "\alpha_{Ki}"]
    \ar[r, shift right=0.5ex, "\beta_{Ki}"']
    & GKi
    \ar[r, twoheadrightarrow, "\gamma_i"]
    & Hi
  \end{tikzcd}
  $$
  
  We must show that there is a unique way to lift the functor
  $\func{H}{\sI}{\sC}$ to a functor $\func{\hat H}{\sJ}{\sC}$ and
  $\Func{\gamma}{GK}{H}$ to~$\Func{\hat\gamma}{G}{\hat H}$; we will
  then have $\hat HK=H$ and $\hat\gamma K=\gamma$.  We must, in
  particular, define $\hat HKi=Hi$ and $\hat\gamma_{Ki}=\gamma_i$ for
  each $i\in\sI$.  If $K$~is not surjective, we do not have a unique
  way to define $\hat H$ on objects of~$\sJ$ in general, so we require
  $K$~to be surjective on objects.  If $K$~is not injective, we might
  have $Ki=Ki'$ but $Hi\ne Ki'$, making it impossible to define
  $\hat HKi$, so we require $K$~to be injective on objects.  In
  summary, we require $K$~to be bijective on objects.

  Therefore, for each $j\in\sJ$, we have $j=Ki$ for a unique
  $i\in\sI$, so $\hat Hj=Hi$ and $\hat\gamma_j=\gamma_i$; we can
  therefore rewrite the above fork as
  $$
  \begin{tikzcd}
    Fj
    \ar[r, shift left=0.5ex, "\alpha_j"]
    \ar[r, shift right=0.5ex, "\beta_j"']
    & Gj
    \ar[r, twoheadrightarrow, "\hat\gamma_i"]
    & \hat Hj
  \end{tikzcd}
  $$

  Arguing in the standard way, given $\func{f}{j}{j'}$ in $\sJ$, we
  have the diagram
  $$
  \begin{tikzcd}
    Fj
    \ar[r, shift left=0.5ex, "\alpha_j"]
    \ar[r, shift right=0.5ex, "\beta_j"']
    \ar[d, "Ff"']
    & Gj
    \ar[r, twoheadrightarrow, "\hat\gamma_j"]
    \ar[d, "Gf"]
    & \hat Hj
    \ar[d, dashed, "\exists!\vphantom{\hat Hf}"', "\hat Hf"]
    \\
    Fj'
    \ar[r, shift left=0.5ex, "\alpha_{j'}"]
    \ar[r, shift right=0.5ex, "\beta_{j'}"']
    & Gj'
    \ar[r, twoheadrightarrow, "\hat\gamma_{j'}"']
    & \hat Hj'
  \end{tikzcd}
  $$
  that we require to commute if $\hat\gamma$ is to be a natural
  transformation; there is a unique way to define $\hat Hf$ to achieve
  this.

  However, as we require $\hat HK=H$ to behave nicely on morphisms as
  well as on objects, we must also have $\hat HKg=Hg$ for every
  $\func{g}{i}{i'}$ in~$\sI$.  We have the diagram
  $$
  \begin{tikzcd}
    FKi
    \ar[r, shift left=0.5ex, "\alpha_{Ki}"]
    \ar[r, shift right=0.5ex, "\beta_{Ki}"']
    \ar[d, "FKg"']
    & GKi
    \ar[r, twoheadrightarrow, "\gamma_i"]
    \ar[d, "GKg"]
    & Hi
    \ar[d, "Hg"]
    \\
    FKi'
    \ar[r, shift left=0.5ex, "\alpha_{Ki'}"]
    \ar[r, shift right=0.5ex, "\beta_{Ki'}"']
    & GKi'
    \ar[r, twoheadrightarrow, "\gamma_{i'}"']
    & Hi'
  \end{tikzcd}
  $$
  which commutes by the naturality of $\gamma$.  Taking $j=Ki$,
  $j'=Ki'$ and $f=Kg$ shows that $\hat Hf=Hg$ as required.

  The argument showing that this fork defines a coequaliser in
  $\sC^\sJ$ is essentially identical to that of the previous exercise:
  supposing that we have a fork in~$\sC^\sJ$:
  $$
  \begin{tikzcd}
    F
    \ar[r, shift left=0.5ex, "\alpha"]
    \ar[r, shift right=0.5ex, "\beta"']
    & G
    \ar[r, "\hat\gamma"]
    \ar[dr, "\delta"']
    & \hat H
    \ar[d, dashed, "\hat\epsilon"]
    \\
    && L
  \end{tikzcd}
  $$
  the underlying fork (obtained by applying $\res_K=K^*$) has a unique
  morphism $\Func{\epsilon}{H}{LK}$ in~$\sC^\sI$; we need to show that
  this lifts to a unique morphism $\Func{\hat\epsilon}{\hat H}{L}$
  in~$\sC^\sJ$.  As $K$~is bijective on objects, we must define
  $\hat\epsilon_{Ki}=\epsilon_i$ for each $i\in\sI$.

  To show that this gives a morphism in~$\sC^\sJ$, let
  $\func{f}{j}{j'}$ be a morphism in~$\sJ$.  We have, using the
  naturality of $\hat\gamma$ and~$\delta$,
  $$Lf\cdot \hat\epsilon_j \cdot \hat\gamma_j = Lf\cdot \delta_j =
  \delta_{j'}\cdot Gf = \hat\epsilon_{j'} \cdot \hat\gamma_{j'} \cdot
  Gf = \hat\epsilon_{j'} \cdot \hat Hf \cdot \hat\gamma_j.$$
  But $\hat\gamma_j$ is an epimorphism, so
  $Kf\cdot \hat\epsilon_j = \hat\epsilon_{j'}\cdot \hat Hf$, showing
  that $\epsilon$~is natural, as required.
\end{proof}

\end{document}

\documentclass[../../solutions]{subfiles}
\title{Template}
\author{}
\begin{document}
\maketitle

\addcontentsline{toc}{subsection}{Lemma 5.5.10}%
\label{lemma:5.5.10}

\settheorem{5}{5}{10}
\begin{lemma}
  For any pullback diagram of monomorphisms, as displayed on the left,
  $$
  \begin{tikzcd}
    A'
    \ar[r, "g"]
    \ar[d, rightarrowtail, "a"']
    \ar[dr, phantom, "\lrcorner", at start]
    & B'
    \ar[d, rightarrowtail, "b"]
    \\
    A
    \ar[r, "f"']
    & B
  \end{tikzcd}
  \hspace{2cm}
  \begin{tikzcd}
    PB'
    \ar[r, "g^{-1}"]
    \ar[d, "b_*"']
    & PA'
    \ar[d, "a_*"]
    \\
    PB
    \ar[r, "f^{-1}"']
    & PA
  \end{tikzcd}
  $$
  the right hand square commutes.
\end{lemma}

The proof in the book makes assertions about various squares being
pullbacks, but it is unclear to me why this should be the case.  The
following argument, appearing on nLab\footnote{%
  \url{https://ncatlab.org/nlab/show/Beck-Chevalley+condition}, in the
  section on Images and preimages, accessed 15 September 2024}, is
more convincing, and also has the benefit that it does not require any
of the morphisms to be monic.  Here is the reworded lemma:

\begin{mlemma}
  For any pullback diagram, as displayed on the left,
  $$
  \begin{tikzcd}
    D
    \ar[r, "g"]
    \ar[d, "h"']
    \ar[dr, phantom, "\lrcorner", at start]
    & B
    \ar[d, "k"]
    \\
    A
    \ar[r, "f"']
    & C
  \end{tikzcd}
  \hspace{2cm}
  \begin{tikzcd}
    PB
    \ar[r, "g^{-1}"]
    \ar[d, "k_*"']
    & PD
    \ar[d, "h_*"]
    \\
    PC
    \ar[r, "f^{-1}"']
    & PA
  \end{tikzcd}
  $$
  the right hand square commutes.
\end{mlemma}

\begin{proof}[Proof]
  From Example 3.2.11 and Definition 3.2.3, we can take
  $D=\{(a,b)\in A\times B:f(a)=k(b)\}$, with $h$ and~$g$
  being the projections onto the first and second components
  respectively.

  Now we can write any set $X\in PB$ as the union of its elements:
  $X=\bigcup_{x\in X}\{x\}$.  Now inverse functions and direct image
  functions both preserve unions, which follows from Example 4.1.8, as
  both are left adjoints, and left adjoints preserve coproducts.  It
  therefore suffices to prove that the right-hand diagram commutes
  when $X=\{x\}\in PB$.  We can then calculate directly:
  \begin{align*}
    h_*g^{-1}(\{x\})
      &= h\bigl(\{(a,x):f(a)=k(x)\}\bigr) \\
      &= \{a:f(a)=k(x)\} \\
      &= f^{-1}(\{k(x)\}) \\
      &= f^{-1}k_*(\{x\}).
  \end{align*}
\end{proof}

\end{document}


\documentclass[../../solutions]{subfiles}
\title{Template}
\author{}
\begin{document}
\maketitle

% \settheorem{1}{2}{3}
% \begin{lemma}
%   
% \end{lemma}
% \popthm

\setexercise{5}{5}{5}
\begin{exercise}
  Generalizing Exercise 5.5.iv, for any small category $\sJ$ and any
  cocomplete category~$\sC$ the forgetful functor $\sC^\sJ \to
  \sC^{\ob\sJ}$ admits a left adjoint
  $\func{\Lan}{\sC^{\ob\sJ}}{\sC^\sJ}$ that sends a functor $F\in
  \sC^{\ob\sJ}$ to the functor $\Lan F\in \sC^\sJ$ defined by
  $$\Lan F(j) = \coprod_{x\in\sJ\vphantom{j}}\coprod_{\sJ(x,j)} Fx.$$
  \begin{enumerate}[label=(\roman*)]
  \item Define $\Lan F$ on morphisms in $\sJ$.
  \item Define $\Lan$ on morphisms in $\sC^{\ob\sJ}$.
  \item Use the Yoneda lemma to show that Lan is left adjoint to the
    forgetful (restriction) functor $\sC^\sJ\to \sC^{\ob\sJ}$.
  \item Prove that this adjunction is monadic by appealing to
    the monadicity theorem.
  \end{enumerate}
\end{exercise}

\begin{proof}
  For parts (i)--(iii), there is only one reasonable way to define the
  required morphisms, and part~(iii) shows that these are consistent
  in the sense that they make $\Lan$ a left adjoint to the forgetful
  functor $\func{U}{\sC^\sJ}{\sC^{\ob\sJ}}$.

  We also note that the forgetful functor $U$ can be written as $i^*$
  (so $UG=Gi$), where $i\colon \ob\sJ\hookrightarrow \sJ$ is the
  inclusion function; we will find it helpful to write this inclusion
  explicitly at various points in the following argument.

  \begin{enumerate}[label=(\roman*)]
  \item If $\func{f}{j}{j'}$ is a morphism in $\sJ$, we define
    $\Lan F(f)$ to be the unique morphism that makes the following
    diagram commute for every $y\in\sJ$ and $g\in\sJ(y,j)$:
    $$
    \begin{tikzcd}
      Fy \ar[r, "\iota_g"] \ar[d, equal]
      & \displaystyle \coprod_{\sJ(y,j)} Fy \ar[r, "\iota_y"]
      & \displaystyle \coprod_{x\in\sJ\vphantom{j}}\coprod_{\sJ(x,j)} Fx
      \ar[d, dashed, "\Lan F(f)"]
      \\
      Fy \ar[r, "\iota_{fg}"']
      & \displaystyle \coprod_{\sJ(y,j')} Fy \ar[r, "\iota_y"']
      & \displaystyle \coprod_{x\in\sJ\vphantom{j}}\coprod_{\sJ(x,j')} Fx
    \end{tikzcd}
    $$

  \item If $\Func{\lambda}{F}{F'}$ is a morphism in $\sC^{\ob\sJ}$, we
    define $\Func{\Lan\lambda}{\Lan F}{\Lan F'}$ by its components.
    For each $j\in\sJ$, define $(\Lan\lambda)_j$ to be the unique
    morphism making the following diagram commute for every $y\in\sJ$
    and $g\in\sJ(y,j)$:
    $$
    \begin{tikzcd}
      Fy \ar[r, "\iota_g"] \ar[d, "\lambda_y"']
      & \displaystyle \coprod_{\sJ(y,j)} Fy \ar[r, "\iota_y"]
      & \displaystyle \coprod_{x\in\sJ\vphantom{j}}\coprod_{\sJ(x,j)} Fx
      \ar[d, dashed, "(\Lan \lambda)_j"]
      \\
      F'y \ar[r, "\iota_g"']
      & \displaystyle \coprod_{\sJ(y,j)} F'y \ar[r, "\iota_y"']
      & \displaystyle \coprod_{x\in\sJ\vphantom{j}}\coprod_{\sJ(x,j)} F'x
    \end{tikzcd}
    $$
    As $\ob\sJ$ is a discrete category, it is trivial that these
    components assemble to give a natural transformation.

  \item \emph{The question asks us to use the Yolanda lemma to prove
      this result, but I cannot see how to do this, especially as we
      have explicity given definitions for the behaviour of $\Lan$ in
      the previous parts.  In the proof of the general result in
      Chapter~6, the Yolanda lemma is only mentioned twice: explicitly
      in the discussion prior to Proposition 6.1.5, but there it is
      not actually used (the natural transformation under
      consideration is just $\gamma^*\cdot K^*$, and this does not
      require the Yolanda lemma), and in Proposition 6.1.5 via
      Proposition 4.3.4.  But as we are defining the morphisms
      explicitly (in particular in part~(ii) above), we do not need
      this proposition.  There may, though, be a much better way to do
      the following!}

    \smallskip

    To show that $\func{\Lan}{\sC^{\ob\sJ}}{\sC^\sJ}$ is left adjoint
    to~$U$, we show that for every $F\in\sC^{\ob J}$ and
    $G\in\sC^\sJ$, we have
    $$\sC^\sJ(\Lan F,G)\cong \sC^{\ob\sJ}(F,UG)=\sC^{\ob\sJ}(F,Gi)$$
    and these isomorphisms are natural in $F$ and $G$.  This would
    also show that $\Lan F$ represents the functor
    $\func{\sC^{\ob\sJ}(F,-\cdot i)}{\sJ}{\Set}$, though we do not
    make use of that here.

    Given $\Func{\alpha^\sharp}{\Lan F}{G}$, define its transpose
    $\Func{\alpha^\flat}{F}{Gi}$ by
    $$
    \begin{tikzcd}
      \alpha^\flat_j\colon Fj \ar[r, "\iota_{1_j}"]
      & \displaystyle \coprod_{\sJ(j,j)} Fj \ar[r, "\iota_j"]
      & \displaystyle \coprod_{x\in\sJ\vphantom{j}}\coprod_{\sJ(x,j)} Fx
      \ar[r, "\alpha^\sharp_j"]
      & Gj
    \end{tikzcd}    
    $$
    Note that these components trivially assemble to give a natural
    transformation as $\ob\sJ$ is discrete.

    Conversely, given $\Func{\alpha^\flat}{F}{Gi}$, we define its
    transpose $\Func{\alpha^\sharp}{\Lan F}{G}$ by components: for
    each $j\in\sJ$, let $\alpha^\sharp_j$ be the unique morphism that
    makes the following diagram commute for each $y\in\sJ$ and
    $g\in\sJ(y,j)$:
    $$
    \begin{tikzcd}
      Fy \ar[r, "\iota_g"] \ar[d, "\alpha^\flat_y"']
      & \displaystyle \coprod_{\sJ(y,j)} Fy \ar[r, "\iota_y"]
      & \displaystyle \coprod_{x\in\sJ\vphantom{j}}\coprod_{\sJ(x,j)} Fx
      \ar[d, dashed, "\alpha^\sharp_j"]
      \\
      Gy
      \ar[rr, "Gg"']
      && Gj
    \end{tikzcd}
    $$
    (Note that the tempting alternative, sending
    $Fy\xrightarrow{Fg}Fj$, is not defined in general, as $F$~is a
    functor from~$\sC^{\ob\sJ}$.)  To show that the resulting
    $\alpha^\sharp$~is a natural transformation, suppose
    $\func{f}{j}{j'}$ is a morphism in~$\sJ$.  Then for each
    $y\in\sJ$, $g\in\sJ(y,j)$, we have a diagram
    $$
    \begin{tikzcd}
      Fy \ar[r, equal] \ar[d, "\iota_{fg}"']
      & Fy \ar[r, "\alpha^\flat_y"] \ar[d, "\iota_g"']
      & Gy \ar[dd, "Gg"'] \ar[ddd, bend left=30, "G(fg)"]
      \\
      \displaystyle \coprod_{\sJ(y,j')} Fy \ar[dd, "\iota_y"']
      & \displaystyle \coprod_{\sJ(y,j)} Fy \ar[d, "\iota_y"']
      \\
      & \Lan F(j) \ar[r, "\alpha^\sharp_j"] \ar[d, "\Lan F(f)"']
      & Gj \ar[d, "Gf"']
      \\
      \Lan F(j') \ar[r, equal]
      & \Lan F(j') \ar[r, "\alpha^\sharp_{j'}"']
      & Gj'
    \end{tikzcd}
    $$
    The left-hand rectangle commutes by the definition of $\Lan F(f)$,
    the upper-middle rectangle and outer boundary commute by the
    definition of~$\alpha^\sharp$, and the right-hand triangle
    commutes by the functoriality of~$G$.  It follows that $Gf\cdot
    \alpha^\sharp_j\cdot \iota_y\cdot \iota_g = \alpha^\sharp_{j'}
    \cdot \Lan F(f)\cdot \iota_y\cdot \iota_g$ for all $y$ and~$g$,
    and so $Gf\cdot \alpha^\sharp_j = \alpha^\sharp_{j'} \cdot \Lan
    F(f)$ by the uniqueness property of the coproduct $\Lan F(j)$.
    Hence $\alpha^\sharp$~is a natural transformation.

    We next show that these proposed transposes do indeed give a
    bijection.  Given $\alpha^\sharp\in\sC^\sJ(\Lan F,G)$, we require
    $(\alpha^\flat)^\sharp=\alpha^\sharp$.  Now
    $(\alpha^\flat)^\sharp_j$ is the unique morphism that makes the
    following diagram commute for all $y\in\sJ$, $g\in\sJ(y,j)$, where
    we have added one extra arrow for $\Lan F(g)$:
    $$
    \begin{tikzcd}
      Fy \ar[r, "\iota_g"] \ar[d, "\iota_{1_y}"']
      & \displaystyle \coprod_{\sJ(y,j)} Fy \ar[r, "\iota_y"]
      & \displaystyle \coprod_{x\in\sJ\vphantom{j}}\coprod_{\sJ(x,j)} Fx
      \ar[dr, dashed, "(\alpha^\flat)^\sharp_j"]
      \\
      \displaystyle \coprod_{\sJ(y,y)} Fy \ar[r, "\iota_y"']
      & \displaystyle \coprod_{x\in\sJ\vphantom{j}}\coprod_{\sJ(x,j)} Fx
      \ar[ur, "\Lan F(g)"']
      \ar[r, "\alpha^\sharp_y"']
      & Gy \ar[r, "Gg"']
      & Gj
    \end{tikzcd}
    $$
    Here, the composite
    $\alpha^\sharp_y\cdot \iota_y\cdot \iota_{1_y}$
    is~$\alpha^\flat_y$, and the left-hand rectangle commutes by the
    definition of $\Lan F(g)$.  Taking
    $(\alpha^\flat)^\sharp_j=\alpha^\sharp_j$ makes the right-hand
    square commute by the naturality of~$\alpha^\sharp$, so uniqueness
    shows that $(\alpha^\flat)^\sharp=\alpha^\sharp$ as required.

    In the opposite direction, given
    $\alpha^\flat\in\sC^{\ob\sJ}(F,Gi)$, we show that
    $(\alpha^\sharp)^\flat=\alpha^\flat$.  For each $j\in\sJ$, the top
    row of the following diagram gives~$(\alpha^\sharp)^\flat_j$:
    $$
    \begin{tikzcd}
      Fj \ar[r, "\iota_{1_j}"] \ar[d, "\alpha^\flat_j"']
      & \displaystyle \coprod_{\sJ(j,j)} Fj \ar[r, "\iota_j"]
      & \displaystyle \coprod_{x\in\sJ\vphantom{j}}\coprod_{\sJ(x,j)} Fx
      \ar[r, "\alpha^\sharp_j"]
      & Gj \ar[d, equal]
      \\
      Gj \ar[rrr, "G1_j"']
      &&& Gj
    \end{tikzcd}
    $$
    and this diagram commutes by the definition of $\alpha^\sharp$
    applied to the morphism $1_j\in\sJ(j,j)$, so
    $(\alpha^\sharp)^\flat_j=\alpha^\flat_j$ as required.  We
    therefore have a bijection $\sC^\sJ(\Lan F,G) \cong
    \sC^{\ob\sJ}(F,Gi)$ for each $F$ and~$G$.

    We now show that these bijections are natural in $F$ and $G$,
    using Lemma 4.1.3 to do so.  We therefore need to show that for
    any morphisms with domains and codomains as shown in the following
    diagrams:
    $$
    \begin{tikzcd}[arrows=Rightarrow]
      \Lan F \ar[r, "\alpha^\sharp"] \ar[d, "\Lan\lambda"']
      & G \ar[d, "\mu"]
      & F \ar[r, "\alpha^\flat"] \ar[d, "\lambda"']
      & Gi \ar[d, "\mu i"]
      \\
      \Lan F' \ar[r, "\beta^\sharp"']
      & G'
      & F' \ar[r, "\beta^\flat"']
      & G'i
    \end{tikzcd}
    $$
    the left-hand square commutes in $\sC^\sJ$ if and only if the
    right-hand square commutes in $\sC^{\ob\sJ}$.  These square
    commute if and only if the following squares commute for each
    $j\in\sJ$:
    \begin{equation}
      \label{eq:5-5-5-1}
      \begin{tikzcd}
        \Lan F(j) \ar[r, "\alpha^\sharp_j"] \ar[d, "(\Lan\lambda)_j"']
        & Gj \ar[d, "\mu_j"]
        & Fj \ar[r, "\alpha^\flat_j"] \ar[d, "\lambda_j"']
        & Gj \ar[d, "\mu_j"]
        \\
        \Lan F'(j) \ar[r, "\beta^\sharp_j"']
        & G'j
        & F'j \ar[r, "\beta^\flat_j"']
        & G'j
      \end{tikzcd}
    \end{equation}
    Suppose that the left-hand square commutes for all $j$.  Then for
    each~$j$, substituting the definitions of the transposes in the
    right-hand square and adding in an arrow for $(\Lan \lambda)_j$
    gives
    $$
    \begin{tikzcd}
      Fj \ar[r, "\iota_{1_j}"] \ar[d, "\lambda_j"']
      & \displaystyle \coprod_{\sJ(j,j)} Fj \ar[r, "\iota_j"]
      & \displaystyle \coprod_{x\in\sJ\vphantom{j}}\coprod_{\sJ(x,j)} Fx
      \ar[r, "\alpha^\sharp_j"]
      \ar[d, "(\Lan \lambda)_j"']
      & Gj \ar[d, "\mu_j"]
      \\
      F'j \ar[r, "\iota_{1_j}"']
      & \displaystyle \coprod_{\sJ(j,j)} F'j \ar[r, "\iota_j"']
      & \displaystyle \coprod_{x\in\sJ\vphantom{j}}\coprod_{\sJ(x,j)} F'x
      \ar[r, "\beta^\sharp_j"']
      & G'j
    \end{tikzcd}
    $$
    The left-hand rectangle in this diagram commutes by the definition of
    $(\Lan\lambda)_j$ and the right-hand square here commutes by
    assumption, so the right-hand square in~\eqref{eq:5-5-5-1} commutes.

    Conversely, assume that the right-hand square
    in~\eqref{eq:5-5-5-1} commutes for all~$j$.  Then for each
    $j\in\sJ$, for each $y\in\sJ$ and $g\in\sJ(y,j)$, we have a
    diagram
    $$
    \begin{tikzcd}
      Fy \ar[r, "\alpha^\flat_y"] \ar[d, "\iota_g"']
      \ar[ddddd, bend right=45, "\lambda_y"']
      & Gy \ar[dd, "Gg"]
      \ar[ddddd, bend left=45, "\mu_y"]
      \\
      \displaystyle \coprod_{\sJ(y,j)} Fy \ar[d, "\iota_y"']
      \\
      \Lan F(j) \ar[r, "\alpha^\sharp_j"] \ar[d, "(\Lan\lambda)_j"']
      & Gj \ar[d, "\mu_j"]
      \\
      \Lan F'(j) \ar[r, "\beta^\sharp_j"']
      & G'j
      \\
      \displaystyle \coprod_{\sJ(y,j)} F'y \ar[u, "\iota_y"]
      \\
      F'y \ar[r, "\beta^\flat_y"'] \ar[u, "\iota_g"]
      & G'y \ar[uu, "G'g"']
    \end{tikzcd}
    $$
    The left side commutes by the definition of $\Lan\lambda$ and the
    right side commutes by the naturality of~$\mu$.  The top and
    bottom rectangles commute by the definition of $\alpha^\sharp_j$
    and~$\beta^\sharp_j$ respectively, and the boundary commutes by
    assumption.  Therefore
    $\mu_j\cdot \alpha^\sharp_j\cdot \iota_y \cdot \iota_g =
    \beta^\sharp_j\cdot (\Lan \lambda)_j \cdot \iota_y \cdot \iota_g$
    for each $y$ and~$g$, and so the central square here, which is the
    left-hand square in~$\eqref{eq:5-5-5-1}$, commutes by the
    uniqueness property of the coproduct $\Lan F(j)$.

    Thus $\Lan$ is left adjoint to the forgetful functor $U$.

  \item We prove that this adjunction is strictly monadic using the
    monadicity theorem.  Suppose we have a parallel pair of morphisms
    $\pfunc{\alpha,\beta}{F}{G}$ in $\sC^\sJ$ whose images in
    $\sC^{\ob\sJ}$ extend to a split coequaliser diagram in
    $\sC^{\ob\sJ}$ with underlying fork
    $$
    \begin{tikzcd}
      UF
      \ar[r, shift left=0.5ex, "U\alpha"]
      \ar[r, shift right=0.5ex, "U\beta"']
      & UG
      \ar[r, twoheadrightarrow, "\gamma"]
      & H
    \end{tikzcd}
    $$
    We must show that there is a unique way to lift the functor
    $\func{H}{\ob\sJ}{\sC}$ to a functor $\func{\hat H}{\sJ}{\sC}$ so
    that $\gamma$~is a natural transformation in~$\sC^\sJ$, and
    moreover that the resulting fork defines a coequaliser
    in~$\sC^\sJ$.

    For $\gamma$ to define a natural transformation in $\sC^\sJ$, we
    require that for every morphism $\func{f}{j}{j'}$ in~$\sJ$, the
    diagram
    $$
    \begin{tikzcd}
      Gj \ar[r, "\gamma_j"] \ar[d, "Gf"']
      & \hat Hj \ar[d, dashed, "\hat Hf"]
      \\
      Gj' \ar[r, "\gamma_{j'}"']
      & \hat Hj'
    \end{tikzcd}
    $$
    commutes, where $\hat Hf$ is yet to be defined; note also that
    $\hat Hj=Hj$ for every $j\in\sJ$.  Since the fork in the split
    coequaliser diagram is an absolute colimit, it is preserved by the
    evaluation functor $\ev_j$.  Thus in the diagram
    $$
    \begin{tikzcd}
      Fj
      \ar[r, shift left=0.5ex, "\alpha_j"]
      \ar[r, shift right=0.5ex, "\beta_j"']
      \ar[d, "Ff"']
      & Gj
      \ar[r, twoheadrightarrow, "\gamma_j"]
      \ar[d, "Gf"]
      & Hj
      \ar[d, dashed, "\exists!\vphantom{f}"', "\hat Hf"]
      \\
      Fj'
      \ar[r, shift left=0.5ex, "\alpha_{j'}"]
      \ar[r, shift right=0.5ex, "\beta_{j'}"']
      & Gj'
      \ar[r, twoheadrightarrow, "\gamma_{j'}"']
      & Hj'
    \end{tikzcd}
    $$
    the top and bottom rows are coequalisers and we must define
    $\hat Hf$ to be the induced morphism.  Uniqueness ensures that the
    resulting $\func{\hat H}{\sJ}{\sC}$ is functorial.  We also
    observe that the image of~$\hat H$ under~$U$ is~$H$, as only the
    identity morphisms exist in $\ob\sJ$.
    
    To show that this fork defines a coequaliser in $\sC^\sJ$, suppose
    we have a fork in $\sC^\sJ$
    $$
    \begin{tikzcd}
      F
      \ar[r, shift left=0.5ex, "\alpha"]
      \ar[r, shift right=0.5ex, "\beta"']
      & G
      \ar[r, "\gamma"]
      \ar[dr, "\delta"']
      & \hat H
      \ar[d, dashed, "\epsilon"]
      \\
      && K
    \end{tikzcd}
    $$
    The underlying fork (obtained by applying $U$) has a unique
    morphism $\func{\epsilon}{H}{K}$ in $\sC^{\ob\sJ}$; we need to
    show that this gives a morphism (natural transformation)
    in~$\sC^\sJ$.  Let $\func{f}{j}{j'}$ be a morphism in~$\sJ$.  We
    have, using the naturality of $\gamma$ and~$\delta$,
    $$Kf\cdot \epsilon_j \cdot \gamma_j = Kf\cdot \delta_j =
    \delta_{j'}\cdot Gf = \epsilon_{j'} \cdot \gamma_{j'} \cdot Gf =
    \epsilon_{j'} \cdot \hat Hf \cdot \gamma_j.$$ But $\gamma_j$ is an
    epimorphism, so $Kf\cdot \epsilon_j = \epsilon_{j'}\cdot \hat Hf$,
    showing that $\epsilon$~is natural, as required.
  \end{enumerate}
\end{proof}

\end{document}

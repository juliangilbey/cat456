\documentclass{article}
\usepackage{cat456}
\title{Style Guide}
\author{Julian Gilbey}

% Matthew's hack...
\def\changemargin#1#2{\list{}{\rightmargin#2\leftmargin#1}\item[]}
\let\endchangemargin=\endlist

\begin{document}
\maketitle

This document is an addendum to
\href{https://github.com/madkous/math490/blob/master/style-guide/style-guide.tex}{Matthew
  Kousoulas's style guide}.  A PDF version of his guide is available
\href{https://github.com/juliangilbey/math490/blob/minor-latex-fixes/style-guide/style-guide.pdf}{here}.

\section*{Changes from Matthew's guide}

\begin{itemize}
\item The `Installation' section can be ignored.  (Also, the style
  file has been renamed as \verb+cat456.sty+ in this repository.)
\item `Preamble' section: the preambles have been pre-populated for
  ease.  Please don't add anything new before \verb+\begin{document}+
  as it will not be included in the main file.  If you would like to
  add a macro to the style file, include this in your pull request.
\item The `Naming Scheme' section can be ignored.
\end{itemize}

\section*{Additional formatting guidelines}

\begin{itemize}
\item Labels and references: \verb+\label{eqn1}+ defines the label
  \verb+eqn1+ throughout the entire set of solutions.  This will
  obviously cause problems if two different solutions use the same
  label name!  So we will always prefix label names with the exercise
  number, for example: \verb+\label{2-3-6-eqn1}+.  It's a bit of a
  pain, but will potentially save trouble later.
\end{itemize}

\section*{Examples}

Please read \verb+cat456.sty+ to see the full list of macros defined.
Here is a small sample of them.

\begin{changemargin}{-1in}{-1in}
  \begin{center}
    \begin{tabular}{l|L|l}
      Command & Output & Notes \\ \hline
      \verb|\sC, \sD, \sE, \dots| & \sC, \sD, \sE, \dots & generic categories \\
      \verb|\Top| & \Top & works in text mode and math mode\\
      \verb|\Setp| & \Setp & works in text mode and math mode\\
      \verb|\Mod{R}| & \Mod{R} & \\
      \verb|\zero, \dots, \nine, \com| & \zero, \dots, \nine, \com & \\
      \verb|\CC, \QQ, \ZZ| & \CC, \QQ, \ZZ & and a few others \\
      \verb|\func{f}{A}{B}| & \func{f}{A}{B} & functions\\
      \verb|\func{\inv{f}}{PB}{PA}| & \func{\inv{f}}{PB}{PA} & inverse functions\\
      \verb|\mono{g}{A}{B}| & \mono{g}{A_{/\ker{f}}}{B} & monomorphisms \\
      \verb|\epic{h}{A}{B}| & \epic{h}{A}{\range(A)} & epimorphisms \\
      \verb|\Func{F}{\sC^\op}{\cat{D}}| & \Func{F}{\sC^\op}{\cat{D}} & \\
      \verb|\pfunc{f,g}{x}{y}| & \pfunc{f,g}{x}{y} & \\
      \verb|\incl{\io}{S^1}{\RR^2}| & \incl{\io}{S^1}{\RR^2} & inclusions \\
      \verb|\idfunc{\NN}| & \idfunc{\NN} & identities \\
      \verb|\oper{+}{\RR}| & \oper{+}{\RR} & operations \\
      \verb|\angl{\lst{x}{n}}| & \angl{\lst{x}{n}} & optional argument changes the first index \\
      \verb|\lst[3]{x}{n}| & \angl{\lst[3]{x}{n}} & \\
      \verb|\fm{y}{\nil}| & \fm{y}{\nil} & optional argument changes the index \\
      \verb|\fm[j]{y}{\sJ}| & \fm[j]{y}{\sJ} & \\
      \verb|\el F| & \el F & category of elements \\
      \verb|F\ladj G, G\radj F| & F\ladj G, G\radj F & there are also rotated versions\\
      \verb|\nper{\norm{\cdot}}{\CC}{n}| & \nper{\norm{\cdot}}{\CC}{n} & $n$-ary operations \\
      \verb|\de\defeq\recip{\ep}| & \de\defeq\recip{\ep} \\
      \verb|\ZZ\fps{x}| & \ZZ\fps{x} & automatically inserts \verb1\left1 \& \verb1\right1 \\
      \verb|2\isinvb3| & 2\isinvb3 & everyone's favourite relation \\
      \verb|\qty{\frac{\len f}{\rtwo}}| & \qty{\frac{\len f}{\rtwo}} \\
      \verb|\inlnmat{\al&\be\\\gm&\de}| & \inlnmat{\al&\be\\\gm&\de} & useful for typesetting permutations \\
    \end{tabular}
  \end{center}
\end{changemargin}

\end{document}

